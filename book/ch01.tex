%%%%%%%%%%%%%%%%%%%%% chapter.tex %%%%%%%%%%%%%%%%%%%%%%%%%%%%%%%%%
%
% sample chapter
%
% Use this file as a template for your own input.
%
%%%%%%%%%%%%%%%%%%%%%%%% Springer-Verlag %%%%%%%%%%%%%%%%%%%%%%%%%%
%\motto{Use the template \emph{chapter.tex} to style the various elements of your chapter content.}
\chapter{Fundamental Concepts}
\label{intro} % Always give a unique label
% use \chaptermark{}
% to alter or adjust the chapter heading in the running head

\abstract{
  Below we discuss some fundamental structuring ideas at Socotra.
}

\section{The Simplified Schema}
\label{sec:01:1}
\import{ch01/}{TheSimplifiedSchema.tex}

\section{Fundamental Objects}
Of the objects in the simplified schema products, policies and modifications are the most common topics
of discussion. A $Product$ is merely a compound, tree structured definition object. It is mostly static,
configuration data, not very computationally exciting. A $Policy$ and its $Modification$ set is much more
interesting and dynamic. And at Socotra the two are used very much like the common functional list type
\begin{equation*}
  data \; List \; a = Nil | Cons \; a
\end{equation*}
To make the analogy more explicit, I will rewrite the above as
\begin{equation*}
  data \; Policy \; a = Init | Modification  \; a
\end{equation*}
The analogy is not quite exact but it brings out an important idea that the meat of what a $Policy$ is and
how it behaves during computation is determined by the time ordered list of modifications that it is made up
of.

Below I have added some additional notes, to give a more detailed, but still simplified, understanding of
the structure of policies and modifications.
\label{sec:01:2}
\import{ch01/}{Policy-sa.tex}

\section{The Basic State Model}
\label{sec:01:3}
\import{ch01/}{TheBasicStateModel.tex}



