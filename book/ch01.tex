\motto{Science is  a means whereby learning is  achieved, not by mere theoretical speculation on the one hand, nor
by the undirected accumulation of practical facts on the other, but rather by a motivated iteration between theory
and practice.}
\chapter{Theory}
\label{intro:01} % Always give a unique label
% use \chaptermark{}
% to alter or adjust the chapter heading in the running head

\abstract{
  Before we begin the process of discussing actual software, we consider how we might structure a financial contract
  in the insurance industry. What might be the fundamental operations on those contract and what properties should
  those operations satisfy.
}
\lstset{style=mystyle}

\section{Contracts}
\label{sec:01:1}
Consider a insurance contract, such as medical insurance. If we wanted describe such contracts in a general way
in code we might come up with an initial simple descriptive framework like that below:
\begin{lstlisting}
type Name = String

data Interval = Interval Integer Integer

data Contract = 
    SimpleContract Name Interval (M.Map String String) |
    CompoundContract Contract Contract |
    And Contract Contract |
    Zero
\end{lstlisting}
There would be a name that describes the contract. It would also be necessary to define the interval from the
date that the contract starts to the date that it ends. Strictly, this interval would be a closed open interval
with Integers representing Unix timestamps, $[from\_timestamp, to\_timestamp)$. Since we are trying to be
general also need a way to describe attributes of policies. For medical insurance, these attributes could
be numerical like co-pay amounts or deductible amounts, but they could also be non numerical attributes like
a customers primary care physician or information on family members covered by the contract. All of this information
I will collect as a $SimpleContract$ as shown above. I might also want to structure contracts in some way. For
instance, my medical insurance might be issued by an insurance company as two distinct contract, one for
domestic and one for international medical insurance. I could describe this contract as the contract
$ And domestic international $. We can get still fancier in how we structure contracts. Socotra likes to structure
its contracts like trees. The most general attributes and time intervals are specified at the root of the
contract tree and the details of the contract get more specific as one moves towards the leaves. Here is
an example of a empty contract structured in the Socotra style:
\begin{lstlisting}
peril :: Contract
peril = SimpleContract "collision" (Interval 1 100) M.empty

exposure :: Contract
exposure = SimpleContract "vehicle" (Interval 1 100) M.empty

policy :: Contract
policy = SimpleContract "auto" (Interval 1 100) M.empty

example :: Contract
example = CompoundContract policy (
            CompoundContract exposure (
                CompoundContract (And peril peril) Zero))
\end{lstlisting}
Don't worry about the details of policies, exposures, and perils right now, just note that the constructor $CompoundContract$
is sufficient to support any structuring need, tree or otherwise. Lastly, I have added a $Zero$ contract, this is nothing more
than what all customers start off with when they initiate a contract with the insurance company. It has not content, cost, or other observables
associated with it.

\section{Charges}
\label{sec:01:2}
In pricing of a contract it's necessary to have a way of delineating all of the charges associated with the contract. We can
define this idea as
\begin{lstlisting}
type Refundable = Bool

data Units = 
    Currency | 
    CurrencyPerMonth | 
    CurrencyPerYear | 
    CurrencyPerInstallment

data Charge= 
    Premium Interval Double Units Refundable |
    Tax Interval Double Units Refundable |
    Commission Interval Double Units Refundable |
    Fee Interval Double Units Refundable
\end{lstlisting}
The above is a minimal definition of some of the concepts that we will need to properly describe charges,
but it's sufficient to touch on some of the essential points. We will want to finely describe our charges, and
for each of charge type we will track the
interval over which the charge is assessed, some numerical value and its units, and lastly a Boolean value
indicating whether the charge is refundable or not. At the moment Socotra does not track what the numerical
amount of a charge represents in units or the refundability of the charge. So the definitions above are more
complete than in practice. That does not mean the concepts are optional however. Tracking the units and
refundability of charges is required to properly price policies in all their complexity.

\section{Observables}
\label{sec:01:3}
Now that we have our contract structures defined and the basic idea of charges we can start to compute
observables. These observables are any external, objective values which are of interest to both parties
to the contract. In a functional language observables will be calculated by evaluating recursively down
the structure of the contract using pattern matching. In an object oriented scenario one would accomplish
the same objective using a object or functional based visitor pattern. At Socotra with a Java backend, the
second approach is very roughly the one that is taken.

One important observable is $rate$ which is defined as
\begin{lstlisting}
rate :: Contract -> [Charge]
\end{lstlisting}
which takes a contract and calculates all of the charges associated with the contract. At socotra the $rate$
function takes the form of a function written in liquid or a plugin written in JavaScript. Other observables
are
\begin{lstlisting}
underwrite :: Contract -> (Bool, String)
\end{lstlisting}
to provide an underwriting decision and a possible rejection reason on the contract, and
\begin{lstlisting}
invoice :: Contract -> [Invoice]
\end{lstlisting}
which takes a contract and greedily generates all of the invoices that will be payable on the contract. The
greedy behavior of the $invoice$ function may seem contrived but Socotra actually generates invoice information
greedily and, as we will see later, we will use this behavior as a convenience in describing the necessary properties
of $invoice$ functions in general.

Before, we get there though, given what we have so far, we can make a few statements on the properties that we can
expect from out observable functions. Given contracts, c1, c2, c3, we expect that observables will be commutative and associative
\begin{eqnarray*}
Obs(c1 \, { ^\backprime}An{d^\backprime} \, c2) &  = & Obs(c2 \, { ^\backprime}An{d^\backprime} \, c1) \\
Obs(c1 \, { ^\backprime}An{d^\backprime} \, c2) \oplus Obs(c3) & = & Obs(c1) \oplus Obs(c2 \, { ^\backprime}An{d^\backprime} \, c3)
\end{eqnarray*}
We also want all observations to comply with the meaning of the $Zero$ contract
\begin{equation*}
Obs(Zero \, { ^\backprime}An{d^\backprime} \, c) = Obs(c) = Obs(c \, { ^\backprime}An{d^\backprime} \, Zero)
\end{equation*}

\section{Modifications}
\label{sec:01:4}
Once a policy has been created, it will continue to undergo changes over its lifetime. Managing these changes is a large part of
what the Socotra API project does. There are three base functions which comprise any modification. These are:
\begin{lstlisting}
extend :: Interval -> Contract -> Contract

reduce :: Interval -> Contract -> Contract

override :: Interval ->  Map String String -> Contract -> Contract
\end{lstlisting}
$extend$ moves the upper bound of the contract interval into the future maintaining the existing attributes of the contract. In
the Socotra API, this function is, in code and in conversation, variously referred to as create, renew, or reinstate. $reduce$ move
the upper bound of the contract interval to a new value, $t^\prime$, such that $contract.start \leq t^\prime < contract.end$.
In the Socotra API, this function is variously referred to as cancel or lapse. Lastly, the $override$ function should be roughly
associated with the Socotra concept of an endorsement. The meaning of $override$ is very close to that of relational override in mathematics.
Here is an example of its use:
\begin{eqnarray*}
\{make: GM, value: 5000\} & \\
& { ^\backprime}overrid{e^\backprime} \{make: Ford, model: F150, value: null\} \\
& = \{make: Ford, model: F150\}
\end{eqnarray*}
This example contains an update, an addition, and a deletion. And you will find that this simple example carries
over to endorsement code that you will eventually see, although in the actually code base the updates,
addition, and deletes are much more explicit.

You may rightly wonder, especially if you are familiar with the Socotra framework, why I have abstracted this
way over existing modifications like create, renew, endorse, cancel, and reinstate. I have done this point out
the compositional nature of modifications which are currently implemented as one off behaviors. Take for instance
a contract from $t_1$ to $t_3$ which we plan on endorsing with added attributes, while also changing the end date
to some $t_2 < t_3$. This operation can be expressed monolithicly and non reusably as
\begin{equation*}
endorse \: (Interval \: t_1 \: t_3) \: policy_{t_1}^{t_3} \: fields \: \{newEnd = t_2\}
\end{equation*}
or it can be expressed compositionally as
\begin{equation*}
reduce \: (Interval \: t_2 \: t_3) \: \$ \: override \: (Interval \: t_1 \: t_3) \: fields \: policy_{t_1}^{t_3}
\end{equation*}
There is all the difference in the world between writing monolithic functions for every modification and variation
there of, and writing three functions that achieve the same effect through functional composition.

Now that we have our basic modification combinators, I am going to use them to define which behaviors must be true
for our code to be correct. There is no particular order to the properties they are merely ideas which are important
or non obvious. I will state the properties in Socotra terminology and write the details in generalized terminology.
That way the idea will not get lost in abstraction.

\begin{description}

\item[Cancellation reinstatement inversion]
Under an observation, a contract that is canceled followed by a full reinstatement must
equal the original contract.
\begin{equation*}
extend \: (Interval \: t_2 \: t_3) \: \$ reduce \: (Interval \: t_2 \: t_3) \: c = c
\end{equation*}
If we designate the left hand side of the equation as $c^\prime$. Then we certainly want $rate \: c^\prime = rate \: c$
Invoicing is slightly more complicated and it turns out we would like to following to be true:
\begin{equation*}
\sum \{i.amount | i \in invoice c^\prime if \, i.status \neq writtenOff\} = \sum \{i.amount | i \in invoice c\}
\end{equation*}
The amount of all the invoices that have not been written off must be equal.

\item[Invoice invariance on cancellation]
When a partial cancellation happens on a policy, all invoiced periods that are fully within the remaining,
uncanceled range must not incur any modifications to their total amount or component amounts.
\begin{eqnarray*}
is & = & invoice \: contract_{t_1}^{t_3} \\
is^\prime & = & invoice \: \$ \: reduce \: (Interval \: t_2 \: t_3) \: contract_{t_1}^{t_3} \\
\{i \in is : within \: i \: (Interval \: t_1 \: t_2)\} & = & \{i^\prime \in is^\prime : within \: i^\prime (Interval \: t_1 \: t_2)\}
\end{eqnarray*}
This should make intuitive sense. If one has a monthly policy and its is canceled after one month, barring cancellation charges,
the amount calculated and owed for the first month should not change.

\item[Endorsement commutativity] Under observation, renewals and reinstatements commute with simple endorsements
\begin{eqnarray*}
c_{t_1}^{t_4} & = & extend \: (Interval \: t_3 \: t_4) \: \$ \: override \: (Interval \: t_2 \: t_3) \: fields \: c_{t_1}^{t_3} \\
            & = & override \: (Interval \: t_2 \: t_4) \: fields \: \$ \: extend \: (Interval \: t_3 \: t_4) \: c_{t_1}^{t_3}
\end{eqnarray*}
Endorsements also commute when the field maps of the two endorsements are disjoint or when the endorsement intervals are
disjoint.

\end{description}

One can get more involved in creating more intricate equalities, but from the simple examples the main points should be clear.
If a contract has a content a particular start and end dates. The modification path that got it to those start and end dates
often should not effects observables calculated on the contract. If a contract is reduced, often observables related to the
remaining part of the policy should not be affected.


Below I have added some additional notes, to give a more detailed, but still simplified, understanding of
the structure of policies and modifications.
\label{sec:01:2}
\import{ch01/}{Policy-sa.tex}


