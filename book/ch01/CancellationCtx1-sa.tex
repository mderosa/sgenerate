\tlatex
\setboolean{shading}{true}
\@x{}\moduleLeftDash\@xx{ {\MODULE} CancellationCtx1}\moduleRightDash\@xx{}%
\@x{ {\EXTENDS} Integers ,\, Policy}%
\begin{lcom}{7.5}%
\begin{cpar}{0}{F}{F}{0}{0}{}%
 In this specification I only model cancellations and renewals processes, but
 I can not
 ignore the effects of other concurrent endorsement or renewals that may
 happen during
 those processes. I model possible endorsements and renewals with the constant
 \ensuremath{EndorsementOrRenewal} which represents such a modification, in
 the accepted state.
\end{cpar}%
\end{lcom}%
\@x{ {\CONSTANTS} EndorsementOrRenewal}%
\@pvspace{8.0pt}%
\begin{lcom}{7.5}%
\begin{cpar}{0}{F}{F}{0}{0}{}%
The no config values indicates that a there is no cancellation configuration
 on the system and at that legacy cancellation logic should be called.
\end{cpar}%
\end{lcom}%
\@x{ NoConfig \.{\defeq} {\CHOOSE} x \.{:} x \.{\notin} {\BOOLEAN}}%
\@pvspace{8.0pt}%
\begin{lcom}{7.5}%
\begin{cpar}{0}{F}{F}{0}{0}{}%
A reinstatement is type of modification. A reinstatement starts at some time
 and reactivates a policy upto one of its previous end timestamps. A
 reinstatement has
 two specialized attributes one, specified in configuration, is the
 \ensuremath{auto\_reinstate}: boolean
 variable. If this variable is true for a reinstatement a payment recieved on
 a finalized
 reinstatement should cause the reinstatement to be issued. The second
 attribute is a
 \ensuremath{deadline\_ts} which is the last time at which the reinstatement
 can be issued.
\end{cpar}%
\end{lcom}%
\@x{ Reinstatements \.{\defeq} [}%
\@x{\@s{16.4} type \.{:} \{\@w{Reinstatement} \} ,\,}%
\@x{\@s{16.4} state \.{:} ModificationState ,\,}%
\@x{\@s{16.4} start\_timestamp \.{:} TimeRange ,\,}%
\@x{\@s{16.4} end\_timestamp \.{:} TimeRange ,\,}%
\@x{\@s{16.4} auto\_reinstate \.{:} {\BOOLEAN} ,\,}%
\@x{\@s{16.4} deadline\_ts \.{:} TimeRange ,\,}%
\@x{\@s{16.4} product\_revision \.{:} 0 \.{\dotdot} maxRevision}%
\@x{ ]}%
\@pvspace{8.0pt}%
\begin{lcom}{0}%
\begin{cpar}{0}{T}{F}{7.5}{0}{}%
 Its possible that different behaviors can be defined for different
 cancelation reasons,
 using liquid scripting. Inside \ensuremath{Java} \ensuremath{API} code, the
 cancellation reason, Lapsed, is used
 to mark cancellations that happen in the automated Grace-Lapse routines.
\end{cpar}%
\vshade{5.0}%
\begin{cpar}{1}{F}{F}{0}{0}{}%
\ensuremath{CancellationReasons \.{\defeq} \{\@w{Lapse},\, \@w{OtherReason}\}
}%
\end{cpar}%
\end{lcom}%
\@x{ CancellationReasons \.{\defeq} \{\@w{OtherReason} \}}%
\@pvspace{8.0pt}%
\begin{lcom}{7.5}%
\begin{cpar}{0}{F}{F}{0}{0}{}%
A cancellation is also a special type of modification. It is different from
 standard modifications in that it does not have a
 \ensuremath{end\_timestamp}. Though not modeled
 in this specification cancellations should record their cancellation reason.
\end{cpar}%
\end{lcom}%
\@x{ Cancellations \.{\defeq} [}%
\@x{\@s{16.4} type \.{:} \{\@w{Cancellation} \} ,\,}%
\@x{\@s{16.4} state \.{:} ModificationState ,\,}%
\@x{\@s{16.4} start\_timestamp \.{:} TimeRange ,\,}%
\@x{\@s{16.4} end\_timestamp \.{:} TimeRange ,\,}%
\@x{\@s{16.4} product\_revision \.{:} 0 \.{\dotdot} maxRevision}%
\@x{ ]}%
\@pvspace{8.0pt}%
\@x{}%
\@y{}%
\@xx{}%
\@x{ Invoice \.{\defeq} [}%
\@x{\@s{16.4} settlement\_type \.{:} {\BOOLEAN} ,\,}%
\@x{\@s{16.4} settlement\_status \.{:} {\BOOLEAN}}%
\@x{ ]}%
\@pvspace{8.0pt}%
\@x{}\bottombar\@xx{}%
\setboolean{shading}{false}
\begin{lcom}{0}%
\begin{cpar}{0}{F}{F}{0}{0}{}%
\ensuremath{\.{\,\backslash\,}}* Modification History
\end{cpar}%
\begin{cpar}{0}{F}{F}{0}{0}{}%
 \ensuremath{\.{\,\backslash\,}}* Last modified \ensuremath{Mon}
 \ensuremath{Jun} 29 17:59:33 \ensuremath{PDT} 2020 by \ensuremath{ASUS
}%
\end{cpar}%
\begin{cpar}{0}{F}{F}{0}{0}{}%
 \ensuremath{\.{\,\backslash\,}}* Last modified Sun \ensuremath{Jun} 28
 12:33:00 \ensuremath{PDT} 2020 by \ensuremath{marco
}%
\end{cpar}%
\begin{cpar}{0}{F}{F}{0}{0}{}%
 \ensuremath{\.{\,\backslash\,}}* \ensuremath{Created} Sat \ensuremath{Jun} 27
 21:12:19 \ensuremath{PDT} 2020 by \ensuremath{ASUS
}%
\end{cpar}%
\end{lcom}%
