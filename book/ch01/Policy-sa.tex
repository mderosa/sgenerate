\tlatex
\setboolean{shading}{true}
\@x{}\moduleLeftDash\@xx{ {\MODULE} Policy}\moduleRightDash\@xx{}%
\@x{ {\EXTENDS} Integers ,\, Product ,\, PolicyCtx}%
\@pvspace{8.0pt}%
\@x{ {\CONSTANT} NoModification}%
\@pvspace{8.0pt}%
\begin{lcom}{7.5}%
\begin{cpar}{0}{F}{F}{0}{0}{}%
 Socotra has several ways of generating invoices. When a modification is
 accepted,
 once a month, and, for premium reporting, on client request. For sufficient
 variety
 I have defined a incomplete set of the billing policy below.
\end{cpar}%
\end{lcom}%
 \@x{ BillingPolicy \.{\defeq} \{\@w{BySystemOnFinalize}
 ,\,\@w{ByClientOnRequest} \}}%
\@pvspace{8.0pt}%
\begin{lcom}{7.5}%
\begin{cpar}{0}{F}{F}{0}{0}{}%
 It is useful to see policies has having just two states. They start off in
 the \ensuremath{OffRisk
 } state and transition to \ensuremath{OnRisk} with their first issued
 modifications. Policies can
 also transition back to \ensuremath{OffRisk} when they are fully cancelled.
\end{cpar}%
\end{lcom}%
\@x{ PolicyState \.{\defeq} \{\@w{OnRisk} ,\,\@w{OffRisk} \}}%
\@pvspace{8.0pt}%
\begin{lcom}{7.5}%
\begin{cpar}{0}{F}{F}{0}{0}{}%
The linear lifecycle of a \ensuremath{Modification} from start to issue is:
\end{cpar}%
\begin{cpar}{0}{F}{F}{0}{0}{}%
(a) Created - Drafted
\end{cpar}%
\begin{cpar}{0}{F}{F}{0}{0}{}%
(b) Accepted - \ensuremath{Finalized
}%
\end{cpar}%
\begin{cpar}{0}{F}{F}{0}{0}{}%
 (c) Issued. These are the common states that are usually discussed at
 \ensuremath{Socotr} along with
 their synonyms.
\end{cpar}%
\end{lcom}%
 \@x{ ModificationState \.{\defeq} \{\@w{Created} ,\,\@w{Finalized}
 ,\,\@w{Issued} ,\,\@w{Expired} ,\,\@w{Invalidated} \}}%
\@pvspace{8.0pt}%
\@x{ TimeRange \.{\defeq} policyMinTs \.{\dotdot} policyMaxTs}%
\@pvspace{8.0pt}%
\begin{lcom}{7.5}%
\begin{cpar}{0}{F}{F}{0}{0}{}%
 There are more modification types than are listed below. Consider this listed
 just
 a sample of commonly discussed modification
\end{cpar}%
\end{lcom}%
 \@x{ ModificationType \.{\defeq} \{\@w{Endorsement} ,\,\@w{Renewal}
 ,\,\@w{Cancellation} \}}%
\@pvspace{8.0pt}%
\begin{lcom}{7.5}%
\begin{cpar}{0}{F}{F}{0}{0}{}%
 Modifcation all have the basic structure below. It is often convinient to
 think of
 them as an elaborated time interval and the action of applying a
 modification to
 a policy is in some respects a kind of interval addition.
\end{cpar}%
\end{lcom}%
\@x{ Modification \.{\defeq} [}%
\@x{\@s{16.4} type \.{:} ModificationType ,\,}%
\@x{\@s{16.4} state \.{:} ModificationState ,\,}%
\@x{\@s{16.4} start\_timestamp \.{:} TimeRange ,\,}%
\@x{\@s{16.4} end\_timestamp \.{:} TimeRange ,\,}%
\@x{\@s{16.4} product\_revision \.{:} 0 \.{\dotdot} maxRevision}%
\@x{\@s{16.4} ]}%
\@pvspace{8.0pt}%
\@x{ InvModification ( mod ) \.{\defeq}}%
\@x{\@s{16.4} \.{\land} mod . start\_timestamp \.{\leq} mod . end\_timestamp}%
\@pvspace{8.0pt}%
\begin{lcom}{7.5}%
\begin{cpar}{0}{F}{F}{0}{0}{}%
 Surprisingly, policies are not a particularly important structure in our
 specifications
 they are convinient for holding summary information. Fields like
 \ensuremath{revision\_start\_ts
 } and \ensuremath{renewal\_start\_timestamp} are examples of such summary
 information. They facilitate
 specification tasks later on and in fact dont correspond to any structures
 in our
 software implementation.
 On the other hand the \ensuremath{pending\_modification} field is used to
 represent the fact that
 a policy can have either no modifications in the accepted state or it can
 have
 one modification in the accepted state. This is an important constraint on
 the
 operation of our software at the moment. It is also exactly an specific case
 of the
 general concept of pessimistic locking used to achieve before and after
 atomicity.
\end{cpar}%
\end{lcom}%
\@x{ Policy \.{\defeq} [}%
\@x{\@s{16.4} original\_start\_ts \.{:} TimeRange ,\,}%
\@x{\@s{16.4} effective\_end\_ts \.{:} TimeRange ,\,}%
\@x{\@s{16.4} billing\_policy \.{:} BillingPolicy ,\,}%
\@x{\@s{16.4} state \.{:} PolicyState ,\,}%
 \@x{\@s{16.4} revision\_start\_timestamps \.{:} {\SUBSET} ( ( 0 \.{\dotdot}
 maxRevision ) \.{\times} TimeRange ) ,\,}%
\@x{\@s{16.4} renewal\_start\_timestamps \.{:} {\SUBSET} ( TimeRange ) ,\,}%
 \@x{\@s{16.4} pending\_modification \.{:} Modification \.{\cup} \{
 NoModification \}}%
\@x{\@s{16.4} ]}%
\@pvspace{8.0pt}%
\@x{}\bottombar\@xx{}%
\setboolean{shading}{false}
\begin{lcom}{0}%
\begin{cpar}{0}{F}{F}{0}{0}{}%
\ensuremath{\.{\,\backslash\,}}* \ensuremath{Modification} History
\end{cpar}%
\begin{cpar}{0}{F}{F}{0}{0}{}%
 \ensuremath{\.{\,\backslash\,}}* Last modified \ensuremath{Mon}
 \ensuremath{Jul} 13 20:22:48 \ensuremath{PDT} 2020 by \ensuremath{ASUS
}%
\end{cpar}%
\begin{cpar}{0}{F}{F}{0}{0}{}%
 \ensuremath{\.{\,\backslash\,}}* Last modified \ensuremath{Thu}
 \ensuremath{Mar} 05 13:20:45 \ensuremath{PST} 2020 by \ensuremath{marcderosa
}%
\end{cpar}%
\begin{cpar}{0}{F}{F}{0}{0}{}%
 \ensuremath{\.{\,\backslash\,}}* \ensuremath{Created} \ensuremath{Wed}
 \ensuremath{Feb} 05 14:05:18 \ensuremath{PST} 2020 by \ensuremath{marcderosa
}%
\end{cpar}%
\end{lcom}%
