
At Socotra we like to draw large complex state machines depicting the lifecycle of a $Policy$. There is really no need
for such extravagances, though, as the system can be succinctly and accurately described using the
standard notation of hierarchical state diagrams. Figure 1.2 shows a state machine for a set of modifications. The vertical
dotted line indicates that each $Modification$ in the set is operating in parallel with other modifications in the set. The
$*$ appended to each $Modification$ type indicates that there can be zero or many modifications of any type. In the
specifications ahead we often abstract further on this idea of many modifications by letting a policy's initial
modification set consist of the universe of all $Drafted$ modifications. Surprisingly this allows us to reduce the
state space in our developments without loss of generality.

Lastly in Figure 1.3 we have a hierarchical state machine for a $Policy$. This diagram is probably
a bit surprising if one is used to thinking casually about policies. Yet on examination you will
find that it is correct and brings out two important points.
\begin{itemize}
\item It is a bit sloppy to describe a $Policy$ as being in the $Accepted$ or some other state.
  Most of the time when we do that we should be talking about modifications.
\item The state changes of significance for policies happen on $onIssue$ events. We can put this statement in
  a real world context by saying equivalently that the state changes of significance are those that
  correspond to both sides signing off on the contractual terms.
\end{itemize}

\begin{figure}[b]
  \begin{tikzpicture}[->,>=stealth']

    \filldraw
    (-3,-0.5) circle node[align=left, below](tl){} --
    (-2,-0.5) circle node[align=left, below](tltext){Creation*\\ Endorsement*\\ Reinstatement* \\ Renewal*} --
    (3,-0.5) circle node[align=left, below](tm){} --
    (4,-0.5) circle node[align=left, below](tmtext){Cancellation*} --
    (9,-0.5) circle node[align=left, below](tr){} --
    (9,-7) circle node[align=left, below] (br) {} --
    (3, -7) circle node[align=left, below] (bm) {} --
    (-3, -7) circle node[align=left, below] (bl) {} --
    (-3,-0.5) circle node[align=left, below] (end) {};

    \draw[-, dotted] (tm) -- (bm);

    \node[circle,
      fill=black
    ](BeginA) {};

    \node[state,
      below of=BeginA,
      node distance=2.25cm
    ](DraftedA) {
      \begin{tabular}{l}
        \textbf{Drafted}
      \end{tabular}
    };

    \node[state,
      below of=DraftedA,
      node distance=2cm
    ] (AcceptedA) {
      \begin{tabular}{l}
        \textbf{Accepted}
      \end{tabular}
    };

    \node[state,
      below of=AcceptedA,
      node distance=2cm
    ] (IssuedA) {
      \begin{tabular}{l}
        \textbf{Issued}
      \end{tabular}
    };

    \node[circle,
      fill=black,
      right of=BeginA,
      node distance=6cm
    ](BeginB) {};

    \node[state,
      below of=BeginB,
      node distance=2.25cm
    ](DraftedB) {
      \begin{tabular}{l}
        \textbf{Drafted}
      \end{tabular}
    };

    \node[state,
      below of=DraftedB,
      node distance=2cm
    ] (IssuedB) {
      \begin{tabular}{l}
        \textbf{Issued}
      \end{tabular}
    };  

    \path
    (BeginA) edge[bend left=20] node[anchor=left,right]{$onCreate$} (DraftedA)
    (DraftedA) edge[loop right] node[anchor=left,right]{$onQuote$}(DraftedA)
    (DraftedA) edge[loop left] node[anchor=right,left]{$onUpdate$}(DraftedA)
    (DraftedA) edge[bend left=20] node[anchor=left,right]{$onAccept$} (AcceptedA)
    (AcceptedA) edge[bend left=20] node[anchor=right,left]{$onInvalidate$} (DraftedA)
    (AcceptedA) edge[bend left=20] node[anchor=left,right]{$onIssue$} (IssuedA)
    (BeginB) edge[bend left=20] node[anchor=left,right]{$onCreate$} (DraftedB)
    (DraftedB) edge[bend left=20] node[anchor=left,right]{$onIssue$} (IssuedB);
  
  \end{tikzpicture}
  \caption{
    The Socotra $Modification$ state model. Any number of modifications can exist
    concurrently in the modifications set. Each modification transitions between
    its states, isolated from the effects of other modifications.
  }
  \label{fig:2}
\end{figure}

\begin{figure}[b]
  \begin{tikzpicture}[->,>=stealth']

    \filldraw
    (-3,0) circle node[align=left, below](tl){} --
    (8,0) circle node[align=left, below](tr){} --
    (8,-6) circle node[align=left, below] (br) {} --
    (-3, -6) circle node[align=left, below] (bl) {} --
    (-3,0) circle node[align=left, below] (end) {};


    \node[state,
      text width=2cm,
      yshift=-3cm,
      xshift=-0.25cm,
      anchor=center
    ] (OffRisk) {
      \begin{tabular}{l}
        \textbf{OffRisk}
      \end{tabular}
    };

    \node[state,    	% layout (defined above)
      text width=2cm, 	% max text width
      right of=OffRisk, 	% Position is to the right of Product
      node distance=5cm,  % distance to Product
      anchor=center       % posistion relative to the center of the 'box'
    ] (OnRisk) {
      \begin{tabular}{l}
        \textbf{OnRisk}
      \end{tabular}
    };

    \node[circle,
      fill=black,
      above of=OffRisk,
      xshift=-2cm
    ](Begin) {};

    \path
    (OffRisk) edge[bend left=60] node[anchor=south,above]{$onIssue[Create]$} (OnRisk)
    (OnRisk) edge[loop above] node[anchor=south, above] {$onIssue[Endorse | Renew | Cancel | Reinstate]$} (OnRisk)
    (OnRisk) edge[bend left=60] node[anchor=north,below]{$onIssue[Cancel]$} (OffRisk)
    (Begin) edge[bend left=30] (OffRisk)
    ;
    
  \end{tikzpicture}
  \caption{
    The Socotra $Policy$ state model. A $Policy$ starts out as $OffRisk$ and its effective
    state changes whenever an $onIssue$ event fires.
  }
  \label{fig:3}
\end{figure}    
