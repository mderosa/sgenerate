
\begin{figure}[h]
\begin{tikzpicture}[->,>=stealth']

  \node[state, text width=2cm] (Product) {
    \begin{tabular}{l}
      \textbf{Product}
    \end{tabular}
  };

  \node[state,    	% layout (defined above)
    text width=4cm, 	% max text width
    right of=Product, 	% Position is to the right of Product
    node distance=6.5cm,  % distance to Product
    anchor=center       % posistion relative to the center of the 'box'
  ] (ProductConfig) {
    \begin{tabular}{l}
      \textbf{ProductConfig}
    \end{tabular}
  };

  \node[state,    	% layout (defined above)
    text width=4cm, 	% max text width
    xshift=1cm,
    below of=ProductConfig, 	% Position is to the right of Product
    node distance=1.4cm,
    anchor=center       % posistion relative to the center of the 'box'
  ] (ProductExposureConfig) {
    \begin{tabular}{l}
      \textbf{ProductExposureConfig}
    \end{tabular}
  };

  \node[state,    	% layout (defined above)
    text width=4cm, 	% max text width
    xshift=1cm,
    below of=ProductExposureConfig, 	% Position is to the right of Product
    node distance=1.4cm,
    anchor=center       % posistion relative to the center of the 'box'
  ] (ProductPerilConfig) {
    \begin{tabular}{l}
      \textbf{ProductPerilConfig}
    \end{tabular}
  };

  \node[state,    	% layout (defined above)
    text width=2cm,
    below of=Product,
    node distance=4.5cm,  % distance to Product
    anchor=center
  ] (Policy) {
    \begin{tabular}{l}
      \textbf{Policy}
    \end{tabular}
  };

  \node[state,
    text width=4cm,
    right of=Policy,
    node distance=5.5cm,
    anchor=center
  ] (PolicyCharacteristics) {
    \begin{tabular}{l}
      \textbf{PolicyCharacteristics}
    \end{tabular}
  };

  \node[state,
    text width=5cm,
    xshift=1cm,
    below of=PolicyCharacteristics,
    node distance=1.5cm,
    anchor=center
  ] (PolicyExposureCharacteristics) {
    \begin{tabular}{l}
      \textbf{PolicyExposureCharacteristics}
    \end{tabular}
  };

  \node[state,
    text width=4cm,
    xshift=1cm,
    below of=PolicyExposureCharacteristics,
    node distance=1.5cm,
    anchor=center
  ] (PolicyPerilCharacteristics) {
    \begin{tabular}{l}
      \textbf{PolicyPerilCharacteristics}
    \end{tabular}
  };

  \node[state,
    xshift=0.5cm,
    below of=Policy,
    node distance=4cm,
    anchor=center
  ] (Modifications) {
    \begin{tabular}{l}
      \textbf{Modifications}\\
      \parbox{3cm}{
        \begin{IEEEeqnarray*}{Cl}
          =& Creation \\
          |& Endorsement \\
          |& Renewal \\
          |& Cancellation \\
          |& Reinstatement
        \end{IEEEeqnarray*}
      }\\[4em]
    \end{tabular}
  };

  \path (ProductConfig) edge node[anchor=south,above]{$defines\;attributes\;of$} (Product)
  (ProductExposureConfig) edge[bend left=20] node[anchor=left,right]{$subgroup$} (ProductConfig)
  (ProductPerilConfig) edge[bend left=20] node[anchor=left,right]{$subgroup$} (ProductExposureConfig)
  (Policy)             edge[bend right=20] node[anchor=north,below]{$current$} (PolicyCharacteristics)
  (Policy) edge node[anchor=left,right]{$has\;associated$} (Product)
  (Policy) edge node[anchor=left,right]{$versioned\;to$} (ProductConfig)
  (PolicyCharacteristics) edge[bend right=20] node[anchor=south,above]{$stores\;values\;of$} (Policy)
  (PolicyExposureCharacteristics) edge[bend left=20] node[anchor=left,right]{$subgroup$} (PolicyCharacteristics)
  (PolicyPerilCharacteristics) edge[bend left=20] node[anchor=left,right]{$subgroup$} (PolicyExposureCharacteristics)
  (Modifications) edge node[anchor=left,right]{$audit\;history\;of$} (Policy);

\end{tikzpicture}
\caption{A simple, summary of the Socotra object model}
\label{fig:1}
\end{figure}

The Socotra object model is built around the concepts of products and policies. A $Product$ at its
simplest is a set of $(Field, Type)$ tuples. These $(Field, Type)$ tuples define the data that are
valid attributes of an instance of a $Product$. A $Policy$ is then an instance of this $Product$ definition where,
for each $(Field, Type)$ in the $Product$ definition, there will be a corresponding $(Field, Value)$ in the
Policy instance. Both of these sets, the $(Field, Type)$ set and the $(Field, Value)$ set, are then
grouped for organization into $Exposure$s and $Peril$s, as shown in Figure 1.1. Note that $(Field, Type)$ elements
are called configuration elements and $(Field, Value)$ elements are called $Characteristics$.

Another concept in the Socotra object model is that policies are formed by a series of modifications. A
policy starts empty. Its first issued $Modification$ is a creation $Modification$ and further modifications
can follow from there in any logical order. A $Policy$ can best be thought of as having two parts, a summary state
which is part of the $Policy$ proper, plus a detail state consisting of all issued modifications in their
time order.
