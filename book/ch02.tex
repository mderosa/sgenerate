\motto{Seek not to have things happen as you choose them, but rather choose them to happen as they do.}
\chapter{Cancellation and Reinstatement}
\label{intro} % Always give a unique label
% use \chaptermark{}
% to alter or adjust the chapter heading in the running head

\abstract{During the course of a policy's lifetime it may be necessary to cancel a policy because invoices have not been paid or for other
  business reasons. These same policies may then need to be restored once the business reasons that caused the cancellation have been resolved.
  The frameworks cancellation and reinstatement functionality effects these policy changes.
}

\section{Structure of Cancellations and Reinstatements}
\label{sec:02:1}
In our discussion of policies so far we dealt with the modifications
\begin{equation*}
  \{Creation, Endorsement, Renewal\}.
\end{equation*}
To this set we now add the modifications
\begin{equation*}
  \{Cancellation, Reinstatement\}.
\end{equation*}
These new modifications are similar to modifications we already
know, with slight differences. Reinstatements are different from previous modifications in that they have two unique
fields, auto\_reinstate and deadline\_timestamp, which we will describe shortly. Cancellations differ from previous
modifications in that their state lifecycle only has the two events onCreate and onIssue. Below is an abstraction which
describes the structural elements of cancellation and reinstatement data items.

\import{ch02/}{CancellationCtx1-sa.tex}

\section{State Machine}

With some understanding of the data in our two new modifications, we below, lay out actions on our policy in the usual
state centric way. 

\import{ch02/}{CancellationMch1-sa.tex}

\section{Interaction with other Modifications}
The development of quotes for Endorsement and Renewals and the development of Cancellation and Reinstatements functionality
partially overlapped and so it turns out that the two functionalities interact in a less than perfectly way.

Before we can discuss potential interactions of the two functionalities, I will quickly review the essentials of quotes. The
purpose of quotes is to price a potential modification given a policy in some state. And the way that given policy state is described
colloquially is as ``a policy where the last applied modifications is x'', or slightly more technically ``a policy with basis x''. The
framework realization of both of these informal descriptions is a field in both the endorsement and
renewal tables called quoted\_policy\_basis which holds a policy modification locator. The quote code would then, conceptually, have
mechanism to allow a quoted endorsement or renewal to progress to the accepted state only when
$ modification.quote\_policy\_base = lastModification(policy).locator $. Now in
fact the actual implementation is not as uniform and explicit as I have described, but the brief explanation here is still a good way
to conceptualize and growing complexity may force uniformity and explicitness in the long term anyways.

The challenge for cancellation and reinstatements is then to prevent situations where a quote is made, some arbitrary cancellation
reinstatement sequence changes the policy and then the quote becomes accepted and issued. There are three mechanisms that
prevent this occurrence that have been added to the cancellation / reinstatment state machine. First whenever a policy is
changed by cancellation or reinstatement we invalidate all policy quotes. By definition if the policy has changed and the quote
could not anticipate that change then the basis of the quote must be then be invalid. Second, if a policy has any outstanding
cancellations then no renewals can be created. This prevents any renewals from effecting the policy until all cancellation
renewal sequences are completed. Lastly endorsement quotes are not permitted when a reinstatement is in the accepted state. This
quotes from being made on policies that are about to change.

