%%%%%%%%%%%%%%%%%%%%% chapter.tex %%%%%%%%%%%%%%%%%%%%%%%%%%%%%%%%%
%
% sample chapter
%
% Use this file as a template for your own input.
%
%%%%%%%%%%%%%%%%%%%%%%%% Springer-Verlag %%%%%%%%%%%%%%%%%%%%%%%%%%
%\motto{Use the template \emph{chapter.tex} to style the various elements of your chapter content.}
\chapter{Cancellation and Renewal}
\label{intro} % Always give a unique label
% use \chaptermark{}
% to alter or adjust the chapter heading in the running head

\abstract{During the course of a policy's lifetime it may be necessary to cancel a policy because invoices have not been paid or for other
  business specific reasons. These same policies may then need to be restored to their state before the cancellation. The cancellation and
  renewal module in the Socotra system handles these policy changes. \newline\indent
  The module has two distinct components which will be discussed below.
  There is an API component which customers use to trigger cancellations or reinstatements. There is also an automated system which can cancel
  a policy or reinstate it when propositions on the policy state become true.
}

\section{Overview}
\label{sec:02:1}
In our discussion of policies so far we dealt with the modifications
\begin{equation*}
  \{Creation, Endorsement, Renewal\}.
\end{equation*}
To this set we now
add the modifications
\begin{equation*}
  \{Cancellation, Reinstatement\}.
\end{equation*}
These new modifications are similiar to modifications we already
know, with slight differences. Reinstatements are different in that they have two unique fields, described below, and Cancellations
are different in that their state lifecycle only has the two events onCreate and onIssue. Below is the general context in which
we can start to model the cancellation and renewal module.

\import{ch02/}{CancellationCtx1-sa.tex}

With the new modifications modeled we can now lay out actions on our policy in the usual state centric way. Cancellations with have
onCreate and onIssue events that they respond to. And Reinstatements respond to onCreate, onAccept and onIssue events. 

\import{ch02/}{CancellationMch1-sa.tex}

\section{Liquid Calculations}
\label{sec:02:2}

\import{ch02/}{CancellationCalculations-sa.tex}
