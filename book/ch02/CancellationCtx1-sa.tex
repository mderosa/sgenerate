\tlatex
\setboolean{shading}{true}
\@x{}\moduleLeftDash\@xx{ {\MODULE} CancellationCtx1}\moduleRightDash\@xx{}%
\@x{ {\EXTENDS} Integers ,\, Policy}%
\begin{lcom}{7.5}%
\begin{cpar}{0}{F}{F}{0}{0}{}%
 In this specification I only model cancellations and renewals processes, but
 I can not
 ignore the effects of other concurrent endorsement or renewals that may
 happen during
 those processes. I model possible endorsements and renewals with the constant
 \ensuremath{EndorsementOrRenewal} which represents such a modification, in
 the accepted state.
\end{cpar}%
\end{lcom}%
\@x{ {\CONSTANTS} EndorsementOrRenewal}%
\@pvspace{8.0pt}%
\begin{lcom}{7.5}%
\begin{cpar}{0}{F}{F}{0}{0}{}%
A reinstatement is type of modification. A reinstatement starts at some time
 and reactivates a policy upto the end timestamp of the cancellation it
 reinstates. A
 reinstatement has two specialized attributes. One, specified in
 configuration, is the
 \ensuremath{auto\_reinstate}: boolean variable. If this variable is true for
 a reinstatement a payment
 recieved on a finalized reinstatement should cause the reinstatement to be
 issued. The
 second attribute is a \ensuremath{deadline\_timestamp} which is the last
 time at which the reinstatement
 can be issued. The deadline timestamp can be specified in configuration or
 it can be
 specified in \ensuremath{Api.reinstatement\_draft()} and
 \ensuremath{Api.reinstatement\_update()} request. If the
 value is never specified it defaults to infinity.
\end{cpar}%
\end{lcom}%
\@x{ Reinstatements \.{\defeq} [}%
\@x{\@s{16.4} type \.{:} \{\@w{Reinstatement} \} ,\,}%
\@x{\@s{16.4} state \.{:} ModificationState ,\,}%
\@x{\@s{16.4} start\_timestamp \.{:} TimeRange ,\,}%
\@x{\@s{16.4} end\_timestamp \.{:} TimeRange ,\,}%
\@x{\@s{16.4} auto\_reinstate \.{:} {\BOOLEAN} ,\,}%
\@x{\@s{16.4} deadline\_ts \.{:} TimeRange ,\,}%
\@x{\@s{16.4} product\_revision \.{:} 0 \.{\dotdot} maxRevision}%
\@x{ ]}%
\@pvspace{8.0pt}%
\begin{lcom}{7.5}%
\begin{cpar}{0}{F}{F}{0}{0}{}%
 A cancellation is also a special type of modification. In the concrete
 implementation
 cancellations have an additional attribute, reason, commented out below. The
 reason is
 one of many user configuration reason strings which the customer uses to
 descriminate
 the purpose of a cancellation. Additionally the system will alway define a
 define a
 cancellation reason of ``lapse'' if such a reason does not exist. The lapse
 reason is
 the reason given by the system to cancellations which result from invoices
 remaining
 unpaid past their grace period. Lapse cancellations are created by the
 scheduled
 lapse routines that run in the background of the \ensuremath{API} application
\end{cpar}%
\end{lcom}%
\@x{ Cancellations \.{\defeq} [}%
\@x{\@s{16.4} type \.{:} \{\@w{Cancellation} \} ,\,}%
\@x{\@s{16.4} state \.{:} ModificationState ,\,}%
\@x{\@s{16.4} start\_timestamp \.{:} TimeRange ,\,}%
\@x{\@s{16.4} end\_timestamp \.{:} TimeRange ,\,}%
\@x{\@s{16.4}}%
\@y{\@s{0}%
 reason: String
}%
\@xx{}%
\@x{\@s{16.4} product\_revision \.{:} 0 \.{\dotdot} maxRevision}%
\@x{ ]}%
\@pvspace{8.0pt}%
\begin{lcom}{7.5}%
\begin{cpar}{0}{F}{F}{0}{0}{}%
Cancellations and reinstatements, like endorsements and renewals, are recorded
 as they are created in the history of policy modifications. Some of these
 modifications
 will take effect, by being issued, others will never be issued. In all
 cases, though,
 modifications retain an identity to which documents and invoices can be
 attached.
 The characteristic of the modifications, which gives them a group
 commonality, is
 that they can operate to change liability of the parties to the contract
\end{cpar}%
\end{lcom}%
\@x{}\bottombar\@xx{}%
\setboolean{shading}{false}
\begin{lcom}{0}%
\begin{cpar}{0}{F}{F}{0}{0}{}%
\ensuremath{\.{\,\backslash\,}}* Modification History
\end{cpar}%
\begin{cpar}{0}{F}{F}{0}{0}{}%
 \ensuremath{\.{\,\backslash\,}}* Last modified \ensuremath{Tue}
 \ensuremath{Dec} 15 15:53:25 \ensuremath{PST} 2020 by \ensuremath{ASUS
}%
\end{cpar}%
\begin{cpar}{0}{F}{F}{0}{0}{}%
 \ensuremath{\.{\,\backslash\,}}* Last modified \ensuremath{Thu}
 \ensuremath{Jul} 16 16:33:53 \ensuremath{PDT} 2020 by \ensuremath{marco
}%
\end{cpar}%
\begin{cpar}{0}{F}{F}{0}{0}{}%
 \ensuremath{\.{\,\backslash\,}}* Created Sat \ensuremath{Jun} 27 21:12:19
 \ensuremath{PDT} 2020 by \ensuremath{ASUS
}%
\end{cpar}%
\end{lcom}%
