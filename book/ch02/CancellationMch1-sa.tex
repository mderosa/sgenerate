\tlatex
\setboolean{shading}{true}
\@x{}\moduleLeftDash\@xx{ {\MODULE} CancellationMch1}\moduleRightDash\@xx{}%
\@x{ {\EXTENDS} CancellationCtx1 ,\, Policy ,\, Sequences}%
\@pvspace{8.0pt}%
 \@x{ {\CONSTANTS} policyTi ,\, policyTf ,\, reinstatementExpireDays ,\,
 Documents}%
\@pvspace{8.0pt}%
\@x{ {\VARIABLES} policy ,\, step ,\, mods ,\, term}%
\begin{lcom}{7.5}%
\begin{cpar}{0}{F}{F}{0}{0}{}%
 Here, I model the total state by separating it into the variables policy and
 \ensuremath{mods}.
 The \ensuremath{mods} variable is a sequential record of the issued
 modifications associated with the
 policy. Any accepted modifications associated with the policy are placed in
 the policy\mbox{'}s
 \ensuremath{pending\_modification} field. The choice to have a
 \ensuremath{pending\_modification} field on the
 policy variable is driven by the need to express the invariant that only one
 modification
 can be in the Accepted state at any time.
\end{cpar}%
\vshade{5.0}%
\begin{cpar}{0}{F}{F}{0}{0}{}%
A cancellation modification follows the state sequence:
 \ensuremath{Init \.{\mm}(create)}--\ensuremath{\.{>} Created
 \.{\mm}(issue)}--\ensuremath{\.{>} Issued} \.{\dots}
 A reinstatement modification follows one of the state sequences:
\end{cpar}%
\begin{cpar}{0}{F}{F}{0}{0}{}%
 \ensuremath{Init \.{\mm}(create)}--\ensuremath{\.{>} Created
 \.{\mm}(accept)}--\ensuremath{\.{>}} Accepted
 \ensuremath{\.{\mm}(issue)}--\ensuremath{\.{>} Issued} \.{\dots}
\end{cpar}%
\begin{cpar}{0}{F}{F}{0}{0}{}%
 \ensuremath{Init \.{\mm}(create)}--\ensuremath{\.{>} Created
 \.{\mm}(accept)}--\ensuremath{\.{>}} Accepted
 \ensuremath{\.{\mm}(invalidate) \.{\rightarrow}} Draft \.{\dots}
\end{cpar}%
\begin{cpar}{0}{F}{F}{0}{0}{}%
 \ensuremath{Init \.{\mm}(create)}--\ensuremath{\.{>} Created
 \.{\mm}(accept)}--\ensuremath{\.{>}} Accepted
 \ensuremath{\.{\mm}(expire)}--\ensuremath{\.{>}} Expired
\end{cpar}%
\vshade{5.0}%
\begin{cpar}{0}{F}{F}{0}{0}{}%
 Conceptually, this development takes the view that only accept and issue
 events of
 essential to understanding state evolution, this view point reduces the
 state space
 without loss of generality.
\end{cpar}%
\end{lcom}%
\@x{}\midbar\@xx{}%
\@x{}%
\@y{\@s{0}%
 Functions
}%
\@xx{}%
\@x{ prevMod \.{\defeq} \.{\LET} last \.{\defeq} Len ( mods )}%
 \@x{\@s{56.61} \.{\IN} {\IF} last \.{=} 0 \.{\THEN} NoModification \.{\ELSE}
 mods [ last ]}%
\@pvspace{8.0pt}%
\@x{ pendingReinstatementExpired ( p ) \.{\defeq}}%
\@x{\@s{16.4} \.{\LET} pending \.{\defeq} p . pending\_modification}%
 \@x{\@s{16.4} \.{\IN} pending \.{\in} Reinstatements \.{\land} pending .
 deadline\_ts \.{<} step}%
\@pvspace{8.0pt}%
\begin{lcom}{0}%
\begin{cpar}{0}{T}{F}{7.5}{0}{}%
 Just a semantic addition for the reader. When this function appears in a
 guard, the
\end{cpar}%
\begin{cpar}{1}{F}{F}{0}{0}{}%
 reader should interpret it a a requirement to check for and invalidate any
 accepted
 modifications
\end{cpar}%
\end{lcom}%
\@x{ invalidate ( mod ) \.{\defeq} {\TRUE}}%
\@pvspace{8.0pt}%
\@x{}\midbar\@xx{}%
\@x{}%
\@y{\@s{0}%
 Invariants
}%
\@xx{}%
\@x{}%
\@y{\@s{0}%
 Below we define a few common invariants of the system.
}%
\@xx{}%
\@pvspace{8.0pt}%
\@x{}%
\@y{\@s{0}%
 Check a few variable fields to ensure that types are correct
}%
\@xx{}%
\@x{ InvType \.{\defeq}}%
\@x{\@s{16.4} \.{\LET} pending \.{\defeq} policy . pending\_modification}%
 \@x{\@s{16.4} \.{\IN} \.{\land} pending \.{\in} {\UNION} \{ Reinstatements
 ,\,}%
\@x{\@s{131.90} Cancellations ,\,}%
\@x{\@s{131.90} \{ NoModification ,\, EndorsementOrRenewal \}}%
\@x{\@s{131.90} \}}%
\@x{\@s{36.79} \.{\land} policy . state \.{\in} PolicyState}%
\@pvspace{8.0pt}%
\begin{lcom}{7.5}%
\begin{cpar}{0}{F}{F}{0}{0}{}%
Policy Invariants are:
\end{cpar}%
\begin{cpar}{0}{T}{F}{2.5}{0}{}%
(1) a policy should have zero or one pending modifications.
\end{cpar}%
\begin{cpar}{0}{F}{F}{0}{0}{}%
 (2) pending modifications must be in the \ensuremath{Finalized} (Accepted)
 state.
\end{cpar}%
\begin{cpar}{0}{F}{F}{0}{0}{}%
 (3) if a policy is \ensuremath{OffRisk} then (a) its \ensuremath{start\_dt}
 must equal its \ensuremath{end\_dt} (b) the most
 recently issued modification must be a cancellations (c) there must be
 either no
 pending modifications or one of type \ensuremath{Reinstatment}.
 (4) all pending reinstatements must have a \ensuremath{deadline\_dt \.{<}}
 now.
\end{cpar}%
\end{lcom}%
\@x{ InvPolicy \.{\defeq}}%
\@x{\@s{16.4} \.{\LET} pending \.{\defeq} policy . pending\_modification}%
\@x{\@s{16.4} \.{\IN} \.{\land} policy . original\_start\_ts \.{=} policyTi}%
 \@x{\@s{36.79} \.{\land} pending \.{\notin} \{ NoModification ,\,
 EndorsementOrRenewal \} \.{\implies}}%
\@x{\@s{181.66} pending . state \.{=}\@w{Finalized}}%
\@x{\@s{36.79} \.{\land} policy . state \.{=}\@w{OffRisk} \.{\implies} (}%
 \@x{\@s{56.11} policy . original\_start\_ts \.{=} policy . effective\_end\_ts
 )}%
 \@x{\@s{36.79} \.{\land} policy . state \.{=}\@w{OffRisk} \.{\implies}
 prevMod \.{\in} Cancellations}%
 \@x{\@s{36.79} \.{\land} policy . state \.{=}\@w{OffRisk} \.{\implies}
 pending\@s{3.62} \.{\in} Reinstatements \.{\cup} \{ NoModification \}}%
\@x{\@s{36.79} \.{\land} ( \.{\land} pending \.{\in} Reinstatements}%
 \@x{\@s{52.01} \.{\land} pending . auto\_reinstate ) \.{\implies} pending .
 deadline\_ts \.{+} 1 \.{\geq} step}%
\@pvspace{8.0pt}%
\begin{lcom}{7.5}%
\begin{cpar}{0}{F}{F}{0}{0}{}%
Modification Invariants are:
 (1) the first policy modification has a \ensuremath{start\_dt} equals to the
 policy\mbox{'}s original
 policy start dt
\end{cpar}%
\begin{cpar}{0}{F}{F}{0}{0}{}%
(2) all modifications have \ensuremath{start\_dt \.{<} end\_dt
}%
\end{cpar}%
\begin{cpar}{0}{F}{F}{0}{0}{}%
(3) once a cancellation is issued that modification can only be followed by a
 cancellation or reinstatement. All issued cancellations must have a
 matching, following
 reinstatement before any endorsements or renewals can be issued
 (4) a cancellation and its matching reinstatement must have equal
 \ensuremath{end\_dts
}%
\end{cpar}%
\end{lcom}%
\@x{ InvMods \.{\defeq}}%
 \@x{\@s{16.4} \.{\LET}\@s{0.86} pending \.{\defeq} policy .
 pending\_modification}%
\@x{\@s{16.4} \.{\IN} \.{\land} prevMod \.{\neq} NoModification \.{\implies}}%
 \@x{\@s{56.11} prevMod . start\_timestamp \.{\geq} policy .
 original\_start\_ts}%
\@pvspace{8.0pt}%
 \@x{\@s{36.79} \.{\land} \A\, idx \.{\in} 1 \.{\dotdot} Len ( mods ) \.{:} (
 mods [ idx ] . start\_timestamp \.{<} mods [ idx ] . end\_timestamp )}%
 \@x{\@s{36.79} \.{\land} \A\, idx \.{\in} 2 \.{\dotdot} Len ( mods ) \.{:}
 mods [ idx ] \.{\in} Reinstatements \.{\implies}}%
 \@x{\@s{67.43} ( \.{\land} mods [ idx\@s{0.21} \.{-} 1 ] \.{\in}
 Cancellations}%
 \@x{\@s{71.53} \.{\land} mods [ idx \.{-} 1 ] . end\_timestamp \.{=} mods [
 idx ] . end\_timestamp )}%
\@pvspace{8.0pt}%
\@x{}\midbar\@xx{}%
\begin{lcom}{7.5}%
\begin{cpar}{0}{F}{F}{0}{0}{}%
 As long as a policy is \ensuremath{OnRisk}, not fully canceled, endorsements
 or renewals for the
 policy may transition into an accepted state, creating a pending modification
\end{cpar}%
\end{lcom}%
\@x{ acceptGeneric \.{\defeq}}%
\@x{\@s{16.4} \.{\land} step \.{<} policyMaxTs}%
 \@x{\@s{16.4} \.{\land} policy . state \.{=}\@w{OnRisk} \.{\land} policy .
 pending\_modification \.{=} NoModification}%
 \@x{\@s{16.4} \.{\land} policy \.{'} \.{=} [ policy {\EXCEPT} {\bang} .
 pending\_modification \.{=} EndorsementOrRenewal ]}%
\@x{\@s{16.4} \.{\land} step \.{'} \.{=} step \.{+} 1}%
\@x{\@s{16.4} \.{\land} {\UNCHANGED} {\langle} mods ,\, term {\rangle}}%
\@pvspace{8.0pt}%
\begin{lcom}{7.5}%
\begin{cpar}{0}{F}{F}{0}{0}{}%
 A cancellation can be created for any policy that is \ensuremath{OnRisk}. The
 start date specified by
 the cancellation should be within the coverage time span of the policy.
 Note that for the specification, cancellation create events do not result in
 additions
 to the policy\mbox{'}s modifications set. This addition does happen in the
 concrete
 implementation, however, we elide the effect here as maintaining the state
 in the policy
 record along with a list of issued modifications is sufficient to validate
 system
 properties.
\end{cpar}%
\end{lcom}%
\@x{ createCancellation ( mod ) \.{\defeq}}%
\@x{\@s{16.4} \.{\land} step \.{<} policyMaxTs}%
\@x{\@s{16.4} \.{\land} policy . state \.{=}\@w{OnRisk}}%
 \@x{\@s{16.4} \.{\land} mod . type \.{=}\@w{Cancellation} \.{\land} mod .
 state \.{=}\@w{Created}}%
 \@x{\@s{16.4} \.{\land} mod . start\_timestamp \.{<} policy .
 effective\_end\_ts}%
 \@x{\@s{16.4} \.{\land} mod . start\_timestamp \.{\geq} policy .
 original\_start\_ts}%
 \@x{\@s{16.4} \.{\land} {\lnot} pendingReinstatementExpired ( policy
 )\@s{16.4}}%
\@y{\@s{0}%
 assumed
}%
\@xx{}%
\@x{\@s{16.4} \.{\land} step \.{'} \.{=} step \.{+} 1}%
 \@x{\@s{16.4} \.{\land} {\UNCHANGED} {\langle} policy ,\, mods ,\, term
 {\rangle}}%
\@pvspace{8.0pt}%
\begin{lcom}{7.5}%
\begin{cpar}{0}{F}{F}{0}{0}{}%
 A cancellation can be issued whenever its policy is \ensuremath{OnRisk}, and
 the cancellation date
 is within the time span of the policy. If any endorsement or renewals are
 pending then
 the cancellation invalidates those endorsements / renewals. In the concrete
 implementation
 the cancellation can also block when there are existing accepted
 modifications. For maximal
 generality, in this development I always assume the cancellation will
 invalidate.
\end{cpar}%
\end{lcom}%
\@x{ issueCancellation ( mod ) \.{\defeq}}%
\@x{\@s{16.4} \.{\LET} pending \.{\defeq} policy . pending\_modification}%
\@x{\@s{16.4} \.{\IN} \.{\land} step \.{<} policyMaxTs}%
\@x{\@s{36.79} \.{\land} policy . state \.{=}\@w{OnRisk}}%
 \@x{\@s{36.79} \.{\land} mod . type \.{=}\@w{Cancellation} \.{\land} mod .
 state \.{=}\@w{Created}}%
 \@x{\@s{36.79} \.{\land} mod . end\_timestamp \.{=} policy .
 effective\_end\_ts\@s{8.2}}%
\@y{\@s{0}%
 assumed
}%
\@xx{}%
 \@x{\@s{36.79} \.{\land} policy . original\_start\_ts \.{\leq} mod .
 start\_timestamp}%
 \@x{\@s{36.79} \.{\land} policy . effective\_end\_ts \.{>} mod .
 start\_timestamp}%
\@x{\@s{36.79}}%
\@y{\@s{0}%
 actions
}%
\@xx{}%
\@x{\@s{36.79} \.{\land} invalidate ( pending )}%
 \@x{\@s{36.79} \.{\land} {\CASE} policy . original\_start\_ts \.{=} mod .
 start\_timestamp \.{\rightarrow}}%
\@x{\@s{56.11} \.{\land}\@s{6.66} policy \.{'} \.{=} [ policy {\EXCEPT}}%
\@x{\@s{156.45} {\bang} . state \.{=}\@w{OffRisk} ,\,}%
\@x{\@s{156.45} {\bang} . pending\_modification \.{=} NoModification ,\,}%
\@x{\@s{156.45} {\bang} . effective\_end\_ts \.{=} mod . start\_timestamp ]}%
\@x{\@s{47.91} {\Box} {\OTHER} \.{\rightarrow}}%
\@x{\@s{56.11} \.{\land} policy \.{'} \.{=} [ policy {\EXCEPT}}%
\@x{\@s{149.78} {\bang} . pending\_modification \.{=} NoModification ,\,}%
\@x{\@s{149.78} {\bang} . effective\_end\_ts \.{=} mod . start\_timestamp ]}%
 \@x{\@s{36.79} \.{\land} mods \.{'} \.{=} Append ( mods ,\, [ mod {\EXCEPT}
 {\bang} . state \.{=}\@w{Issued} ] )}%
\@x{\@s{36.79} \.{\land} step \.{'}\@s{5.04} \.{=} step \.{+} 1}%
\@x{\@s{36.79} \.{\land} {\UNCHANGED} {\langle} term {\rangle}}%
\@pvspace{8.0pt}%
\begin{lcom}{7.5}%
\begin{cpar}{0}{F}{F}{0}{0}{}%
 The legacy lapse action is only active if the \ensuremath{cancellations.json}
 configuration items
 are not present in the policy\mbox{'}s product configuration. We will assume
 this is the
 case for the policy we are considering here, so legacy lapse will never
 execute.
\end{cpar}%
\end{lcom}%
\@x{ legacyLapse ( mod ) \.{\defeq} {\FALSE}}%
\@pvspace{8.0pt}%
\begin{lcom}{7.5}%
\begin{cpar}{0}{F}{F}{0}{0}{}%
As with legacy lapse, above, legacy reinstatements will be disabled if a
 \ensuremath{cancellations.json} file exists in the asset bundle.
\end{cpar}%
\end{lcom}%
\@x{ legacyReinstatement ( mod ) \.{\defeq} {\FALSE}}%
\@pvspace{8.0pt}%
\begin{lcom}{7.5}%
\begin{cpar}{0}{F}{F}{0}{0}{}%
 If a policy\mbox{'}s most recent modification is a cancellation then then we
 can begin a draft
 reinstatement to reinstate it. The draft must specify an effective start
 date within the
 timespan of the cancellation which preceded it, and an end date equal to the
 preceeding
 cancellation.
 As with \ensuremath{createCancellation} we don\mbox{'}t model the addition
 to the modification set, though
 that would be part of the concrete implementation. Also, in the
 implementation, drafts
 are matched 1:1 with a cancellation and that matching is immutable once
 established.
\end{cpar}%
\end{lcom}%
\@x{ draftReinstatement ( mod ) \.{\defeq}}%
\@x{\@s{16.4} \.{\LET} pending \.{\defeq} policy . pending\_modification}%
\@x{\@s{16.4} \.{\IN} \.{\land} step \.{<} policyMaxTs}%
 \@x{\@s{36.79} \.{\land} mod . type \.{=}\@w{Reinstatement} \.{\land} mod .
 deadline\_ts \.{>} step}%
 \@x{\@s{36.79} \.{\land} mod . start\_timestamp \.{\geq} policy .
 effective\_end\_ts}%
\@x{\@s{36.79} \.{\land} mod . start\_timestamp \.{<} mod . end\_timestamp}%
\@x{\@s{36.79} \.{\land} mod . deadline\_ts \.{+} 1\@s{4.63} \.{<} step}%
 \@x{\@s{36.79} \.{\land} prevMod \.{\in} Cancellations \.{\land} prevMod .
 end\_timestamp \.{=} mod . end\_timestamp}%
 \@x{\@s{36.79} \.{\land} {\lnot} pendingReinstatementExpired ( policy
 )\@s{16.4}}%
\@y{\@s{0}%
 assumed
}%
\@xx{}%
\@x{\@s{36.79} \.{\land} step \.{'} \.{=} step \.{+} 1}%
 \@x{\@s{36.79} \.{\land} {\UNCHANGED} {\langle} policy ,\, mods ,\, term
 {\rangle}}%
\@pvspace{8.0pt}%
\begin{lcom}{7.5}%
\begin{cpar}{0}{F}{F}{0}{0}{}%
 If a reinstatement exists past its \ensuremath{deadline\_ts} without being
 issued then it expires.
 On expiration, the reinstatement is removed as a pending modification; its
 documents are
 discarded; and its invoices are settled.
 Note that it is not possible for a draft reinstatement to transition to the
 expired
 state even if \ensuremath{deadline\_dt \.{<}} now. In this case the draft
 reinstatement will be unable to
 transition to the accept state and its \ensuremath{deadline\_dt} will have
 to be updated via the
 update \ensuremath{API} to take part in further transitions.
\end{cpar}%
\end{lcom}%
\@x{ expireReinstatement \.{\defeq}}%
\@x{\@s{16.4} \.{\LET} pending \.{\defeq} policy . pending\_modification}%
\@x{\@s{16.4} \.{\IN} \.{\land} step \.{<} policyMaxTs}%
\@x{\@s{36.79} \.{\land} pendingReinstatementExpired ( policy )}%
\@x{\@s{36.79} \.{\land} invalidate ( pending )}%
 \@x{\@s{36.79} \.{\land} policy \.{'} \.{=} [ policy {\EXCEPT} {\bang} .
 pending\_modification \.{=} NoModification ]}%
\@x{\@s{36.79} \.{\land} step \.{'} \.{=} step \.{+} 1}%
\@x{\@s{36.79} \.{\land} {\UNCHANGED} {\langle} mods ,\, term {\rangle}}%
\@pvspace{8.0pt}%
\begin{lcom}{7.5}%
\begin{cpar}{0}{F}{F}{0}{0}{}%
 In order to accept a reinstatement the start date of the reinstatement must
 be within
 the time span of the cancellation it restores. There must also be (1) no
 other
 accepted modification pending and (2) (in this model) the most recently
 issued
 modification must be a \ensuremath{Cancellation}.
 In the concrete implementation, we dont require that a reinstatement follow
 its
 cancellation. We are slightly more relaxed on only require that all
 cancelations must
 be reinstated before we can quote or accept endorsements or renewals
\end{cpar}%
\end{lcom}%
\@x{ acceptReinstatement ( mod ) \.{\defeq}}%
\@x{\@s{16.4} \.{\LET} pending \.{\defeq} policy . pending\_modification}%
\@x{\@s{16.4} \.{\IN} \.{\land} step \.{<} policyMaxTs}%
\@x{\@s{36.79} \.{\land} pending \.{=} NoModification}%
 \@x{\@s{36.79} \.{\land} mod . type \.{=}\@w{Reinstatement} \.{\land} mod .
 state \.{=}\@w{Created} \.{\land} mod . deadline\_ts \.{>} step}%
 \@x{\@s{36.79} \.{\land} mod . start\_timestamp \.{\geq} policy .
 effective\_end\_ts}%
\@x{\@s{36.79} \.{\land} mod . start\_timestamp \.{<} mod . end\_timestamp}%
 \@x{\@s{36.79} \.{\land} prevMod \.{\in} Cancellations \.{\land} prevMod .
 end\_timestamp \.{=} mod . end\_timestamp}%
\@x{\@s{36.79} \.{\land} policy \.{'}\@s{9.24} \.{=} [ policy {\EXCEPT}}%
 \@x{\@s{127.42} {\bang} . pending\_modification \.{=} [ mod {\EXCEPT} {\bang}
 . state \.{=}\@w{Finalized} ] ]}%
\@x{\@s{36.79} \.{\land} step \.{'} \.{=} step \.{+} 1}%
\@x{\@s{36.79} \.{\land} {\UNCHANGED} {\langle} mods ,\, term {\rangle}}%
\@pvspace{8.0pt}%
\begin{lcom}{7.5}%
\begin{cpar}{0}{F}{F}{0}{0}{}%
 Issuing a reinstatement reactivates the policy restoring the end date and
 coverage as
 of the previous cancellation.
 Note that with cancellations and reinstatements it becomes possible to have
 gaps in a
 policy\mbox{'}s coverage. These gaps are new to the framework and are of
 importance for
 financial calculations and claim coverage. The gaps are recorded in the
 database in
 the policy\mbox{'}s currently active \ensuremath{policy\_characteristics}
 set. These \ensuremath{policy\_characteristics
 } previously would have been sequential, meeting, and non overlapping. Now
 they must be,
 just, sequential and not overlapping.
\end{cpar}%
\end{lcom}%
\@x{ issueReinstatement \.{\defeq}}%
\@x{\@s{16.4} \.{\LET} pending \.{\defeq} policy . pending\_modification}%
\@x{\@s{16.4} \.{\IN} \.{\land} step \.{<} policyMaxTs}%
\@x{\@s{36.79} \.{\land} pending \.{\in} Reinstatements}%
 \@x{\@s{36.79} \.{\land} {\lnot} pendingReinstatementExpired ( policy
 )\@s{16.4}}%
\@y{\@s{0}%
 assumed
}%
\@xx{}%
 \@x{\@s{36.79} \.{\land} policy \.{'} \.{=} [ policy {\EXCEPT} {\bang} .
 state \.{=}\@w{OnRisk} ,\,}%
\@x{\@s{160.19} {\bang} . pending\_modification \.{=} NoModification ]}%
 \@x{\@s{36.79} \.{\land} mods \.{'} \.{=} Append ( mods ,\, [ pending
 {\EXCEPT} {\bang} . state \.{=}\@w{Issued} ] )}%
\@x{\@s{36.79} \.{\land} step \.{'}\@s{5.04} \.{=} step \.{+} 1}%
\@x{\@s{36.79} \.{\land} {\UNCHANGED} {\langle} term {\rangle}}%
\@pvspace{8.0pt}%
\begin{lcom}{7.5}%
\begin{cpar}{0}{F}{F}{0}{0}{}%
 When we invalidate a reinstatement we return the modification to the draft
 state. We
 discard all documents associated with the reinstatement and then settle any
 invoices.
\end{cpar}%
\end{lcom}%
\@x{ invalidateReinstatement \.{\defeq}}%
\@x{\@s{16.4} \.{\LET} pending \.{\defeq} policy . pending\_modification}%
\@x{\@s{16.4} \.{\IN} \.{\land} step \.{<} policyMaxTs}%
 \@x{\@s{36.79} \.{\land} pending \.{\in} Reinstatements \.{\land} step
 \.{\leq} pending . deadline\_ts}%
\@x{\@s{36.79} \.{\land} invalidate ( pending )}%
 \@x{\@s{36.79} \.{\land} policy \.{'} \.{=} [ policy {\EXCEPT} {\bang} .
 pending\_modification \.{=} NoModification ]}%
\@x{\@s{36.79} \.{\land} step \.{'} \.{=} step \.{+} 1}%
\@x{\@s{36.79} \.{\land} {\UNCHANGED} {\langle} mods ,\, term {\rangle}}%
\@pvspace{8.0pt}%
\begin{lcom}{7.5}%
\begin{cpar}{0}{F}{F}{0}{0}{}%
If we receive a payment when we have a finalized reinstatement in the
 pending modifications set then the policy can be issued if
 \ensuremath{autoReinstate
} is on.
 The operation below is just a placeholder to mark the existence of the auto
 reinstate
 functionality. It is not included in the model as inclusion would just
 repeat the
 \ensuremath{issueReinstatement} operation above.
\end{cpar}%
\end{lcom}%
\@x{ payInvoiceReinstatement \.{\defeq}}%
\@x{\@s{16.4} \.{\LET} mod \.{\defeq} policy . pending\_modification}%
\@x{\@s{16.4} \.{\IN} \.{\land} step \.{<} policyMaxTs}%
\@x{\@s{36.79} \.{\land} mod . auto\_reinstate \.{=} {\TRUE}}%
\@x{\@s{36.79} \.{\land} issueReinstatement}%
\@pvspace{8.0pt}%
\@x{ doStep \.{\defeq}}%
\@x{\@s{16.4} \.{\LET} mod \.{\defeq} policy . pending\_modification}%
\@x{\@s{16.4} \.{\IN} \.{\land} step \.{<} policyMaxTs}%
 \@x{\@s{36.79} \.{\land} {\IF} ( \.{\land} mod \.{\notin} \{ NoModification
 ,\, EndorsementOrRenewal \}}%
\@x{\@s{64.16} \.{\land} mod . type \.{=}\@w{Reinstatement}}%
\@x{\@s{64.16} \.{\land} step \.{>} mod . deadline\_ts )}%
\@x{\@s{47.91} \.{\THEN} expireReinstatement}%
\@x{\@s{47.91} \.{\ELSE} \.{\land} step \.{'} \.{=} step \.{+} 1}%
 \@x{\@s{79.22} \.{\land} {\UNCHANGED} {\langle} policy ,\, term ,\, mods
 {\rangle}}%
\@pvspace{8.0pt}%
\@x{ doTerm \.{\defeq}}%
\@x{\@s{16.4} \.{\land}\@s{6.86} step \.{=} policyMaxTs}%
\@x{\@s{16.4} \.{\land}\@s{6.86} term \.{'} \.{=} {\TRUE}}%
 \@x{\@s{16.4} \.{\land}\@s{6.86} {\UNCHANGED} {\langle} policy ,\, step ,\,
 mods {\rangle}}%
\@pvspace{8.0pt}%
\begin{lcom}{7.5}%
\begin{cpar}{0}{F}{F}{0}{0}{}%
 Testing notes: With configurations that calculate all charges as a percentage
 of
 premium (not all), there are many sequences of cancellation and reinstatment
 operations
 that should leave the total charges invariant. For example, it should be the
 case that
\end{cpar}%
\vshade{5.0}%
\begin{cpar}{0}{F}{F}{0}{0}{}%
Issue; Cancel; Reinstate; \ensuremath{\.{=}} Issue
\end{cpar}%
\vshade{5.0}%
\begin{cpar}{0}{F}{F}{0}{0}{}%
 These sorts of relationship are useful for checking financial calculation
 routines.
\end{cpar}%
\end{lcom}%
\@x{}\midbar\@xx{}%
\@x{ Init \.{\defeq}}%
\@x{\@s{16.4} \.{\land} step \.{=} 0}%
\@x{\@s{16.4} \.{\land} policy \.{=} [}%
\@x{\@s{31.61} original\_start\_ts \.{\mapsto} policyTi ,\,}%
\@x{\@s{31.61} effective\_end\_ts \.{\mapsto} policyTf ,\,}%
\@x{\@s{31.61} billing\_policy \.{\mapsto}\@w{BySystemOnFinalize} ,\,}%
\@x{\@s{31.61} state \.{\mapsto}\@w{OnRisk} ,\,}%
 \@x{\@s{31.61} revision\_start\_timestamps \.{\mapsto} \{ {\langle} 0 ,\,
 policyTi {\rangle} \} ,\,}%
\@x{\@s{31.61} renewal\_start\_timestamps \.{\mapsto} \{ \} ,\,}%
\@x{\@s{31.61} pending\_modification \.{\mapsto} NoModification ]}%
\@x{\@s{16.4} \.{\land} mods \.{=} {\langle} {\rangle}}%
\@x{\@s{16.4} \.{\land} term\@s{1.71} \.{=} {\FALSE}}%
\@pvspace{8.0pt}%
\@x{ Next \.{\defeq}}%
\@x{\@s{16.4} \.{\lor} acceptGeneric}%
 \@x{\@s{16.4} \.{\lor} \exists\, m \.{\in} Cancellations \.{:}
 createCancellation ( m )}%
 \@x{\@s{16.4} \.{\lor} \exists\, m \.{\in} Cancellations \.{:}
 issueCancellation ( m )}%
 \@x{\@s{16.4} \.{\lor} \exists\, m \.{\in} Cancellations \.{:} legacyLapse (
 m )}%
 \@x{\@s{16.4} \.{\lor} \exists\, m \.{\in} Reinstatements \.{:}
 draftReinstatement ( m )}%
 \@x{\@s{16.4} \.{\lor} \exists\, m \.{\in} Reinstatements \.{:}
 acceptReinstatement ( m )}%
\@x{\@s{16.4} \.{\lor} expireReinstatement}%
\@x{\@s{16.4} \.{\lor} issueReinstatement}%
\@x{\@s{16.4} \.{\lor} invalidateReinstatement}%
\@x{\@s{16.4} \.{\lor} doStep}%
\@x{\@s{16.4} \.{\lor} doTerm}%
 \@x{\@s{16.4} \.{\lor} ( term \.{=} {\TRUE} \.{\land} {\UNCHANGED} {\langle}
 policy ,\, step ,\, mods ,\, term {\rangle} )}%
\@pvspace{8.0pt}%
 \@x{ Spec \.{\defeq} Init \.{\land} {\Box} [ Next ]_{ {\langle} policy ,\,
 step ,\, mods ,\, term {\rangle}}}%
\@pvspace{8.0pt}%
\@x{}\bottombar\@xx{}%
\setboolean{shading}{false}
\begin{lcom}{0}%
\begin{cpar}{0}{F}{F}{0}{0}{}%
\ensuremath{\.{\,\backslash\,}}* Modification History
\end{cpar}%
\begin{cpar}{0}{F}{F}{0}{0}{}%
 \ensuremath{\.{\,\backslash\,}}* Last modified \ensuremath{Tue}
 \ensuremath{Dec} 15 12:15:46 \ensuremath{PST} 2020 by \ensuremath{ASUS
}%
\end{cpar}%
\begin{cpar}{0}{F}{F}{0}{0}{}%
 \ensuremath{\.{\,\backslash\,}}* Last modified \ensuremath{Thu}
 \ensuremath{Jul} 16 16:36:58 \ensuremath{PDT} 2020 by \ensuremath{marco
}%
\end{cpar}%
\begin{cpar}{0}{F}{F}{0}{0}{}%
 \ensuremath{\.{\,\backslash\,}}* \ensuremath{Created} Sat \ensuremath{Jun} 27
 21:12:50 \ensuremath{PDT} 2020 by \ensuremath{ASUS
}%
\end{cpar}%
\end{lcom}%
