\tlatex
\setboolean{shading}{true}
\@x{}\moduleLeftDash\@xx{ {\MODULE} CancellationMch1}\moduleRightDash\@xx{}%
\@x{ {\EXTENDS} CancellationCtx1 ,\, Policy ,\, Sequences}%
\@pvspace{8.0pt}%
 \@x{ {\CONSTANTS} policyTi ,\, policyTf ,\, reinstatementExpireDays ,\,
 Documents}%
\@pvspace{8.0pt}%
\@x{ {\VARIABLES} policy ,\, step ,\, mods ,\, term}%
\begin{lcom}{7.5}%
\begin{cpar}{0}{F}{F}{0}{0}{}%
 Here, I model the state by separating it into a \ensuremath{Policy} state and
 a Modification Set
 state. Observing just the state of the the Modification Set, each
 Modification follows
 a stand alone, independent state machine which in greatest generality is:
 . \ensuremath{\.{\rightarrow} Created \.{\rightarrow}} Accepted
 \ensuremath{\.{\rightarrow} Issued \.{\,|\,}} Invalidated
 \ensuremath{\.{\rightarrow}} . Observing just policy state,
 a \ensuremath{Policy} only listens to \ensuremath{Issued} events, and
 updates its state on those events.
 In the system all modifications are retained but for this abstraction only
 issued
 modifications are recorded. This omission reduces the state space without
 loss of
 generality.
\end{cpar}%
\end{lcom}%
\@x{}\midbar\@xx{}%
\@x{}%
\@y{\@s{0}%
 Functions
}%
\@xx{}%
\@x{ prevMod \.{\defeq} \.{\LET} last \.{\defeq} Len ( mods )}%
 \@x{\@s{56.61} \.{\IN} {\IF} last \.{=} 0 \.{\THEN} NoModification \.{\ELSE}
 mods [ last ]}%
\@pvspace{8.0pt}%
\@x{ pendingReinstatementExpired ( p ) \.{\defeq}}%
\@x{\@s{16.4} \.{\LET} pending \.{\defeq} p . pending\_modification}%
 \@x{\@s{16.4} \.{\IN} pending \.{\in} Reinstatements \.{\land} pending .
 deadline\_ts \.{<} step}%
\@pvspace{8.0pt}%
\begin{lcom}{0}%
\begin{cpar}{0}{T}{F}{7.5}{0}{}%
 Just a semantic addition for the reader. When this function appears in a
 guard, the
\end{cpar}%
\begin{cpar}{1}{F}{F}{0}{0}{}%
 reader should interpret it a a requirement to check and invalidate any
 accepted
 modifications
\end{cpar}%
\end{lcom}%
\@x{ invalidate ( mod ) \.{\defeq} {\TRUE}}%
\@pvspace{8.0pt}%
\@x{}\midbar\@xx{}%
\@x{}%
\@y{\@s{0}%
 Invariants
}%
\@xx{}%
\@x{}%
\@y{\@s{0}%
 Below we define a few common invariants of the system.
}%
\@xx{}%
\@x{ InvType \.{\defeq}}%
\@x{\@s{16.4} \.{\LET} pending \.{\defeq} policy . pending\_modification}%
 \@x{\@s{16.4} \.{\IN} \.{\land} pending \.{\in} {\UNION} \{ Reinstatements
 ,\,}%
\@x{\@s{131.90} Cancellations ,\,}%
\@x{\@s{131.90} \{ NoModification ,\, EndorsementOrRenewal \}}%
\@x{\@s{131.90} \}}%
\@x{\@s{36.79} \.{\land} policy . state \.{\in} PolicyState}%
\@pvspace{8.0pt}%
\begin{lcom}{7.5}%
\begin{cpar}{0}{F}{F}{0}{0}{}%
 Every policy can have zero or one pending modifications and that pending
 modification
 must be in the \ensuremath{Finalized} (Accepted) state.
\end{cpar}%
\end{lcom}%
\@x{ InvPolicy \.{\defeq}}%
\@x{\@s{16.4} \.{\LET} pending \.{\defeq} policy . pending\_modification}%
\@x{\@s{16.4} \.{\IN} \.{\land} policy . original\_start\_ts \.{=} policyTi}%
 \@x{\@s{36.79} \.{\land} pending \.{\notin} \{ NoModification ,\,
 EndorsementOrRenewal \} \.{\implies}}%
\@x{\@s{181.66} pending . state \.{=}\@w{Finalized}}%
\@x{\@s{36.79} \.{\land} policy . state \.{=}\@w{OffRisk} \.{\implies} (}%
\@x{\@s{56.11} pending \.{\neq} EndorsementOrRenewal}%
 \@x{\@s{56.11} \.{\land} policy . original\_start\_ts \.{=} policy .
 effective\_end\_ts )}%
\@x{\@s{36.79} \.{\land} (\@s{4.31} \.{\land} pending \.{\in} Reinstatements}%
 \@x{\@s{52.01} \.{\land} pending . auto\_reinstate ) \.{\implies} pending .
 deadline\_ts \.{+} 1 \.{\geq} step}%
\@pvspace{8.0pt}%
\begin{lcom}{7.5}%
\begin{cpar}{0}{F}{F}{0}{0}{}%
 When a policy is fully cancelled it is \ensuremath{OffRisk} and the time
 interval of the policy
 has zero duration. From that state it duration can only be increased by a
 reinstatement.
\end{cpar}%
\end{lcom}%
\@x{ InvMods \.{\defeq}}%
 \@x{\@s{16.4} \.{\LET}\@s{0.86} pending \.{\defeq} policy .
 pending\_modification}%
\@x{\@s{16.4} \.{\IN} \.{\land} prevMod \.{\neq} NoModification \.{\implies}}%
 \@x{\@s{56.11} prevMod . start\_timestamp \.{\geq} policy .
 original\_start\_ts}%
 \@x{\@s{36.79} \.{\land} policy . state \.{=}\@w{OffRisk} \.{\implies}
 prevMod \.{\in} Cancellations}%
 \@x{\@s{36.79} \.{\land} policy . state \.{=}\@w{OffRisk} \.{\implies}
 pending\@s{3.62} \.{\in} Reinstatements \.{\cup} \{ NoModification \}}%
 \@x{\@s{36.79} \.{\land} \A\, idx \.{\in} 1 \.{\dotdot} Len ( mods ) \.{:}
 mods [ idx ] \.{\in} Cancellations \.{\implies}}%
 \@x{\@s{67.43} ( mods [ idx ] . start\_timestamp \.{<} mods [ idx ] .
 end\_timestamp )}%
 \@x{\@s{36.79} \.{\land} \A\, idx \.{\in} 2 \.{\dotdot} Len ( mods ) \.{:}
 mods [ idx ] \.{\in} Reinstatements \.{\implies}}%
 \@x{\@s{67.43} ( \.{\land} mods [ idx\@s{0.21} \.{-} 1 ] \.{\in}
 Cancellations}%
 \@x{\@s{71.53} \.{\land} mods [ idx \.{-} 1 ] . end\_timestamp \.{=} mods [
 idx ] . end\_timestamp )}%
\@pvspace{8.0pt}%
\@x{}\midbar\@xx{}%
\begin{lcom}{7.5}%
\begin{cpar}{0}{F}{F}{0}{0}{}%
 As long as a policy is \ensuremath{OnRisk}, not fully canceled, endorsements
 or renewals for the
 policy may transition into an accepted state, creating a pending modification
\end{cpar}%
\end{lcom}%
\@x{ acceptGeneric \.{\defeq}}%
\@x{\@s{16.4} \.{\land} step \.{<} policyMaxTs}%
 \@x{\@s{16.4} \.{\land} policy . state \.{=}\@w{OnRisk} \.{\land} policy .
 pending\_modification \.{=} NoModification}%
 \@x{\@s{16.4} \.{\land} policy \.{'} \.{=} [ policy {\EXCEPT} {\bang} .
 pending\_modification \.{=} EndorsementOrRenewal ]}%
\@x{\@s{16.4} \.{\land} step \.{'} \.{=} step \.{+} 1}%
\@x{\@s{16.4} \.{\land} {\UNCHANGED} {\langle} mods ,\, term {\rangle}}%
\@pvspace{8.0pt}%
\begin{lcom}{7.5}%
\begin{cpar}{0}{F}{F}{0}{0}{}%
 A cancellation can be created for any policy that is \ensuremath{OnRisk}. The
 start date specified by
 the cancellation should be within the coverage time span of the policy.
 Note that for the specification, we do not have cancellation create events
 result in
 additions to the modifications set. This addition would ordinarily happen in
 a concrete
 implementation, however, we elide the effect here as maintaining the state
 in the policy
 record along with a list of issued modifications is sufficient to validate
 system
 properties.
\end{cpar}%
\end{lcom}%
\@x{ createCancellation ( mod ) \.{\defeq}}%
\@x{\@s{16.4} \.{\land} step \.{<} policyMaxTs}%
\@x{\@s{16.4} \.{\land} policy . state \.{=}\@w{OnRisk}}%
 \@x{\@s{16.4} \.{\land} mod . type \.{=}\@w{Cancellation} \.{\land} mod .
 state \.{=}\@w{Created}}%
 \@x{\@s{16.4} \.{\land} mod . start\_timestamp \.{<} policy .
 effective\_end\_ts}%
 \@x{\@s{16.4} \.{\land} mod . start\_timestamp \.{\geq} policy .
 original\_start\_ts}%
 \@x{\@s{16.4} \.{\land} {\lnot} pendingReinstatementExpired ( policy
 )\@s{16.4}}%
\@y{\@s{0}%
 assumed
}%
\@xx{}%
\@x{\@s{16.4} \.{\land} step \.{'} \.{=} step \.{+} 1}%
 \@x{\@s{16.4} \.{\land} {\UNCHANGED} {\langle} policy ,\, mods ,\, term
 {\rangle}}%
\@pvspace{8.0pt}%
\begin{lcom}{7.5}%
\begin{cpar}{0}{F}{F}{0}{0}{}%
 A cancellation can be issued whenever its policy is \ensuremath{OnRisk}.
 Further, a cancellation can be
 issued regardless of any current, pending modifications on its policy. The
 only
 restriction is that the cancellation date must be within the time span of
 the policy. If
 any endorsement or renewals are pending then those must be invalidated.
\end{cpar}%
\end{lcom}%
\@x{ issueCancellation ( mod ) \.{\defeq}}%
\@x{\@s{16.4} \.{\LET} pending \.{\defeq} policy . pending\_modification}%
\@x{\@s{16.4} \.{\IN} \.{\land} step \.{<} policyMaxTs}%
\@x{\@s{36.79} \.{\land} policy . state \.{=}\@w{OnRisk}}%
 \@x{\@s{36.79} \.{\land} mod . type \.{=}\@w{Cancellation} \.{\land} mod .
 state \.{=}\@w{Created}}%
 \@x{\@s{36.79} \.{\land} mod . end\_timestamp \.{=} policy .
 effective\_end\_ts\@s{8.2}}%
\@y{\@s{0}%
 assumed
}%
\@xx{}%
 \@x{\@s{36.79} \.{\land} policy . original\_start\_ts \.{\leq} mod .
 start\_timestamp}%
 \@x{\@s{36.79} \.{\land} policy . effective\_end\_ts \.{>} mod .
 start\_timestamp}%
\@x{\@s{36.79}}%
\@y{\@s{0}%
 actions
}%
\@xx{}%
\@x{\@s{36.79} \.{\land} invalidate ( pending )}%
 \@x{\@s{36.79} \.{\land} {\CASE} policy . original\_start\_ts \.{=} mod .
 start\_timestamp \.{\rightarrow}}%
\@x{\@s{56.11} \.{\land}\@s{6.66} policy \.{'} \.{=} [ policy {\EXCEPT}}%
\@x{\@s{156.45} {\bang} . state \.{=}\@w{OffRisk} ,\,}%
\@x{\@s{156.45} {\bang} . pending\_modification \.{=} NoModification ,\,}%
\@x{\@s{156.45} {\bang} . effective\_end\_ts \.{=} mod . start\_timestamp ]}%
\@x{\@s{47.91} {\Box} {\OTHER} \.{\rightarrow}}%
\@x{\@s{56.11} \.{\land} policy \.{'} \.{=} [ policy {\EXCEPT}}%
\@x{\@s{149.78} {\bang} . pending\_modification \.{=} NoModification ,\,}%
\@x{\@s{149.78} {\bang} . effective\_end\_ts \.{=} mod . start\_timestamp ]}%
 \@x{\@s{36.79} \.{\land} mods \.{'} \.{=} Append ( mods ,\, [ mod {\EXCEPT}
 {\bang} . state \.{=}\@w{Issued} ] )}%
\@x{\@s{36.79} \.{\land} step \.{'}\@s{5.04} \.{=} step \.{+} 1}%
\@x{\@s{36.79} \.{\land} {\UNCHANGED} {\langle} term {\rangle}}%
\@pvspace{8.0pt}%
\begin{lcom}{7.5}%
\begin{cpar}{0}{F}{F}{0}{0}{}%
 The legacy lapse action acts similarly to the above cancellation method. It
 can be
 thought of as the sequential application of a cancellation create + issue.
 There is a
 switch in the new cancellation configuration to disable the
 \ensuremath{legacyLapse} action, possibly
 but not preferably, as shown in the first guard condition below
\end{cpar}%
\end{lcom}%
\@x{ legacyLapse ( mod ) \.{\defeq}}%
\@x{\@s{16.4} \.{\land}\@w{Lapse} \.{\notin} CancellationReasons}%
\@x{\@s{16.4} \.{\land} issueCancellation ( mod )}%
\@pvspace{8.0pt}%
\begin{lcom}{7.5}%
\begin{cpar}{0}{F}{F}{0}{0}{}%
 I have not included \ensuremath{legacyReinstatement} in the model, as one can
 see in the \ensuremath{Next} definition.
 Instead I just note that \ensuremath{legacyReinstatment} should equal the
 effect of consecutive draft,
 accept, issue operations in the new \ensuremath{API}, and so it is
 effectively already modeled.
 Because this function is defined as draft + accept + issue, the
 preconditions for this
 functions proper operation must be the conjunction of the guard conditions
 for draft,
 accept, and issue below.
\end{cpar}%
\end{lcom}%
\@x{ legacyReinstatement ( mod ) \.{\defeq}}%
\@x{\@s{16.4} \.{\land}\@w{Lapse} \.{\notin} CancellationReasons}%
\@x{\@s{16.4} \.{\land} step \.{<} policyMaxTs}%
\@x{\@s{16.4} \.{\land} policy . state \.{=}\@w{OffRisk}}%
 \@x{\@s{16.4} \.{\land} mod . type \.{=}\@w{Reinstatement} \.{\land} mod .
 state \.{=}\@w{Created}}%
 \@x{\@s{16.4} \.{\land} mod . start\_timestamp \.{<} policy .
 effective\_end\_ts}%
 \@x{\@s{16.4} \.{\land} mod . start\_timestamp \.{\geq} prevMod .
 start\_timestamp}%
\@x{\@s{16.4} \.{\land} mod . deadline\_ts \.{\geq} step}%
 \@x{\@s{16.4} \.{\land} policy \.{'} \.{=} [ policy {\EXCEPT} {\bang} . state
 \.{=}\@w{OnRisk} ,\,}%
\@x{\@s{139.79} {\bang} . pending\_modification \.{=} NoModification ]}%
 \@x{\@s{16.4} \.{\land} mods \.{'} \.{=} Append ( mods ,\, [ mod {\EXCEPT}
 {\bang} . state \.{=}\@w{Issued} ] )}%
\@x{\@s{16.4} \.{\land} step \.{'}\@s{5.04} \.{=} step \.{+} 1}%
\@x{\@s{16.4} \.{\land} {\UNCHANGED} {\langle} term {\rangle}}%
\@pvspace{8.0pt}%
\begin{lcom}{7.5}%
\begin{cpar}{0}{F}{F}{0}{0}{}%
 If a policy is in the canceled state then we can begin a draft reinstatement
 to reinstate
 it. The draft must specify an effective start date within the timespan
 truncated by the
 cancellation which preceded it. For the purposes of specification, updating
 a draft
 reinstatement is modeled as having the same processing effect as adding a
 new draft to
 the modification set.
 As with \ensuremath{createCancellation} we don\mbox{'}t model the addition
 to the modification set, though
 that would be part of the concrete implementation.
\end{cpar}%
\end{lcom}%
\@x{ draftReinstatement ( mod ) \.{\defeq}}%
\@x{\@s{16.4} \.{\LET} pending \.{\defeq} policy . pending\_modification}%
\@x{\@s{16.4} \.{\IN} \.{\land} step \.{<} policyMaxTs}%
 \@x{\@s{36.79} \.{\land} mod . type \.{=}\@w{Reinstatement} \.{\land} mod .
 deadline\_ts \.{>} step}%
 \@x{\@s{36.79} \.{\land} mod . start\_timestamp \.{\geq} policy .
 effective\_end\_ts}%
\@x{\@s{36.79} \.{\land} mod . deadline\_ts \.{+} 1\@s{4.63} \.{<} step}%
\@x{\@s{36.79} \.{\land} prevMod \.{\in} Cancellations}%
 \@x{\@s{36.79} \.{\land} {\lnot} pendingReinstatementExpired ( policy
 )\@s{16.4}}%
\@y{\@s{0}%
 assumed
}%
\@xx{}%
\@x{\@s{36.79} \.{\land} step \.{'} \.{=} step \.{+} 1}%
 \@x{\@s{36.79} \.{\land} {\UNCHANGED} {\langle} policy ,\, mods ,\, term
 {\rangle}}%
\@pvspace{8.0pt}%
\begin{lcom}{7.5}%
\begin{cpar}{0}{F}{F}{0}{0}{}%
 If a reinstatement exists past its \ensuremath{deadline\_ts} without being
 issued then it expires.
 At that point the reinstatement is removed as a pending modification, its
 documents are
 discarded, and its invoices are settled.
 Though not modeled here it is possible for a reinstatement in the draft
 state to expire.
 These can just be discarded.
\end{cpar}%
\end{lcom}%
\@x{ expireReinstatement \.{\defeq}}%
\@x{\@s{16.4} \.{\LET} pending \.{\defeq} policy . pending\_modification}%
\@x{\@s{16.4} \.{\IN} \.{\land} step \.{<} policyMaxTs}%
\@x{\@s{36.79} \.{\land} pendingReinstatementExpired ( policy )}%
\@x{\@s{36.79} \.{\land} invalidate ( pending )}%
 \@x{\@s{36.79} \.{\land} policy \.{'} \.{=} [ policy {\EXCEPT} {\bang} .
 pending\_modification \.{=} NoModification ]}%
\@x{\@s{36.79} \.{\land} step \.{'} \.{=} step \.{+} 1}%
\@x{\@s{36.79} \.{\land} {\UNCHANGED} {\langle} mods ,\, term {\rangle}}%
\@pvspace{8.0pt}%
\begin{lcom}{7.5}%
\begin{cpar}{0}{F}{F}{0}{0}{}%
 In order to accept a reinstatement the start date of the reinstatement must
 be within
 the time span truncated by the cancellation it follows. There must also be
 (1) no other
 accepted modifications pending and (2) (in this model) the most recently
 issued
 modification must be a \ensuremath{Cancellation}.
\end{cpar}%
\vshade{5.0}%
\begin{cpar}{0}{F}{F}{0}{0}{}%
 (1) means the reinstatement waits, like other modification operations, when
 it is blocked.
 Formally it participates in the system Draft \ensuremath{\.{\rightarrow}}
 Accept \ensuremath{\.{\rightarrow}} Issue workflow. Note that
 this behavior is unlike the \ensuremath{Cancellation} workflow, which does
 not block and is
 privileged to clear the global accept lock when it does its work.
\end{cpar}%
\vshade{5.0}%
\begin{cpar}{0}{F}{F}{0}{0}{}%
(2) is a bit more restrictive than what we want for the concrete
 implementation where we will only insist that a cancellation designated in
 the request
 exists and has not since been followed by a renewal
\end{cpar}%
\end{lcom}%
\@x{ acceptReinstatement ( mod ) \.{\defeq}}%
\@x{\@s{16.4} \.{\LET} pending \.{\defeq} policy . pending\_modification}%
\@x{\@s{16.4} \.{\IN} \.{\land} step \.{<} policyMaxTs}%
\@x{\@s{36.79} \.{\land} pending \.{=} NoModification}%
 \@x{\@s{36.79} \.{\land} mod . type \.{=}\@w{Reinstatement} \.{\land} mod .
 state \.{=}\@w{Created} \.{\land} mod . deadline\_ts \.{>} step}%
 \@x{\@s{36.79} \.{\land} mod . start\_timestamp \.{\geq} policy .
 effective\_end\_ts}%
\@x{\@s{36.79} \.{\land} mod . start\_timestamp \.{<} mod . end\_timestamp}%
 \@x{\@s{36.79} \.{\land} prevMod \.{\in} Cancellations \.{\land} prevMod .
 end\_timestamp \.{=} mod . end\_timestamp}%
\@x{\@s{36.79} \.{\land} policy \.{'}\@s{9.24} \.{=} [ policy {\EXCEPT}}%
 \@x{\@s{127.42} {\bang} . pending\_modification \.{=} [ mod {\EXCEPT} {\bang}
 . state \.{=}\@w{Finalized} ] ]}%
\@x{\@s{36.79} \.{\land} step \.{'} \.{=} step \.{+} 1}%
\@x{\@s{36.79} \.{\land} {\UNCHANGED} {\langle} mods ,\, term {\rangle}}%
\@pvspace{8.0pt}%
\begin{lcom}{7.5}%
\begin{cpar}{0}{F}{F}{0}{0}{}%
 Issuing a reinstatement reactivates the policy restoring the end date and
 coverage as
 of the previous cancellation.
 Note that with cancellations and reinstatements it becomes possible to have
 gaps in a
 policies coverage. These gaps will be of some potential concern for financial
 calculations and claim coverage. Gaps will also factor into the timestamps
 on peril
 characteristic records in ways that will need to be defined and documented.
\end{cpar}%
\end{lcom}%
\@x{ issueReinstatement \.{\defeq}}%
\@x{\@s{16.4} \.{\LET} pending \.{\defeq} policy . pending\_modification}%
\@x{\@s{16.4} \.{\IN} \.{\land} step \.{<} policyMaxTs}%
\@x{\@s{36.79} \.{\land} pending \.{\in} Reinstatements}%
 \@x{\@s{36.79} \.{\land} {\lnot} pendingReinstatementExpired ( policy
 )\@s{16.4}}%
\@y{\@s{0}%
 assumed
}%
\@xx{}%
 \@x{\@s{36.79} \.{\land} policy \.{'} \.{=} [ policy {\EXCEPT} {\bang} .
 state \.{=}\@w{OnRisk} ,\,}%
\@x{\@s{160.19} {\bang} . pending\_modification \.{=} NoModification ]}%
 \@x{\@s{36.79} \.{\land} mods \.{'} \.{=} Append ( mods ,\, [ pending
 {\EXCEPT} {\bang} . state \.{=}\@w{Issued} ] )}%
\@x{\@s{36.79} \.{\land} step \.{'}\@s{5.04} \.{=} step \.{+} 1}%
\@x{\@s{36.79} \.{\land} {\UNCHANGED} {\langle} term {\rangle}}%
\@pvspace{8.0pt}%
\begin{lcom}{7.5}%
\begin{cpar}{0}{F}{F}{0}{0}{}%
 When we invalidate a reinstatement we return the modification to the draft
 state. We
 discard all documents associated with the reinstatement and then settle any
 invoices.
\end{cpar}%
\end{lcom}%
\@x{ invalidateReinstatement \.{\defeq}}%
\@x{\@s{16.4} \.{\LET} pending \.{\defeq} policy . pending\_modification}%
\@x{\@s{16.4} \.{\IN} \.{\land} step \.{<} policyMaxTs}%
 \@x{\@s{36.79} \.{\land} pending \.{\in} Reinstatements \.{\land} step
 \.{\leq} pending . deadline\_ts}%
\@x{\@s{36.79} \.{\land} invalidate ( pending )}%
 \@x{\@s{36.79} \.{\land} policy \.{'} \.{=} [ policy {\EXCEPT} {\bang} .
 pending\_modification \.{=} NoModification ]}%
\@x{\@s{36.79} \.{\land} step \.{'} \.{=} step \.{+} 1}%
\@x{\@s{36.79} \.{\land} {\UNCHANGED} {\langle} mods ,\, term {\rangle}}%
\@pvspace{8.0pt}%
\begin{lcom}{7.5}%
\begin{cpar}{0}{F}{F}{0}{0}{}%
If we receive a payment when we have a finalized reinstatement in the
 pending modifications set then the policy can be issued if
 \ensuremath{autoReinstate
} is on.
 The operation below is just a placeholder to make the existence of the auto
 reinstate
 functionality. Implementing it would just repeat the
 \ensuremath{issueReinstatement} operation
 above.
\end{cpar}%
\end{lcom}%
\@x{ payInvoiceReinstatement \.{\defeq}}%
\@x{\@s{16.4} \.{\LET} mod \.{\defeq} policy . pending\_modification}%
\@x{\@s{16.4} \.{\IN} \.{\land} step \.{<} policyMaxTs}%
\@x{\@s{36.79} \.{\land} mod . auto\_reinstate \.{=} {\TRUE}}%
\@x{\@s{36.79} \.{\land} issueReinstatement}%
\@pvspace{8.0pt}%
\@x{ doStep \.{\defeq}}%
\@x{\@s{16.4} \.{\LET} mod \.{\defeq} policy . pending\_modification}%
\@x{\@s{16.4} \.{\IN} \.{\land} step \.{<} policyMaxTs}%
 \@x{\@s{36.79} \.{\land} {\IF} ( \.{\land} mod \.{\notin} \{ NoModification
 ,\, EndorsementOrRenewal \}}%
\@x{\@s{64.16} \.{\land} mod . type \.{=}\@w{Reinstatement}}%
\@x{\@s{64.16} \.{\land} step \.{>} mod . deadline\_ts )}%
\@x{\@s{47.91} \.{\THEN} expireReinstatement}%
\@x{\@s{47.91} \.{\ELSE} \.{\land} step \.{'} \.{=} step \.{+} 1}%
 \@x{\@s{79.22} \.{\land} {\UNCHANGED} {\langle} policy ,\, term ,\, mods
 {\rangle}}%
\@pvspace{8.0pt}%
\@x{ doTerm \.{\defeq}}%
\@x{\@s{16.4} \.{\land}\@s{6.86} step \.{=} policyMaxTs}%
\@x{\@s{16.4} \.{\land}\@s{6.86} term \.{'} \.{=} {\TRUE}}%
 \@x{\@s{16.4} \.{\land}\@s{6.86} {\UNCHANGED} {\langle} policy ,\, step ,\,
 mods {\rangle}}%
\@pvspace{8.0pt}%
\begin{lcom}{7.5}%
\begin{cpar}{0}{F}{F}{0}{0}{}%
 General observations: For cancellations I believe that if we issue a policy
 from \ensuremath{t1} to
 \ensuremath{t3} and then cancel back to some intermediate time
 \ensuremath{t2}, the effect, from the pov of an
 external observer of total payments, should be as if the policy was issued
 from \ensuremath{t1} to \ensuremath{t2}.
 Or, more generally, there are many sequences where the following will hold
\end{cpar}%
\vshade{5.0}%
\begin{cpar}{0}{F}{F}{0}{0}{}%
Issue; Cancel; Reinstate; \ensuremath{\.{=}} Issue
\end{cpar}%
\vshade{5.0}%
\begin{cpar}{0}{F}{F}{0}{0}{}%
 These sorts of relationship should be useful in checking the financial
 calculation
 routines to the extent they need to be checked.
\end{cpar}%
\end{lcom}%
\@x{}\midbar\@xx{}%
\@x{ Init \.{\defeq}}%
\@x{\@s{16.4} \.{\land} step \.{=} 0}%
\@x{\@s{16.4} \.{\land} policy \.{=} [}%
\@x{\@s{31.61} original\_start\_ts \.{\mapsto} policyTi ,\,}%
\@x{\@s{31.61} effective\_end\_ts \.{\mapsto} policyTf ,\,}%
\@x{\@s{31.61} billing\_policy \.{\mapsto}\@w{BySystemOnFinalize} ,\,}%
\@x{\@s{31.61} state \.{\mapsto}\@w{OnRisk} ,\,}%
 \@x{\@s{31.61} revision\_start\_timestamps \.{\mapsto} \{ {\langle} 0 ,\,
 policyTi {\rangle} \} ,\,}%
\@x{\@s{31.61} renewal\_start\_timestamps \.{\mapsto} \{ \} ,\,}%
\@x{\@s{31.61} pending\_modification \.{\mapsto} NoModification ]}%
\@x{\@s{16.4} \.{\land} mods \.{=} {\langle} {\rangle}}%
\@x{\@s{16.4} \.{\land} term\@s{1.71} \.{=} {\FALSE}}%
\@pvspace{8.0pt}%
\@x{ Next \.{\defeq}}%
\@x{\@s{16.4} \.{\lor} acceptGeneric}%
 \@x{\@s{16.4} \.{\lor} \exists\, m \.{\in} Cancellations \.{:}
 createCancellation ( m )}%
 \@x{\@s{16.4} \.{\lor} \exists\, m \.{\in} Cancellations \.{:}
 issueCancellation ( m )}%
 \@x{\@s{16.4} \.{\lor} \exists\, m \.{\in} Cancellations \.{:} legacyLapse (
 m )}%
 \@x{\@s{16.4} \.{\lor} \exists\, m \.{\in} Reinstatements \.{:}
 draftReinstatement ( m )}%
 \@x{\@s{16.4} \.{\lor} \exists\, m \.{\in} Reinstatements \.{:}
 acceptReinstatement ( m )}%
\@x{\@s{16.4} \.{\lor} expireReinstatement}%
\@x{\@s{16.4} \.{\lor} issueReinstatement}%
\@x{\@s{16.4}}%
\@y{\@s{0}%
 \ensuremath{\.{\lor} \E\, m \.{\in} Reinstatements} :
 \ensuremath{legacyReinstatement(m)
}}%
\@xx{}%
\@x{\@s{16.4} \.{\lor} invalidateReinstatement}%
\@x{\@s{16.4} \.{\lor} doStep}%
\@x{\@s{16.4} \.{\lor} doTerm}%
 \@x{\@s{16.4} \.{\lor} ( term \.{=} {\TRUE} \.{\land} {\UNCHANGED} {\langle}
 policy ,\, step ,\, mods ,\, term {\rangle} )}%
\@pvspace{8.0pt}%
 \@x{ Spec \.{\defeq} Init \.{\land} {\Box} [ Next ]_{ {\langle} policy ,\,
 step ,\, mods ,\, term {\rangle}}}%
\@pvspace{8.0pt}%
\@x{}\bottombar\@xx{}%
\setboolean{shading}{false}
\begin{lcom}{0}%
\begin{cpar}{0}{F}{F}{0}{0}{}%
\ensuremath{\.{\,\backslash\,}}* Modification History
\end{cpar}%
\begin{cpar}{0}{F}{F}{0}{0}{}%
 \ensuremath{\.{\,\backslash\,}}* Last modified \ensuremath{Tue}
 \ensuremath{Sep} 15 17:44:30 \ensuremath{PDT} 2020 by \ensuremath{ASUS
}%
\end{cpar}%
\begin{cpar}{0}{F}{F}{0}{0}{}%
 \ensuremath{\.{\,\backslash\,}}* Last modified \ensuremath{Thu}
 \ensuremath{Jul} 16 16:36:58 \ensuremath{PDT} 2020 by \ensuremath{marco
}%
\end{cpar}%
\begin{cpar}{0}{F}{F}{0}{0}{}%
 \ensuremath{\.{\,\backslash\,}}* \ensuremath{Created} Sat \ensuremath{Jun} 27
 21:12:50 \ensuremath{PDT} 2020 by \ensuremath{ASUS
}%
\end{cpar}%
\end{lcom}%
