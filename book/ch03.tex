\motto{}
\chapter{Pervasive Concepts}
\label{intro03} % Always give a unique label
% use \chaptermark{}
% to alter or adjust the chapter heading in the running head

\abstract{
}

\section{Characteristics}
\label{sec:03:1}

Characteristics are containers that record a set of attribute values
(ex: number\_of\_drivers = 2, make\_of\_car = Honda, resale\_value=10000)
which a policy contract is based on, and additionally a time interval over which
those attribute values are valid. Abstractly, it's a combination of a map and a time
interval. 

Below we just consider the time interval part of a characteristics and detail
how those intervals split as the standard modification operations are applied
to intervals
\import{ch03/}{CharacteristicsSplitCtx0-sa.tex}
\import{ch03/}{CharacteristicsSplitMch0-sa.tex}

\section{Liquid Calculations}
\label{sec:03:2}

\import{ch03/}{CancellationCalculations-sa.tex}

\section{Intervals}
As you work with framework objects, you will soon come to see that most objects cover
an interval from $start\_timestamp$ to $end\_timestamp$. These times delineate an extent of
a coverage interval or in the case of a modification the times delineate the time range for
which a coverage change was requested.

There is also a second type of framework object which is very common and interval like. These
are temporal objects and are identifiable by the fields $issued\_timestamp$ and $replaced\_timestamp$.
More commonly you may have seen these fields in other frameworks by the names $valid\_from$ and
$valid\_to$. This simple structure is used to record an audit history and allow easy aquision of the current set
of valid objects with a filter of the form
\begin{equation*}
issuedTimestamp \neq null \land replacedTimestamp = null
\end{equation*}

Here are some standard interval functions that are available for use in the framework.
\import{ch03/}{IntervalOps-sa.tex}

\section{Holdbacks}
\label{sec:03:3}
\subsection{Concept}

\begin{figure}[b]
  \begin{tikzpicture}[->,>=stealth']
    \end{tikzpicture}
  \caption{
    A policy cancellation
  }
  \label{fig:3:1}
\end{figure}

\subsection{Specification}
\import{ch03/}{ProrationCtx2-sa.tex}
\import{ch03/}{ProrationMch2-sa.tex}

