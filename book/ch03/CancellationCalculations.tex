\documentclass[a4paper,11pt]{article}
\usepackage{amsmath}
\usepackage[retainorgcmds]{IEEEtrantools}

\begin{document}

\subsection{Rating Premiums, Taxes, Commissions}
Premiums, taxes and commissions are calculated at the peril level via the framework function, Rater.getPricedCharacteristics,
which has the simplified signature:
\begin{equation*}
  getPricedCharacteristics: RaterPriceRequest \to Map \; Locator \; PerilCharacteristics
\end{equation*}
The RaterPriceRequest contains a collection of unpriced (policy characteristic, exposure characteristic, peril characteristic) triplets
which have the appropriate field values which one intends to price.
\begin{equation*}
triple: (PolicyCharacteristics, \; ExposureCharacteristics, \; PerilCharacteristics)
\end{equation*}
The triplet is formed such that the interval of the
exposure characteristic contains the peril characteristics and the interval of the policy characteristic contains the
peril characteristic. This containment property, which has been discussed, dictates how peril characteristics split
their parent characteristics and how parent characteristics split peril characteristics. One can see that the containment property
is driven by the needs of the rater to have homogeneous coverage across a triplet of characteristics in order to price at the peril
level. The containment property also tells us why we are generally unable to price one peril based on properties of another peril.
Perils do not compare to one another as most generally a peril characteristic in one peril will not be contained by any
peril characteristics in another peril.

The RaterPriceRequest also contains a policy object. The policy object should satisfy the condition that for any
characteristics triplet that exists in the RaterPriceRequest each of those characteristics must also be discoverable by
iterating through the policy, exposure, peril tree. Also any policy\_modification that can be discovered on any
characteristics triplet must be discoverable in the policy.getModifications() collection.

With the request object formed, $req: (policy, \; triple)$, the base $rater$ function is called on each triple to obtain
a premium, commission, and tax
\begin{equation*}
premium = \Delta t \cdot rater(policy, triple_{peril}).premium
\end{equation*}
Similarly for commissions and taxes
\begin{eqnarray*}
commission & = & \Delta t \cdot rater(policy, triple_{peril}).commission \\
tax & = & rater(policy, triple_{peril}).tax
\end{eqnarray*}
Lastly, the $getPricedCharacteristics$ function assembles and returns new priced characteristics indexed by locator.

\subsection{Rating Fees}
Fees are calculated at the policy level via the framework function, LiquidRenderClient.renderPolicyCalculations, which has the
simplified signature:
\begin{equation*}
  renderPolicyCalculations: Context \to List \; Fee
\end{equation*}
The context contains a policy and the in process policy modification. The policy must contain fully calculated premiums, commission, and taxes
on all its characteristics. And the policy modification must have all gross premium and gross tax amounts calculated.
Then, a list of fees are calculated by calling the rater function as:
\begin{equation*}
renderPolicyCalculations = rater(policy', policyModification')
\end{equation*}

\end{document}

