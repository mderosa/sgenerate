\documentclass[a4paper,11pt]{article}
\usepackage{amsmath}
\usepackage[retainorgcmds]{IEEEtrantools}

\begin{document}

\subsection{Premiums, Taxes, Commissions}
Premiums, taxes and commissions are calculated at the peril level via the framework Rater.getPricedCharacteristics function
which has the simplified signature:
\begin{equation*}
  getPricedCharacteristics: RaterPriceRequest \to Map \; Locator \; PerilCharacteristics
\end{equation*}
The RaterPriceRequest contains a collection of unpriced (policy characteristic, exposure characteristic, peril characteristic) triplets
which have the appropriate field values which one intends to price. The triplet is formed such that the interval of the
exposure characteristic contains the peril characteristics and the interval of the policy characteristic contains the
peril characteristic. This containment property, which has been discussed, dictates how peril characteristics split
their parent characteristics and how parent characteristics split peril characteristics. One can see that the containment property
is driven by the needs of the rater to have homogeneous coverage across a triplet of characteristics in order to price at the peril
level. The containment property also tells us why we are generally unable to price one peril based on properties of another peril.
Perils do not compare to one another as most generally a peril characteristic in one peril will not be contained by any
peril characteristics in another peril.

The RaterPriceRequest also contains a policy object. The policy object should satisfy the condition that for any
characteristics triplet that exists in the RaterPriceRequest each of those characteristics can also be discovered by
iterating through the policy, exposure, peril tree. Also any policy\_modification that can be discovered on any
characteristics triplet can be be found in the policy.getModifications() collection.

Once the Rater.getPricedCharacterisitcs function is called, then the following operations are performed. Each
characteristics triplet is taken in turn
\begin{equation*}
Let \; obj_{peril} = (peril.policy.characteristics, \; peril.exposure.characteristics, \; peril.characteristics)
\end{equation*}
Then the liquid file is accessed for the peril and a premium is calculated
\begin{equation*}
GrossPremium = \sum_{p \in Perils} \Delta t \cdot p.liquid.premium(obj_p) 
\end{equation*}
Similarly for commissions
\begin{equation*}
GrossCommission = \sum_{p \in Perils} \Delta t \cdot p.liquid.commission(obj_p)
\end{equation*}
Premiums and commissions are then placed on the peril characteristic and tax calculations are made
\begin{equation*}
  GrossTax = \sum_{p \in Perils} \sum_{tx \in Taxes} tx.liquid.tax(obj'_p) 
\end{equation*}
Lastly, as can be seen from the signature of getPricedCharacteristics, above, the rater returns each of the fully priced peril characteristics, indexed
by locator.

\subsection{Fees}
Fees are calculated at the policy level via the framework LiquidRenderClient.renderPolicyCalculations function, which has the
simplified signature:
\begin{equation*}
  renderPolicyCalculations: Context \to List \; Fee
\end{equation*}
The context contains a policy and an indicator of the in process policy modification. The function expects the policy object to
have to modification applied in all detail, including having the above fully calculated premiums, commission, and taxes.
Then with this fully formed policy fees are calculated by accessing the policy.liquid file and calculating:
\begin{equation*}
GrossFee = \sum_{f \in Fees} f.liquid.fee(policy', policyModification')
\end{equation*}

\end{document}

