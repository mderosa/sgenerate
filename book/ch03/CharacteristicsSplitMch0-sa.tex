\tlatex
\setboolean{shading}{true}
 \@x{}\moduleLeftDash\@xx{ {\MODULE}
 CharacteristicsSplitMch0}\moduleRightDash\@xx{}%
\@x{ {\EXTENDS} CharacteristicsSplitCtx0 ,\, FiniteSets}%
\@pvspace{8.0pt}%
 \@x{ {\VARIABLES} policyChs ,\, exChs ,\, perilChs ,\, cPolicyChs ,\, cExChs
 ,\, cPerilChs ,\, pc}%
\@pvspace{8.0pt}%
\begin{lcom}{0}%
\begin{cpar}{0}{T}{F}{7.5}{0}{}%
There are three tables in the persistance layer that hold characteristics,
\end{cpar}%
\begin{cpar}{1}{F}{F}{0}{0}{}%
 \ensuremath{policy\_characteristics}, \ensuremath{exposure\_characteristics},
 and \ensuremath{peril\_characteristics}. All of these
 tables are examples of a well exstablished sql modeling structure called a
 temporal
 table. The purpose of that structure is to describe a current valid state
 and at the same
 time to record the entire transaction history, making it easily
 reconstructable for any
 given time.
 Here, I examine characteristic transformations for simple one exposure, one
 peril policy.
 the valid state of the policy\mbox{'}s characteristics are held in the
 variables \ensuremath{policyChs}, \ensuremath{exChs},
 and \ensuremath{perilChs}. Then, I mimic the history functionality of a
 temporal table by storing
 history in the variables \ensuremath{cPolicyChs}, \ensuremath{cExChs},
 \ensuremath{cPerilChs}. The framework itself only needs
 to access characteristic history during a cancellation reinstatement. So
 these variables
 hold characteristic state just before the application of the last
 cancellation.
\end{cpar}%
\end{lcom}%
\@x{}\midbar\@xx{}%
\@x{ max ( a ,\, b ) \.{\defeq} {\IF} a \.{\geq} b \.{\THEN} a \.{\ELSE} b}%
 \@x{ min ( a ,\, b )\@s{1.49} \.{\defeq} {\IF} a \.{\leq} b \.{\THEN} a
 \.{\ELSE} b}%
\@pvspace{8.0pt}%
\@x{}%
\@y{\@s{0}%
 Interval Predicates
}%
\@xx{}%
\begin{lcom}{0}%
\begin{cpar}{0}{T}{F}{7.5}{0}{}%
 Since characteristics are intervals, making decisions on them is facilitated
 by a
\end{cpar}%
\begin{cpar}{1}{F}{F}{0}{0}{}%
 fundamental set of predicate combinators over intervals. The fundamental
 interval
 combinators sufficient for all characteristic decisions at
 \ensuremath{Socotra} are listedbelow. Most
 of them are implemented in the framework Duration class (or when not, are
 implemented in
 an ad-hoc manner).
\end{cpar}%
\end{lcom}%
 \@x{ notOverlaps ( chA ,\, chB ) \.{\defeq} chB . end\_ts \.{\leq} chA .
 start\_ts}%
\@x{\@s{16.4} \.{\lor} chB . start\_ts \.{\geq} chA . end\_ts}%
\@pvspace{8.0pt}%
 \@x{ overlaps ( chA ,\, chB )\@s{0.63} \.{\defeq} {\lnot} notOverlaps ( chA
 ,\, chB )}%
\@pvspace{8.0pt}%
 \@x{ contains ( chA ,\, chB ) \.{\defeq} chA . start\_ts \.{\leq} chB .
 start\_ts \.{\land} chB . end\_ts \.{\leq} chA . end\_ts}%
\@pvspace{8.0pt}%
\@x{ equals ( chA ,\, chB ) \.{\defeq}}%
\@x{\@s{16.4} \.{\land} chA . start\_ts \.{=} chB . start\_ts}%
\@x{\@s{16.4} \.{\land} chA . end\_ts \.{=} chB . end\_ts}%
\@pvspace{8.0pt}%
\@x{ width ( ch ) \.{\defeq} ch . end\_ts \.{-} ch . start\_ts}%
\@pvspace{8.0pt}%
 \@x{ within ( tm ,\, ch ) \.{\defeq} ch . start\_ts \.{<} tm \.{\land} tm
 \.{<} ch . end\_ts}%
\@pvspace{8.0pt}%
\@x{}%
\@y{\@s{0}%
 Interval Functions
}%
\@xx{}%
\begin{lcom}{0}%
\begin{cpar}{0}{T}{F}{7.5}{0}{}%
 There are two inverval functions which are the fundamental combinators on
 which all
\end{cpar}%
\begin{cpar}{1}{F}{F}{0}{0}{}%
characteristic functions are based. These are:
 (1) subtract: Interval \ensuremath{\.{\rightarrow}} Interval
 \ensuremath{\.{\rightarrow}} Set Interval, returns the parts of the first
 interval
 that do not overlap with the second interval. And will be a set of
 cardinality between
 0 and 2.
 (2) union: Interval \ensuremath{\.{\rightarrow} Inteval \.{\rightarrow}} Set
 Interval, returns the part of the first and second
 intervals that overlap. And will be a set of cardinality between 0 and 1.
\end{cpar}%
\end{lcom}%
\@x{ subtract ( chA ,\, chB ) \.{\defeq}}%
\@x{\@s{16.4} {\CASE} notOverlaps ( chA ,\, chB ) \.{\rightarrow} \{ chA \}}%
\@x{\@s{16.4} {\Box} {\OTHER} \.{\rightarrow}}%
\@x{\@s{27.97} {\IF} within ( chB . start\_ts ,\, chA ) \.{\THEN}}%
\@x{\@s{44.22} {\IF} within ( chB . end\_ts ,\, chA ) \.{\THEN} \{}%
 \@x{\@s{60.48} [ start\_ts \.{\mapsto} chA . start\_ts ,\, end\_ts
 \.{\mapsto} chB . start\_ts ] ,\,}%
 \@x{\@s{60.48} [ start\_ts \.{\mapsto} chB . end\_ts ,\, end\_ts \.{\mapsto}
 chA . end\_ts ]}%
\@x{\@s{60.48} \}}%
\@x{\@s{44.22} \.{\ELSE}}%
 \@x{\@s{60.62} \{ [ start\_ts \.{\mapsto} chA . start\_ts ,\, end\_ts
 \.{\mapsto} chB . start\_ts ] \}}%
\@x{\@s{27.97} \.{\ELSE}}%
\@x{\@s{44.37} {\IF} within ( chB . end\_ts ,\, chA ) \.{\THEN}}%
 \@x{\@s{60.62} \{ [ start\_ts \.{\mapsto} chB . end\_ts ,\, end\_ts
 \.{\mapsto} chA . end\_ts ] \}}%
\@x{\@s{44.37} \.{\ELSE}}%
\@x{\@s{60.77} \{ \}}%
\@pvspace{8.0pt}%
\@x{ union ( chA ,\, chB ) \.{\defeq}}%
\@x{\@s{16.4} {\IF} notOverlaps ( chA ,\, chB )}%
\@x{\@s{16.4} \.{\THEN} \{ \}}%
\@x{\@s{16.4} \.{\ELSE} \{ [}%
 \@x{\@s{60.91} start\_ts \.{\mapsto} max ( chA . start\_ts ,\, chB .
 start\_ts ) ,\,}%
\@x{\@s{60.91} end\_ts \.{\mapsto} min ( chA . end\_ts ,\, chB . end\_ts )}%
\@x{\@s{47.71} ] \}}%
\@pvspace{8.0pt}%
\@x{}%
\@y{\@s{0}%
 Other functions
}%
\@xx{}%
\@x{ maxEndTs ( chs ) \.{\defeq}}%
 \@x{\@s{16.4} \.{\LET}\@s{10.70} endTss \.{\defeq} \{ ch . end\_ts \.{:} ch
 \.{\in} chs \}}%
 \@x{\@s{16.4} \.{\IN} {\CHOOSE} tm \.{\in} endTss \.{:} \A\, el \.{\in}
 endTss \.{:} tm \.{\geq} el}%
\@pvspace{8.0pt}%
\@x{ minStartTs ( chs ) \.{\defeq}}%
 \@x{\@s{16.4} \.{\LET} startTss \.{\defeq} \{ ch . start\_ts \.{:} ch \.{\in}
 chs \}}%
 \@x{\@s{16.4} \.{\IN} {\CHOOSE} tm \.{\in} startTss \.{:} \A\, el\@s{1.66}
 \.{\in} startTss \.{:} tm \.{\leq} el}%
\@pvspace{8.0pt}%
\@x{ one ( es ) \.{\defeq} {\CHOOSE} e \.{\in} es \.{:} {\TRUE}}%
\@pvspace{8.0pt}%
\begin{lcom}{0}%
\begin{cpar}{0}{T}{F}{7.5}{0}{}%
 In the socotra product a policy always has at least one policy
 characteristic. One might
\end{cpar}%
\begin{cpar}{1}{F}{F}{0}{0}{}%
 think that when a policy is fully cancelled it would have 0 policy
 characteristics but
 this is not the case. A fully cancelled policy has a single characteristic,
 fully
 populated in all respects but with zero width.
\end{cpar}%
\end{lcom}%
\@x{ offPolicy \.{\defeq}}%
\@x{\@s{16.4} \.{\land} Cardinality ( policyChs ) \.{=} 1}%
\@x{\@s{16.4} \.{\land} \.{\LET} ch \.{\defeq} one ( policyChs )}%
\@x{\@s{27.51} \.{\IN} width ( ch ) \.{=} 0}%
\@pvspace{8.0pt}%
\@x{ onPolicy \.{\defeq} {\lnot} offPolicy}%
\@x{}\midbar\@xx{}%
\begin{lcom}{0}%
\begin{cpar}{0}{T}{F}{7.5}{0}{}%
 For a policy\mbox{'}s characteristic data to be correct. All manipulations
 should retain the
\end{cpar}%
\begin{cpar}{1}{F}{F}{0}{0}{}%
properties below, with some caveats as noted below.
\end{cpar}%
\vshade{5.0}%
\begin{cpar}{0}{T}{F}{2.5}{0}{}%
 Containment Property: Each peril characteristic must be contained by one
 characteristic
\end{cpar}%
\begin{cpar}{1}{F}{F}{0}{0}{}%
 in its parent exposure. Each peril characteristic must be contained by one
 characteristic
 in its policy. There are no containment relationships, however, between
 policy and
 exposure characteristics. The effect of this property is that splits of peril
 characteristics migrate up to split both exposure and policy
 characteristics, while splits
 of policy or exposure characteristics only migrate down to peril
 characteristics.
\end{cpar}%
\end{lcom}%
\@x{ InvContainment \.{\defeq}}%
 \@x{\@s{16.4} \.{\land} \A\, ch \.{\in} perilChs \.{:} ( \E\, pch \.{\in}
 policyChs \.{:} contains ( pch ,\, ch ) )}%
 \@x{\@s{16.4} \.{\land} \A\, ch \.{\in} perilChs \.{:} ( \E\, ech\@s{0.51}
 \.{\in} exChs \.{:} contains ( ech ,\, ch ) )}%
\@pvspace{8.0pt}%
\begin{lcom}{0}%
\begin{cpar}{0}{T}{F}{7.5}{0}{}%
 Existance Property: a policy, exposure, or peril always has at least one
 characteristic.
\end{cpar}%
\begin{cpar}{1}{F}{F}{0}{0}{}%
 In the concrete implementation this property sometimes will not hold after a
 \ensuremath{reduction()
 } operation like cancellation. Specifically if the start date of a
 cancellation operation
 is less than the minimum start date of a set of characteristics then all the
 characteristics
 will be removed.
\end{cpar}%
\end{lcom}%
\@x{ InvNotEmpty \.{\defeq} Cardinality ( policyChs ) \.{>} 0}%
\@x{\@s{16.4} \.{\land} Cardinality ( exChs ) \.{>} 0}%
\@x{\@s{16.4} \.{\land} Cardinality ( perilChs ) \.{>} 0}%
\@pvspace{8.0pt}%
\begin{lcom}{0}%
\begin{cpar}{0}{T}{F}{7.5}{0}{}%
 Off \ensuremath{Policy} Property: When a policy has been fully cancelled each
 node in the
\end{cpar}%
\begin{cpar}{1}{F}{F}{0}{0}{}%
 characteristics tree contains 1 characteristic of zero length. In the
 concrete implementation
 there is some deviance here as explained in the \ensuremath{Existance}
 Property. Of particular note,
 exposure characteristics do not behave like policy or peril characteristics
 which grow
 and shrink with operations. Exposure characteristics only grow in size. The
 minimum
 start date in a set of exposure characteristics is always equal to the start
 date of the
 exposure when created. The maximum end date in a set of exposure
 characteristics is
 always equal to the maximum policy end date that the set of exposures has
 observed during
 its lifetime.
\end{cpar}%
\end{lcom}%
\@x{ InvOffPolicy \.{\defeq} offPolicy \.{\implies}}%
\@x{\@s{49.19} (}%
\@x{\@s{65.6} \.{\land} Cardinality ( exChs ) \.{=} 1}%
\@x{\@s{65.6} \.{\land} Cardinality ( perilChs ) \.{=} 1}%
\@x{\@s{65.6} \.{\land} \.{\LET} pch \.{\defeq} one ( policyChs )}%
\@x{\@s{97.11} ech\@s{0.51} \.{\defeq} one ( exChs )}%
\@x{\@s{97.11} prlCh \.{\defeq} one ( perilChs )}%
\@x{\@s{76.71} \.{\IN} width ( ech ) \.{=} 0}%
\@x{\@s{80.81} \.{\land} width ( prlCh ) \.{=} 0}%
\@x{\@s{80.81} \.{\land} ech . start\_ts\@s{5.63} \.{=} pch . start\_ts}%
\@x{\@s{80.81} \.{\land} ech . end\_ts \.{=} pch . end\_ts}%
\@x{\@s{80.81} \.{\land} prlCh . start\_ts \.{=} pch . start\_ts}%
\@x{\@s{80.81} \.{\land} prlCh . end\_ts \.{=} pch . end\_ts}%
\@x{\@s{49.19} )}%
\@pvspace{8.0pt}%
\@x{}%
\@y{\@s{0}%
 On \ensuremath{Policy} Property: When a policy is active, it has no zero
 length characteristics
}%
\@xx{}%
\@x{ InvOnPolicyNoZeroLengthChs \.{\defeq} onPolicy \.{\implies} (}%
 \@x{\@s{16.4} \.{\land} \A\, prlCh \.{\in} perilChs \.{:} prlCh . start\_ts
 \.{<} prlCh . end\_ts}%
 \@x{\@s{16.4} \.{\land} \A\, exCh \.{\in} exChs \.{:} exCh . start\_ts \.{<}
 exCh . end\_ts}%
 \@x{\@s{16.4} \.{\land} \A\, ch \.{\in} policyChs \.{:} ch .
 start\_ts\@s{8.03} \.{<} ch . end\_ts}%
\@x{\@s{16.4} )}%
\@pvspace{8.0pt}%
\begin{lcom}{7.5}%
\begin{cpar}{0}{F}{F}{0}{0}{}%
 Bounding Property: For all the exposures characteristics at least 1 must
 start the
 policy interval; for all the peril characteristics at least one must start
 the policy
 interval. The same is true for finishing the policy interval.
\end{cpar}%
\end{lcom}%
 \@x{ InvBoundsStart \.{\defeq} \.{\LET} start \.{\defeq} minStartTs (
 policyChs )}%
 \@x{\@s{16.4} \.{\IN} \.{\land} \E\, ech\@s{0.51} \.{\in} exChs \.{:} ech .
 start\_ts \.{=} start}%
 \@x{\@s{36.79} \.{\land} \E\, pch \.{\in} perilChs \.{:} pch . start\_ts
 \.{=} start}%
\@x{ InvBoundsEnd \.{\defeq} \.{\LET} end \.{\defeq} maxEndTs ( policyChs )}%
 \@x{\@s{16.4} \.{\IN} \.{\land} \E\, ech\@s{0.51} \.{\in} exChs \.{:} ech .
 end\_ts \.{=} end}%
 \@x{\@s{36.79} \.{\land} \E\, pch \.{\in} perilChs \.{:} pch . end\_ts \.{=}
 end}%
\@pvspace{8.0pt}%
\@x{}\midbar\@xx{}%
\begin{lcom}{0}%
\begin{cpar}{0}{T}{F}{7.5}{0}{}%
 The various modification operations in the framework affect characteristics
 in 3
\end{cpar}%
\begin{cpar}{1}{F}{F}{0}{0}{}%
 abstract ways. \ensuremath{Api.create()}, \ensuremath{Api.reinstate()}, and
 \ensuremath{Api.renew()} modifications are abstractly
 \ensuremath{extend()} operations. \ensuremath{Api.cancel()} is abstractly a
 \ensuremath{reduce()} operation. And \ensuremath{Api.endorse()},
 \ensuremath{Api.update()} are abstractly \ensuremath{update()} operations.
 Below I discuss the character of these
 abstract operations in more detail.
\end{cpar}%
\vshade{5.0}%
\begin{cpar}{0}{T}{F}{2.5}{0}{}%
From the point of view of characteristics a reinstatement can be thought of as
\end{cpar}%
\begin{cpar}{1}{F}{F}{0}{0}{}%
 application of the set union operator. For example, given an existing set of
 characteristics,
\end{cpar}%
\begin{cpar}{0}{F}{F}{0}{0}{}%
 \ensuremath{\{i1,\, i2,\, i3\}} and a set that are being reinstated,
 \ensuremath{\{i4,\, i5\}}. The result is
\end{cpar}%
\begin{cpar}{0}{F}{F}{0}{0}{}%
 \ensuremath{\{i1,\, i2,\, i3\} \.{+} \{i4,\, i5\} \.{=} \{i1,\, i2,\, i3,\,
 i4,\, i5\}
}%
\end{cpar}%
\end{lcom}%
\@x{ reinstate ( mod ) \.{\defeq}}%
\@x{\@s{16.4} \.{\land} pc \.{<} 5}%
\@x{\@s{16.4} \.{\land} cPolicyChs \.{\neq} \{ \}}%
\@x{\@s{16.4} \.{\land} mod . start\_ts \.{<} mod . end\_ts}%
\@x{\@s{16.4} \.{\land} mod . start\_ts \.{\geq} minStartTs ( cPolicyChs )}%
\@x{\@s{16.4} \.{\land} mod . end\_ts \.{=} maxEndTs ( cPolicyChs )}%
\@x{\@s{16.4} \.{\land} {\IF} offPolicy}%
 \@x{\@s{31.61} \.{\THEN} \.{\land} policyChs \.{'} \.{=} {\UNION} \{ union (
 plcyCh ,\, mod ) \.{:} plcyCh \.{\in} cPolicyChs \}}%
 \@x{\@s{62.92} \.{\land} exChs \.{'} \.{=} {\UNION} \{ union ( exCh ,\, mod )
 \.{:} exCh \.{\in} cExChs \}}%
 \@x{\@s{62.92} \.{\land} perilChs \.{'} \.{=} {\UNION} \{ union ( prlCh ,\,
 mod ) \.{:} prlCh \.{\in} cPerilChs \}}%
\@x{\@s{31.61} \.{\ELSE} \.{\land} policyChs \.{'} \.{=} {\UNION} \{}%
\@x{\@s{90.43} policyChs ,\,}%
 \@x{\@s{90.43} {\UNION} \{ union ( plcyCh ,\, mod ) \.{:} plcyCh \.{\in}
 cPolicyChs \} \}}%
\@x{\@s{62.92} \.{\land} exChs \.{'} \.{=} {\UNION} \{}%
\@x{\@s{90.43} exChs ,\,}%
 \@x{\@s{90.43} {\UNION} \{ union ( exCh ,\, mod ) \.{:} exCh \.{\in} cExChs
 \} \}}%
\@x{\@s{62.92} \.{\land} perilChs \.{'} \.{=} {\UNION} \{}%
\@x{\@s{90.43} perilChs ,\,}%
 \@x{\@s{90.43} {\UNION} \{ union ( prlCh ,\, mod ) \.{:} prlCh \.{\in}
 cPerilChs \} \}}%
\@x{\@s{16.4} \.{\land} cPolicyChs \.{'} \.{=} \{ \}}%
\@x{\@s{16.4} \.{\land} cExChs \.{'} \.{=} \{ \}}%
\@x{\@s{16.4} \.{\land} cPerilChs \.{'} \.{=} \{ \}}%
\@x{\@s{16.4} \.{\land} pc \.{'} \.{=} pc \.{+} 1}%
\@pvspace{8.0pt}%
\begin{lcom}{0}%
\begin{cpar}{0}{T}{F}{7.5}{0}{}%
 From the point of view of characteristics a cancellation is the application
 of the
\end{cpar}%
\begin{cpar}{1}{F}{F}{0}{0}{}%
 subtract operator, above, over the set of existing characteristics. For
 illustration,
 consider the intervals
\end{cpar}%
\begin{cpar}{0}{F}{F}{0}{0}{}%
 \ensuremath{\.{\vdash}}\cdash{4} A \cdash{6}
 \ensuremath{\.{\vdash}}\@s{15.0}\cdash{6} \ensuremath{B}\@s{15.0}\cdash{6}
 \ensuremath{\.{\vdash}}\@s{15.0}\cdash{5} \ensuremath{C}\@s{12.5}\cdash{5}
 \ensuremath{\.{\,|\,}}\@s{15.0}an existing set of characteristics
\end{cpar}%
\begin{cpar}{0}{F}{F}{0}{0}{}%
 \ensuremath{\.{\vdash}}\cdash{14} \ensuremath{D}\cdash{16}
 \ensuremath{\.{\,|\,}\@s{42.5}a} cancellation interval. Subtracting
 \ensuremath{D} from each of A,
 \ensuremath{B}, and \ensuremath{C} leaves just a short prefix of A, which we
 write mathematically as
\end{cpar}%
\begin{cpar}{0}{F}{F}{0}{0}{}%
\ensuremath{\{subtract(x,\, D) \.{\,|\,} x} \ensuremath{in \{A,\, B,\, C\}\}
}%
\end{cpar}%
\end{lcom}%
\@x{ cancel ( mod ) \.{\defeq}}%
\@x{\@s{16.4} \.{\LET} startTs \.{\defeq} minStartTs ( policyChs )}%
\@x{\@s{16.4} \.{\IN}}%
\@x{\@s{16.4} \.{\land} pc \.{<} 5}%
\@x{\@s{16.4} \.{\land} cPolicyChs \.{=} \{ \}}%
\@x{\@s{16.4} \.{\land} mod . start\_ts \.{<} mod . end\_ts}%
\@x{\@s{16.4} \.{\land} mod . start\_ts \.{\geq} startTs}%
\@x{\@s{16.4} \.{\land} mod . end\_ts \.{=} maxEndTs ( policyChs )}%
\@x{\@s{16.4} \.{\land} pc \.{'} \.{=} pc \.{+} 1}%
\@x{\@s{16.4} \.{\land} {\IF} mod . start\_ts \.{=} startTs}%
 \@x{\@s{31.61} \.{\THEN} \.{\land} policyChs \.{'} \.{=} \{ [ start\_ts
 \.{\mapsto} startTs ,\, end\_ts \.{\mapsto} startTs ] \}}%
 \@x{\@s{62.92} \.{\land} exChs \.{'} \.{=} \{ [ start\_ts \.{\mapsto} startTs
 ,\, end\_ts \.{\mapsto} startTs ] \}}%
 \@x{\@s{62.92} \.{\land} perilChs \.{'} \.{=} \{ [ start\_ts \.{\mapsto}
 startTs ,\, end\_ts \.{\mapsto} startTs ] \}}%
\@x{\@s{62.92} \.{\land} cPolicyChs \.{'} \.{=} policyChs}%
\@x{\@s{62.92} \.{\land} cExChs \.{'} \.{=} exChs}%
\@x{\@s{62.92} \.{\land} cPerilChs \.{'} \.{=} perilChs}%
 \@x{\@s{31.61} \.{\ELSE} \.{\land} policyChs \.{'}\@s{1.03} \.{=} {\UNION} \{
 subtract ( plcyCh ,\, mod ) \.{:} plcyCh \.{\in} policyChs \}}%
 \@x{\@s{62.92} \.{\land} exChs \.{'} \.{=} {\UNION} \{ subtract ( exCh ,\,
 mod ) \.{:} exCh \.{\in} exChs \}}%
 \@x{\@s{62.92} \.{\land} perilChs \.{'} \.{=} {\UNION} \{ subtract ( prlCh
 ,\, mod )\@s{8.26} \.{:} prlCh \.{\in} perilChs \}}%
 \@x{\@s{62.92} \.{\land} cPolicyChs \.{'} \.{=} {\UNION} \{ union ( plcyCh
 ,\, mod ) \.{:} plcyCh \.{\in} policyChs \}}%
 \@x{\@s{62.92} \.{\land} cExChs \.{'} \.{=} {\UNION} \{ union ( exCh ,\, mod
 ) \.{:} exCh \.{\in} exChs \}}%
 \@x{\@s{62.92} \.{\land} cPerilChs \.{'} \.{=} {\UNION} \{ union ( prlCh ,\,
 mod ) \.{:} prlCh \.{\in} perilChs \}}%
\@pvspace{8.0pt}%
\begin{lcom}{0}%
\begin{cpar}{0}{T}{F}{7.5}{0}{}%
 From the point of view of characteristics an endorsement is the application
 of the
\end{cpar}%
\begin{cpar}{1}{F}{F}{0}{0}{}%
 subtract operator over the set of existing characteristics
 (set\ensuremath{\.{-})}unioned with the the
 application of the (interval\ensuremath{\.{-})}union operator, above, over
 the set of existing
 characteristics. The reader can verify this by drawing two sets of intervals
 on a piece
 of paper, following illustration for the cancel function, and visually doing
 the
 subtractions and unions. We can write the endorsement splitting process
 mathematically as
\end{cpar}%
\begin{cpar}{0}{F}{F}{0}{0}{}%
 \ensuremath{\{subtract(x,\, D) \.{\,|\,} x} \ensuremath{in \{A,\, B,\, C\}\}
 \.{+} \{union(x,\, D) \.{\,|\,} x} \ensuremath{in \{A,\, B,\, C\}\}
}%
\end{cpar}%
\end{lcom}%
\@x{ endorsePolicy ( mod ) \.{\defeq}}%
\@x{\@s{16.4} \.{\land} pc \.{<} 5}%
\@x{\@s{16.4} \.{\land} cPolicyChs \.{=} \{ \}}%
\@x{\@s{16.4} \.{\land} mod . start\_ts \.{<} mod . end\_ts}%
\@x{\@s{16.4} \.{\land} mod . start\_ts \.{\geq} minStartTs ( policyChs )}%
\@x{\@s{16.4} \.{\land} mod . end\_ts \.{\leq} maxEndTs ( policyChs )}%
\@x{\@s{16.4} \.{\land} pc \.{'} \.{=} pc \.{+} 1}%
\@x{\@s{16.4} \.{\land} policyChs \.{'} \.{=} {\UNION} \{}%
 \@x{\@s{31.61} {\UNION} \{ subtract ( policyCh ,\, mod ) \.{:} policyCh
 \.{\in} policyChs \} ,\,}%
 \@x{\@s{31.61} {\UNION} \{ union ( policyCh ,\, mod ) \.{:} policyCh \.{\in}
 policyChs \} \}}%
\@x{\@s{16.4} \.{\land} perilChs \.{'} \.{=} {\UNION} \{}%
 \@x{\@s{48.01} {\UNION} \{ subtract ( prlCh ,\, mod ) \.{:} prlCh \.{\in}
 perilChs \} ,\,}%
 \@x{\@s{48.01} {\UNION} \{ union ( prlCh ,\, mod ) \.{:} prlCh \.{\in}
 perilChs \} \}}%
 \@x{\@s{16.4} \.{\land} {\UNCHANGED} {\langle} exChs ,\, cPolicyChs ,\,
 cExChs ,\, cPerilChs {\rangle}}%
\@pvspace{8.0pt}%
\@x{ endorseEx ( mod ) \.{\defeq}}%
\@x{\@s{16.4} \.{\land} pc \.{<} 5}%
\@x{\@s{16.4} \.{\land} cPolicyChs \.{=} \{ \}}%
\@x{\@s{16.4} \.{\land} mod . start\_ts \.{<} mod . end\_ts}%
\@x{\@s{16.4} \.{\land} mod . start\_ts \.{\geq} minStartTs ( exChs )}%
\@x{\@s{16.4} \.{\land} mod . end\_ts \.{\leq} maxEndTs ( exChs )}%
\@x{\@s{16.4} \.{\land} pc \.{'} \.{=} pc \.{+} 1}%
\@x{\@s{16.4} \.{\land} exChs \.{'} \.{=} {\UNION} \{}%
 \@x{\@s{31.61} {\UNION} \{ subtract ( exCh ,\, mod ) \.{:} exCh \.{\in} exChs
 \} ,\,}%
 \@x{\@s{31.61} {\UNION} \{ union ( exCh ,\, mod ) \.{:} exCh \.{\in} exChs \}
 \}}%
\@x{\@s{16.4} \.{\land} perilChs \.{'} \.{=} {\UNION} \{}%
 \@x{\@s{48.01} {\UNION} \{ subtract ( prlCh ,\, mod ) \.{:} prlCh \.{\in}
 perilChs \} ,\,}%
 \@x{\@s{48.01} {\UNION} \{ union ( prlCh ,\, mod ) \.{:} prlCh \.{\in}
 perilChs \} \}}%
 \@x{\@s{16.4} \.{\land} {\UNCHANGED} {\langle} policyChs ,\, cPolicyChs ,\,
 cExChs ,\, cPerilChs {\rangle}}%
\@pvspace{8.0pt}%
\@x{ endorsePeril ( mod ) \.{\defeq}}%
\@x{\@s{16.4} \.{\land} pc \.{<} 5}%
\@x{\@s{16.4} \.{\land} cPolicyChs \.{=} \{ \}}%
\@x{\@s{16.4} \.{\land} mod . start\_ts \.{<} mod . end\_ts}%
\@x{\@s{16.4} \.{\land} mod . start\_ts \.{\geq} minStartTs ( perilChs )}%
\@x{\@s{16.4} \.{\land} mod . end\_ts \.{\leq} maxEndTs ( perilChs )}%
\@x{\@s{16.4} \.{\land} pc \.{'} \.{=} pc \.{+} 1}%
\@x{\@s{16.4} \.{\land} perilChs \.{'} \.{=} {\UNION} \{}%
 \@x{\@s{100.36} {\UNION} \{ subtract ( prlCh ,\, mod ) \.{:} prlCh \.{\in}
 perilChs \} ,\,}%
 \@x{\@s{100.36} {\UNION} \{ union ( prlCh ,\, mod ) \.{:} prlCh \.{\in}
 perilChs \} \}}%
 \@x{\@s{16.4} \.{\land} {\UNCHANGED} {\langle} policyChs ,\, exChs ,\,
 cPolicyChs ,\, cExChs ,\, cPerilChs {\rangle}}%
\@pvspace{8.0pt}%
\begin{lcom}{0}%
\begin{cpar}{0}{T}{F}{7.5}{0}{}%
 Lastly, splitting characteristics for renewal is identical to the splitting
 process
\end{cpar}%
\begin{cpar}{1}{F}{F}{0}{0}{}%
 for reinstatements. It is just less general as for renewals there is always
 only a
 single added characteristics which is added in to the existing valid set.
\end{cpar}%
\end{lcom}%
\@x{ renew ( mod ) \.{\defeq}}%
\@x{\@s{16.4} \.{\land} pc \.{<} 5}%
\@x{\@s{16.4} \.{\land} cPolicyChs \.{=} \{ \}}%
\@x{\@s{16.4} \.{\land} mod . start\_ts \.{<} mod . end\_ts}%
\@x{\@s{16.4} \.{\land} mod . start\_ts \.{=} maxEndTs ( policyChs )}%
\@x{\@s{16.4} \.{\land} policyChs \.{'} \.{=} policyChs \.{\cup} \{ mod \}}%
\@x{\@s{16.4} \.{\land} exChs \.{'} \.{=} exChs \.{\cup} \{ mod \}}%
\@x{\@s{16.4} \.{\land} perilChs \.{'} \.{=} perilChs \.{\cup} \{ mod \}}%
\@x{\@s{16.4} \.{\land} pc \.{'} \.{=} pc \.{+} 1}%
 \@x{\@s{16.4} \.{\land} {\UNCHANGED} {\langle} cPolicyChs ,\, cExChs ,\,
 cPerilChs {\rangle}}%
\@pvspace{8.0pt}%
\@x{ term \.{\defeq}}%
\@x{\@s{16.4} \.{\land} pc \.{=} 5}%
 \@x{\@s{16.4} \.{\land} {\UNCHANGED} {\langle} policyChs ,\, exChs ,\,
 perilChs ,\, cPolicyChs ,\, cExChs ,\, cPerilChs ,\, pc {\rangle}}%
\@pvspace{8.0pt}%
\@x{}\midbar\@xx{}%
 \@x{ Init \.{\defeq} \.{\LET} ch \.{\defeq} [ start\_ts \.{\mapsto} 0 ,\,
 end\_ts \.{\mapsto} 3 ]}%
\@x{\@s{16.4} \.{\IN} \.{\land} policyChs \.{=} \{ ch \}}%
\@x{\@s{36.79} \.{\land} exChs \.{=} \{ ch \}}%
\@x{\@s{36.79} \.{\land} perilChs \.{=} \{ ch \}}%
 \@x{\@s{36.79} \.{\land} cPolicyChs \.{=} \{ \} \.{\land} cExChs \.{=} \{ \}
 \.{\land} cPerilChs \.{=} \{ \}}%
\@x{\@s{36.79} \.{\land} pc \.{=} 0}%
\@pvspace{8.0pt}%
\@x{ Next \.{\defeq}}%
\@x{\@s{16.4} \.{\lor} term}%
\@x{\@s{16.4} \.{\lor} \E\, ch \.{\in} VCharacteristics \.{:}}%
\@x{\@s{42.93} \.{\lor} cancel ( ch )}%
\@x{\@s{42.93} \.{\lor} reinstate ( ch )}%
\@x{\@s{42.93} \.{\lor} renew ( ch )}%
\@x{\@s{42.93} \.{\lor} endorsePeril ( ch )}%
\@x{\@s{42.93} \.{\lor} endorseEx ( ch )}%
\@x{\@s{42.93} \.{\lor} endorsePolicy ( ch )}%
\@pvspace{16.0pt}%
 \@x{ Spec \.{\defeq} Init \.{\land} {\Box} [ Next ]_{ {\langle} policyChs ,\,
 exChs ,\, perilChs ,\, cPolicyChs ,\, cExChs ,\, cPerilChs ,\, pc
 {\rangle}}}%
\@x{}\bottombar\@xx{}%
\setboolean{shading}{false}
\begin{lcom}{0}%
\begin{cpar}{0}{F}{F}{0}{0}{}%
\ensuremath{\.{\,\backslash\,}}* Modification History
\end{cpar}%
\begin{cpar}{0}{F}{F}{0}{0}{}%
 \ensuremath{\.{\,\backslash\,}}* Last modified \ensuremath{Tue}
 \ensuremath{Dec} 15 15:14:20 \ensuremath{PST} 2020 by \ensuremath{ASUS
}%
\end{cpar}%
\begin{cpar}{0}{F}{F}{0}{0}{}%
 \ensuremath{\.{\,\backslash\,}}* \ensuremath{Created} \ensuremath{Fri}
 \ensuremath{Oct} 16 13:25:40 \ensuremath{PDT} 2020 by \ensuremath{ASUS
}%
\end{cpar}%
\end{lcom}%
