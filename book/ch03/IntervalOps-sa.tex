\tlatex
\setboolean{shading}{true}
\@x{}\moduleLeftDash\@xx{ {\MODULE} IntervalOps}\moduleRightDash\@xx{}%
\@x{ {\EXTENDS} Integers ,\, Reals}%
\@pvspace{8.0pt}%
\@x{ max ( a ,\, b ) \.{\defeq} {\IF} a \.{\geq} b \.{\THEN} a \.{\ELSE} b}%
 \@x{ min ( a ,\, b )\@s{1.49} \.{\defeq} {\IF} a \.{\leq} b \.{\THEN} a
 \.{\ELSE} b}%
\@pvspace{8.0pt}%
\@x{}%
\@y{\@s{0}%
 Interval Predicates
}%
\@xx{}%
\begin{lcom}{0}%
\begin{cpar}{0}{T}{F}{7.5}{0}{}%
 Since characteristics are intervals, making decisions on them is facilitated
 by a
\end{cpar}%
\begin{cpar}{1}{F}{F}{0}{0}{}%
 fundamental set of predicate combinators over intervals. The fundamental
 interval
 combinators sufficient for all characteristic decisions at
 \ensuremath{Socotra} are listedbelow. Most
 of them are implemented in the framework Duration class (or when not, are
 implemented in
 an ad-hoc manner).
\end{cpar}%
\end{lcom}%
 \@x{ notOverlaps ( chA ,\, chB ) \.{\defeq} chB . end\_ts \.{\leq} chA .
 start\_ts}%
\@x{\@s{16.4} \.{\lor} chB . start\_ts \.{\geq} chA . end\_ts}%
\@pvspace{8.0pt}%
\@x{ overlaps ( chA ,\, chB ) \.{\defeq} {\lnot} notOverlaps ( chA ,\, chB )}%
\@pvspace{8.0pt}%
 \@x{ starts ( chA ,\, chB ) \.{\defeq} chA . start\_ts \.{=} chB . start\_ts
 \.{\land} chA . end\_ts \.{<} chB . end\_ts}%
\@pvspace{8.0pt}%
 \@x{ contains ( chA ,\, chB ) \.{\defeq} chA . start\_ts \.{\leq} chB .
 start\_ts \.{\land} chB . end\_ts \.{\leq} chA . end\_ts}%
\@pvspace{8.0pt}%
\@x{ equals ( chA ,\, chB ) \.{\defeq}}%
\@x{\@s{16.4} \.{\land} chA . start\_ts \.{=} chB . start\_ts}%
\@x{\@s{16.4} \.{\land} chA . end\_ts \.{=} chB . end\_ts}%
\@pvspace{8.0pt}%
\@x{ width ( ch ) \.{\defeq} ch . end\_ts \.{-} ch . start\_ts}%
\@pvspace{8.0pt}%
 \@x{ within ( tm ,\, ch ) \.{\defeq} ch . start\_ts \.{<} tm \.{\land} tm
 \.{<} ch . end\_ts}%
\@pvspace{8.0pt}%
\@x{ {\RECURSIVE} overlapsAny ( \_ ,\, \_ )}%
\@x{ overlapsAny ( ch ,\, chs ) \.{\defeq}}%
\@x{\@s{16.4} {\IF} chs \.{=} \{ \} \.{\THEN} {\FALSE}}%
 \@x{\@s{16.4} \.{\ELSE} \.{\LET} tmp \.{\defeq} {\CHOOSE} x \.{\in} chs \.{:}
 {\TRUE}}%
\@x{\@s{47.71} \.{\IN} {\IF} overlaps ( ch ,\, tmp ) \.{\THEN} {\TRUE}}%
 \@x{\@s{68.11} \.{\ELSE} overlapsAny ( ch ,\, chs \.{\,\backslash\,} \{ tmp
 \} )}%
\@pvspace{8.0pt}%
\@x{ {\RECURSIVE} startsAny ( \_ ,\, \_ )}%
\@x{ startsAny ( ch ,\, chs ) \.{\defeq}}%
\@x{\@s{16.4} {\IF} chs \.{=} \{ \} \.{\THEN} {\FALSE}}%
 \@x{\@s{16.4} \.{\ELSE} \.{\LET} tmp \.{\defeq} {\CHOOSE} x \.{\in} chs \.{:}
 {\TRUE}}%
\@x{\@s{47.71} \.{\IN} {\IF} starts ( ch ,\, tmp ) \.{\THEN} {\TRUE}}%
 \@x{\@s{68.11} \.{\ELSE} startsAny ( ch ,\, chs \.{\,\backslash\,} \{ tmp \}
 )}%
\@pvspace{8.0pt}%
\@x{}%
\@y{\@s{0}%
 Interval Functions
}%
\@xx{}%
\begin{lcom}{0}%
\begin{cpar}{0}{T}{F}{7.5}{0}{}%
 There are two inverval functions which are the fundamental combinators on
 which all
\end{cpar}%
\begin{cpar}{1}{F}{F}{0}{0}{}%
characteristic functions are based. These are:
 (1) subtract: Interval \ensuremath{\.{\rightarrow}} Interval
 \ensuremath{\.{\rightarrow}} Set Interval, returns the parts of the first
 interval
 that do not overlap with the second interval. And will be a set of
 cardinality between
 0 and 2.
 (2) union: Interval \ensuremath{\.{\rightarrow} Inteval \.{\rightarrow}} Set
 Interval, returns the part of the first and second
 intervals that overlap. And will be a set of cardinality between 0 and 1.
\end{cpar}%
\end{lcom}%
\@x{ subtract ( chA ,\, chB ) \.{\defeq}}%
\@x{\@s{16.4} {\CASE} notOverlaps ( chA ,\, chB ) \.{\rightarrow} \{ chA \}}%
\@x{\@s{16.4} {\Box} {\OTHER} \.{\rightarrow}}%
\@x{\@s{27.97} {\IF} within ( chB . start\_ts ,\, chA ) \.{\THEN}}%
\@x{\@s{44.22} {\IF} within ( chB . end\_ts ,\, chA ) \.{\THEN} \{}%
 \@x{\@s{60.48} [ chA {\EXCEPT} {\bang} . start\_ts \.{=} chA . start\_ts ,\,
 {\bang} . end\_ts \.{=} chB . start\_ts ] ,\,}%
 \@x{\@s{60.48} [ chA {\EXCEPT} {\bang} . start\_ts \.{=} chB . end\_ts ,\,
 {\bang} . end\_ts \.{=} chA . end\_ts ]}%
\@x{\@s{60.48} \}}%
\@x{\@s{44.22} \.{\ELSE}}%
 \@x{\@s{60.62} \{ [ chA {\EXCEPT} {\bang} . start\_ts \.{=} chA . start\_ts
 ,\, {\bang} . end\_ts \.{=} chB . start\_ts ] \}}%
\@x{\@s{27.97} \.{\ELSE}}%
\@x{\@s{44.37} {\IF} within ( chB . end\_ts ,\, chA ) \.{\THEN}}%
 \@x{\@s{60.62} \{ [ chA {\EXCEPT} {\bang} . start\_ts \.{=} chB . end\_ts ,\,
 {\bang} . end\_ts \.{=} chA . end\_ts ] \}}%
\@x{\@s{44.37} \.{\ELSE}}%
\@x{\@s{60.77} \{ \}}%
\@pvspace{8.0pt}%
\@x{ union ( chA ,\, chB ) \.{\defeq}}%
 \@x{\@s{16.4} {\IF} overlaps ( chA ,\, chB ) \.{\lor} chA . end\_ts \.{=} chB
 . start\_ts \.{\lor} chB . end\_ts \.{=} chA . start\_ts}%
 \@x{\@s{16.4} \.{\THEN} \{ [ chA {\EXCEPT} {\bang} . start\_ts \.{=} min (
 chA . start\_ts ,\, chB . start\_ts ) ,\,}%
 \@x{\@s{114.65} {\bang} . end\_ts \.{=} max ( chA . end\_ts ,\, chB . end\_ts
 ) ] \}}%
\@x{\@s{16.4} \.{\ELSE} \{ \}}%
\@pvspace{8.0pt}%
\@x{ intersect ( chA ,\, chB ) \.{\defeq}}%
\@x{\@s{16.4} {\IF} notOverlaps ( chA ,\, chB )}%
\@x{\@s{16.4} \.{\THEN} \{ \}}%
 \@x{\@s{16.4} \.{\ELSE} \{ [ chA {\EXCEPT} {\bang} . start\_ts \.{=} max (
 chA . start\_ts ,\, chB . start\_ts ) ,\,}%
 \@x{\@s{114.65} {\bang} . end\_ts \.{=} min ( chA . end\_ts ,\, chB . end\_ts
 )}%
\@x{\@s{47.71} ] \}}%
\@pvspace{8.0pt}%
\@x{ extend ( chA ,\, chB ) \.{\defeq}}%
 \@x{\@s{16.4} [ chA {\EXCEPT} {\bang} . start\_ts \.{=} min ( chA . start\_ts
 ,\, chB . start\_ts ) ,\,}%
 \@x{\@s{78.34} {\bang} . end\_ts \.{=} max ( chA . end\_ts ,\, chB . end\_ts
 ) ]}%
\@pvspace{8.0pt}%
 \@x{ subtractAll ( chs ,\, chB ) \.{\defeq} {\UNION} \{ subtract ( chA ,\,
 chB ) \.{:} chA \.{\in} chs \}}%
\@pvspace{8.0pt}%
 \@x{ intersectAll ( chs ,\, chB ) \.{\defeq} {\UNION} \{ intersect ( chA ,\,
 chB ) \.{:} chA \.{\in} chs \}}%
\@pvspace{8.0pt}%
 \@x{ overlapsAll ( chs ,\, int ) \.{\defeq} \{ ch \.{\in} chs \.{:} overlaps
 ( ch ,\, int ) \}}%
\@pvspace{8.0pt}%
\@x{}\bottombar\@xx{}%
\setboolean{shading}{false}
\begin{lcom}{0}%
\begin{cpar}{0}{F}{F}{0}{0}{}%
\ensuremath{\.{\,\backslash\,}}* Modification History
\end{cpar}%
\begin{cpar}{0}{F}{F}{0}{0}{}%
 \ensuremath{\.{\,\backslash\,}}* Last modified \ensuremath{Wed}
 \ensuremath{Jul} 14 21:11:27 \ensuremath{PDT} 2021 by \ensuremath{ASUS
}%
\end{cpar}%
\begin{cpar}{0}{F}{F}{0}{0}{}%
 \ensuremath{\.{\,\backslash\,}}* Created \ensuremath{Mon} \ensuremath{Dec} 28
 09:50:41 \ensuremath{PST} 2020 by \ensuremath{ASUS
}%
\end{cpar}%
\end{lcom}%
