\tlatex
\setboolean{shading}{true}
\@x{}\moduleLeftDash\@xx{ {\MODULE} Policy}\moduleRightDash\@xx{}%
\@x{ {\EXTENDS} Integers ,\, Product ,\, PolicyCtx}%
\@pvspace{8.0pt}%
\@x{ {\CONSTANT} NoModification}%
\@pvspace{8.0pt}%
\begin{lcom}{7.5}%
\begin{cpar}{0}{F}{F}{0}{0}{}%
 Socotra has several ways of generating invoices. It can generate invoices
 when a
 modification is accepted, on a schedule like once a month, and, for premium
 reporting,
 on client request. To give a sense that there is some variety, I have
 defined a
 incomplete set of the billing policies below.
\end{cpar}%
\end{lcom}%
 \@x{ BillingPolicy \.{\defeq} \{\@w{BySystemOnFinalize}
 ,\,\@w{ByClientOnRequest} \}}%
\@pvspace{8.0pt}%
\begin{lcom}{7.5}%
\begin{cpar}{0}{F}{F}{0}{0}{}%
 It is useful to see policies has having just two states. They start off in
 the \ensuremath{OffRisk
 } state and transition to \ensuremath{OnRisk} with their first issued
 modifications. Policies can
 also transition back to \ensuremath{OffRisk} when they are fully cancelled.
\end{cpar}%
\end{lcom}%
\@x{ PolicyState \.{\defeq} \{\@w{OnRisk} ,\,\@w{OffRisk} \}}%
\@pvspace{8.0pt}%
\begin{lcom}{7.5}%
\begin{cpar}{0}{F}{F}{0}{0}{}%
The linear lifecycle of a \ensuremath{Modification} from start to issue is:
\end{cpar}%
\vshade{5.0}%
\begin{cpar}{0}{F}{F}{0}{0}{}%
(a) Draft - also commonly referred to as Create
\end{cpar}%
\vshade{5.0}%
\begin{cpar}{0}{F}{F}{0}{0}{}%
(\ensuremath{b}) Accepted - also commonly referred to as \ensuremath{Finalized
}%
\end{cpar}%
\vshade{5.0}%
\begin{cpar}{0}{F}{F}{0}{0}{}%
(c) Issued.
\end{cpar}%
\vshade{5.0}%
\begin{cpar}{0}{F}{F}{0}{0}{}%
 There are a few other states which pop up with modifications as well but are
 less often
 seen. Expired is particular to reinstatements which have an associated
 lifetime. If the
 reinstatement is not issued before its expiration. Then it expires.
 Invalidation moves
 a modification from the accepted state back into the draft state. And
 rescind completely
 discards a modification.
\end{cpar}%
\end{lcom}%
 \@x{ ModificationState \.{\defeq} \{\@w{Created} ,\,\@w{Finalized}
 ,\,\@w{Issued} ,\,\@w{Expired} ,\,\@w{Invalidated} ,\,\@w{Rescind} \}}%
\@pvspace{8.0pt}%
\@x{ TimeRange \.{\defeq} policyMinTs \.{\dotdot} policyMaxTs}%
\@pvspace{8.0pt}%
\begin{lcom}{7.5}%
\begin{cpar}{0}{T}{F}{0}{2.5}{}%
 There are more modification types than are listed below. These are just three
 that will
 represent the extend, reduce, override possibilities.
\end{cpar}%
\end{lcom}%
 \@x{ ModificationType \.{\defeq} \{\@w{Endorsement} ,\,\@w{Renewal}
 ,\,\@w{Cancellation} \}}%
\@pvspace{8.0pt}%
\begin{lcom}{7.5}%
\begin{cpar}{0}{F}{F}{0}{0}{}%
Modifcations all have the basic structure below. It is convinient to think of
 modifications as a elaborated time interval and the action of applying a
 modification to
 a policy as an interval extension or reduction. Note that in the database the
 \ensuremath{policy\_modification} table does not include an
 \ensuremath{end\_timestamp}. This absence turns out to
 be very problematical as in many places in the framework it is a crucial
 piece of
 information too have handy. Maybe this data mishap will get fixed in the
 future.
\end{cpar}%
\end{lcom}%
\@x{ Modification \.{\defeq} [}%
\@x{\@s{16.4} type \.{:} ModificationType ,\,}%
\@x{\@s{16.4} state \.{:} ModificationState ,\,}%
\@x{\@s{16.4} start\_timestamp \.{:} TimeRange ,\,}%
\@x{\@s{16.4} end\_timestamp \.{:} TimeRange ,\,}%
\@x{\@s{16.4} product\_revision \.{:} 0 \.{\dotdot} maxRevision}%
\@x{\@s{16.4} ]}%
\@pvspace{8.0pt}%
\@x{ InvModification ( mod ) \.{\defeq}}%
\@x{\@s{16.4} \.{\land} mod . start\_timestamp \.{\leq} mod . end\_timestamp}%
\@pvspace{8.0pt}%
\begin{lcom}{7.5}%
\begin{cpar}{0}{F}{F}{0}{0}{}%
 In retrospect, the simple policy representation below is not all that good
 for a
 discussion of policies, so here are some of facets of policy data to be
 familiar with.
 \ensuremath{original\_start\_timestamp} will be the start date when the
 policy issued, and will be
 \ensuremath{\.{\leq}} to the \ensuremath{effective\_start\_timestamp}. The
 \ensuremath{effective\_end\_timestamp} is always the date when
 the policy ends. The policy state is not explicitly tracked in policy data.
 But
 implicitly when the \ensuremath{effective\_start\_ts \.{=}
 effective\_end\_ts} then the policy is off policy
 and when \ensuremath{effective\_start\_ts \.{>} effective\_end\_ts} then the
 policy is on policy.
\end{cpar}%
\vshade{5.0}%
\begin{cpar}{0}{F}{F}{0}{0}{}%
 The \ensuremath{pending\_modification} is also not part of the policy data.
 Rather, here, it represents
 an important concept in dealing with policies that a policy must have either
 no
 modifications in the accepted state or one modification in the accepted
 state. An
 accepted modification acts like a global lock on the policy. No modifications
 can advance into the accepted state when another modification holds that
 ``accept lock''.
 The existing accepted modification must either be issued or invalidated for
 other
 modifications to be issued.
\end{cpar}%
\end{lcom}%
\@x{ Policy \.{\defeq} [}%
\@x{\@s{16.4} original\_start\_ts \.{:} TimeRange ,\,}%
\@x{\@s{16.4} effective\_end\_ts \.{:} TimeRange ,\,}%
\@x{\@s{16.4} billing\_policy \.{:} BillingPolicy ,\,}%
\@x{\@s{16.4} state \.{:} PolicyState ,\,}%
 \@x{\@s{16.4} revision\_start\_timestamps \.{:} {\SUBSET} ( ( 0 \.{\dotdot}
 maxRevision ) \.{\times} TimeRange ) ,\,}%
\@x{\@s{16.4} renewal\_start\_timestamps \.{:} {\SUBSET} ( TimeRange ) ,\,}%
 \@x{\@s{16.4} pending\_modification \.{:} Modification \.{\cup} \{
 NoModification \}}%
\@x{\@s{16.4} ]}%
\@pvspace{8.0pt}%
\@x{}\bottombar\@xx{}%
\setboolean{shading}{false}
\begin{lcom}{0}%
\begin{cpar}{0}{F}{F}{0}{0}{}%
\ensuremath{\.{\,\backslash\,}}* \ensuremath{Modification} History
\end{cpar}%
\begin{cpar}{0}{F}{F}{0}{0}{}%
 \ensuremath{\.{\,\backslash\,}}* Last modified \ensuremath{Mon}
 \ensuremath{Oct} 11 14:08:06 \ensuremath{PDT} 2021 by \ensuremath{marco
}%
\end{cpar}%
\begin{cpar}{0}{F}{F}{0}{0}{}%
 \ensuremath{\.{\,\backslash\,}}* Last modified \ensuremath{Mon}
 \ensuremath{Jul} 13 20:22:48 \ensuremath{PDT} 2020 by \ensuremath{ASUS
}%
\end{cpar}%
\begin{cpar}{0}{F}{F}{0}{0}{}%
 \ensuremath{\.{\,\backslash\,}}* \ensuremath{Created} Sat \ensuremath{Jun} 27
 14:05:18 \ensuremath{PDT} 2020 by marcderosa
\end{cpar}%
\end{lcom}%
