\tlatex
\setboolean{shading}{true}
\@x{}\moduleLeftDash\@xx{ {\MODULE} ProrationMch2}\moduleRightDash\@xx{}%
 \@x{ {\EXTENDS} ProrationCtx2 ,\, Sequences ,\, IntervalOps ,\, FiniteSets
 ,\, TLC}%
\@pvspace{8.0pt}%
\@x{ {\VARIABLES} policy ,\, policyChs ,\, pc ,\, cancelQ ,\, holdbacks}%
\@pvspace{8.0pt}%
\@x{ {\CONSTANTS} maxPc ,\, maxT ,\, pluginActive}%
\@pvspace{8.0pt}%
\@x{}\midbar\@xx{}%
\@pvspace{8.0pt}%
\@x{ oneOf ( S )\@s{2.59} \.{\defeq} {\CHOOSE} x \.{\in} S \.{:} {\TRUE}}%
 \@x{ maxOf ( S ) \.{\defeq} {\CHOOSE} x \.{\in} S \.{:} \A\, y \.{\in} S
 \.{:} x \.{\geq} y}%
 \@x{ minOf ( S )\@s{1.06} \.{\defeq} {\CHOOSE} x \.{\in} S \.{:} \A\, y
 \.{\in} S \.{:} x \.{\leq} y}%
\@pvspace{8.0pt}%
 \@x{ maxCh ( chs ) \.{\defeq} {\CHOOSE} ch \.{\in} chs \.{:} ( \A\, ds
 \.{\in} chs \.{:} ch . start\_ts \.{\geq} ds . start\_ts )}%
\@x{ addPremium ( n ,\, chB ) \.{\defeq} n \.{+} chB . premium}%
\@pvspace{8.0pt}%
 \@x{ isNotZeroWidth ( ch )\@s{5.83} \.{\defeq} ch . start\_ts \.{<} ch .
 end\_ts}%
\@pvspace{8.0pt}%
\@x{ {\RECURSIVE} fold ( \_ ,\, \_ ,\, \_ )}%
 \@x{ fold ( Op ( \_ ,\, \_ ) ,\, init ,\, Txs ) \.{\defeq} {\IF} Txs \.{=} \{
 \}}%
\@x{\@s{16.87} \.{\THEN} init}%
\@x{\@s{16.87} \.{\ELSE} \.{\LET} tx \.{\defeq} oneOf ( Txs )}%
 \@x{\@s{48.18} \.{\IN} fold ( Op ,\, Op ( init ,\, tx ) ,\, Txs
 \.{\,\backslash\,} \{ tx \} )}%
\@pvspace{8.0pt}%
\@x{ sumPremium ( chs ) \.{\defeq} fold ( addPremium ,\, 0 ,\, chs )}%
\@pvspace{8.0pt}%
 \@x{ filter ( fn ( \_ ) ,\, xs ) \.{\defeq} \{ x \.{\in} xs \.{:} fn ( x )
 \}}%
\@pvspace{8.0pt}%
\@x{}\midbar\@xx{}%
\@pvspace{8.0pt}%
 \@x{ InvTypeQ \.{\defeq} \A\, n \.{\in} 1 \.{\dotdot} Len ( cancelQ ) \.{:}
 cancelQ [ n ] \.{\in} Modification}%
 \@x{ InvTypeCh \.{\defeq} \A\, ch \.{\in} policyChs \.{:} ch \.{\in}
 Characteristic}%
 \@x{ InvTypeHoldback \.{\defeq} \A\, hb \.{\in} holdbacks \.{:} hb \.{\in}
 Holdback}%
\@pvspace{8.0pt}%
\@x{}%
\@y{\@s{0}%
 When the policy is fully canceled there is always 1 policy characteristic
}%
\@xx{}%
\@x{ InvPolicy \.{\defeq}}%
\@x{\@s{16.4} \.{\land} policy \.{\in} Interval}%
\@x{\@s{16.4} \.{\land} policy . start\_ts \.{\leq} policy . end\_ts}%
 \@x{\@s{16.4} \.{\land} ( policy . start\_ts \.{=} policy . end\_ts \.{\land}
 pc \.{>} 0 ) \.{\implies} Cardinality ( policyChs ) \.{=} 1}%
 \@x{\@s{16.4} \.{\land} ( policy . start\_ts \.{=} policy . end\_ts \.{\land}
 pc \.{>} 0 ) \.{\implies} (}%
 \@x{\@s{47.79} \.{\LET} ch \.{\defeq} {\CHOOSE} ch \.{\in} policyChs \.{:}
 {\TRUE}}%
 \@x{\@s{47.79} \.{\IN} ch . start\_ts \.{=} ch . end\_ts \.{\land} ch .
 start\_ts \.{=} policy . start\_ts}%
\@x{\@s{47.79} )}%
\@pvspace{8.0pt}%
\begin{lcom}{7.5}%
\begin{cpar}{0}{F}{F}{0}{0}{}%
 The start and ends of the policy exactly bracket the range of the
 characteristics.
 And none of the characteristics, which are valid for the policy, may overlap.
\end{cpar}%
\end{lcom}%
\@x{ InvPolicyChs \.{\defeq}}%
 \@x{\@s{16.4} \.{\land} pc \.{>} 0 \.{\implies} Cardinality ( policyChs )
 \.{>} 0}%
 \@x{\@s{16.4} \.{\land} pc \.{>} 0 \.{\implies} policy . start\_ts \.{=}
 minOf ( \{ ch . start\_ts \.{:} ch \.{\in} policyChs \} )}%
 \@x{\@s{16.4} \.{\land} pc \.{>} 0 \.{\implies} policy . end\_ts \.{=} maxOf
 ( \{ ch . end\_ts \.{:} ch \.{\in} policyChs \} )}%
 \@x{\@s{16.4} \.{\land} ( \exists\, ch \.{\in} policyChs \.{:} ch . start\_ts
 \.{=} ch . end\_ts ) \.{\implies}}%
\@x{\@s{43.69} policy . start\_ts \.{=} policy . end\_ts}%
 \@x{\@s{16.4} \.{\land} \A\, ch1 ,\, ch2 \.{\in} policyChs \.{:} ( ch1
 \.{\neq} ch2 ) \.{\implies} notOverlaps ( ch1 ,\, ch2 )}%
\@pvspace{8.0pt}%
\begin{lcom}{7.5}%
\begin{cpar}{0}{F}{F}{0}{0}{}%
Holdbacks are always bounded by the interval of the policy. When there are no
 outstanding cancellations, then there can be no active
 \ensuremath{holdbacks}. This last property
 comes about as only cancellations create \ensuremath{holdbacks} and
 reinstatements always reverse
 the \ensuremath{holdbacks} created by their matched cancellation.
\end{cpar}%
\end{lcom}%
\@x{ InvHoldbacks \.{\defeq}}%
\@x{\@s{16.4} \.{\land} Cardinality ( holdbacks ) \.{=} Len ( cancelQ )}%
 \@x{\@s{16.4} \.{\land} \A\, hb \.{\in} holdbacks \.{:} \.{\land} hb .
 start\_ts \.{=} policy . start\_ts}%
\@x{\@s{107.94} \.{\land} hb . end\_ts \.{\leq} policy . end\_ts}%
\@pvspace{8.0pt}%
\begin{lcom}{7.5}%
\begin{cpar}{0}{F}{F}{0}{0}{}%
 When the policy start date equals the policy end date, that is a sign that
 there are
 one or more unreversed cancellations in the policies modification set. In
 most cases
 we expect that the most recent, unreversed cancellation will have a start
 date equal
 to the policies end data, but this is not always the case. Since policies
 can have
 gaps in coverage, the actual policy end date can sometimes be less than the
 most
 recent cancellations start date. Then if there are unrevered cancellations
 we may have
 valid \ensuremath{holdbacks} and when all cancellations have been reversed
 there should be no
 valid \ensuremath{holdbacks}.
\end{cpar}%
\end{lcom}%
\@x{ InvCancelQ \.{\defeq}}%
 \@x{\@s{16.4} \.{\land} ( pc \.{\neq} 0 \.{\land} policy . start\_ts \.{=}
 policy . end\_ts ) \.{\implies} Len ( cancelQ ) \.{>} 0}%
 \@x{\@s{16.4} \.{\land} Len ( cancelQ ) \.{>} 0 \.{\implies} \.{\LET} last
 \.{\defeq} cancelQ [ Len ( cancelQ ) ]}%
\@x{\@s{120.60} \.{\IN} last . start\_ts \.{\geq} policy . end\_ts}%
 \@x{\@s{16.4} \.{\land} \A\, n \.{\in} 1 \.{\dotdot} Len ( cancelQ ) \.{:} (
 \exists\, hb \.{\in} holdbacks \.{:} hb . pc \.{=} cancelQ [ n ] . pc )}%
\@pvspace{8.0pt}%
\@x{}\midbar\@xx{}%
\@x{}%
\@y{\@s{0}%
 make a default, priced characteristic
}%
\@xx{}%
\@x{ mkDfltCh ( from ,\, to ,\, progCtr ) \.{\defeq} [}%
\@x{\@s{42.56} start\_ts \.{\mapsto} from ,\,}%
\@x{\@s{42.56} end\_ts \.{\mapsto} to ,\,}%
\@x{\@s{42.56} premium \.{\mapsto} to \.{-} from ,\,}%
\@x{\@s{42.56} monthPremium \.{\mapsto} 1 ,\,}%
\@x{\@s{42.56} pc \.{\mapsto} progCtr}%
\@x{\@s{16.4} ]}%
\@pvspace{8.0pt}%
\@x{ prorate ( chs ,\, progCtr ) \.{\defeq} \{}%
 \@x{\@s{16.4} [ ch {\EXCEPT} {\bang} . premium \.{=} ch . monthPremium \.{*}
 ( ch . end\_ts \.{-} ch . start\_ts ) ,\,}%
\@x{\@s{71.67} {\bang} . pc \.{=} progCtr \.{+} 1 ] \.{:} ch \.{\in} chs \}}%
\@pvspace{8.0pt}%
 \@x{ reprice ( chs ,\, progCtr ) \.{\defeq} \{ [ ch {\EXCEPT} {\bang} .
 premium \.{=} ch . end\_ts \.{-} ch . start\_ts ,\,}%
\@x{\@s{171.60} {\bang} . pc \.{=} progCtr \.{+} 1 ] \.{:} ch \.{\in} chs \}}%
\@pvspace{8.0pt}%
\@x{ reducedChsFrwork ( chs ,\, int ,\, progCtr ) \.{\defeq}}%
\@x{\@s{16.4} \.{\LET} remainSet \.{\defeq} subtractAll ( chs ,\, int )}%
 \@x{\@s{36.79} prorateSet \.{\defeq} \{ r \.{\in} remainSet \.{:} r
 \.{\notin} chs \}}%
\@x{\@s{16.4} \.{\IN} {\IF} remainSet \.{=} \{ \}}%
\@x{\@s{36.79} \.{\THEN} \{ [}%
\@x{\@s{85.41} start\_ts \.{\mapsto} int . start\_ts ,\,}%
\@x{\@s{85.41} end\_ts \.{\mapsto} int . start\_ts ,\,}%
\@x{\@s{85.41} premium \.{\mapsto} 0 ,\,}%
\@x{\@s{85.41} pc \.{\mapsto} progCtr \.{+} 1}%
\@x{\@s{68.11} ] \}}%
\@x{\@s{36.79} \.{\ELSE} {\UNION} \{}%
\@x{\@s{84.51} ( remainSet \.{\,\backslash\,} prorateSet ) ,\,}%
\@x{\@s{84.51} prorate ( prorateSet ,\, progCtr )}%
\@x{\@s{68.11} \}}%
\@pvspace{8.0pt}%
\begin{lcom}{7.5}%
\begin{cpar}{0}{F}{F}{0}{0}{}%
 In our abstract model the proration has the effect of adding a retained
 amount to the
 last characteristic in the reduced set. Since we generally want to add a
 retained amount,
 when we split on a characteristic boundary, we have to add the retained
 amount to
 a characteristic ouside of the modification interval.
 In general, this function would potentially change premium, tax, or fee
 amounts.
\end{cpar}%
\end{lcom}%
\@x{ reducedChsPlugin ( chs ,\, int ,\, progCtr ) \.{\defeq}}%
\@x{\@s{16.4} \.{\LET} remainSet \.{\defeq} subtractAll ( chs ,\, int )}%
 \@x{\@s{36.79} prorateSet \.{\defeq} \{ r \.{\in} remainSet \.{:} r
 \.{\notin} chs \}}%
\@x{\@s{36.79} mx \.{\defeq} {\IF} prorateSet \.{=} \{ \}}%
\@x{\@s{69.70} \.{\THEN} maxCh ( remainSet )}%
\@x{\@s{69.70} \.{\ELSE} maxCh ( prorateSet )}%
\@x{\@s{16.4} \.{\IN} {\IF} remainSet \.{=} \{ \}}%
\@x{\@s{36.79} \.{\THEN} \{ [}%
\@x{\@s{85.41} start\_ts \.{\mapsto} int . start\_ts ,\,}%
\@x{\@s{85.41} end\_ts \.{\mapsto} int . start\_ts ,\,}%
\@x{\@s{85.41} premium \.{\mapsto} 0 ,\,}%
\@x{\@s{85.41} pc \.{\mapsto} progCtr \.{+} 1}%
\@x{\@s{68.11} ] \}}%
\@x{\@s{36.79} \.{\ELSE} {\UNION} \{}%
\@x{\@s{80.41} ( remainSet \.{\,\backslash\,} \{ mx \} ) ,\,}%
\@x{\@s{80.41} prorate ( \{ mx \} ,\, progCtr )}%
\@x{\@s{68.11} \}}%
\@pvspace{8.0pt}%
\@x{ reducedChs ( chs ,\, int ,\, progCtr ) \.{\defeq}}%
\@x{\@s{16.4} {\IF} pluginActive \.{=} 0}%
\@x{\@s{16.4} \.{\THEN} reducedChsFrwork ( chs ,\, int ,\, progCtr )}%
\@x{\@s{16.4} \.{\ELSE} reducedChsPlugin ( chs ,\, int ,\, progCtr )}%
\@pvspace{8.0pt}%
\@x{}\midbar\@xx{}%
\begin{lcom}{7.5}%
\begin{cpar}{0}{F}{F}{0}{0}{}%
 Below \ensuremath{ts} is the start date of the endorsement and
 \ensuremath{te} is the resulting end date of the
 endorsed policy. Although \ensuremath{te} can be greater than the end of the
 policy, in this spec
 we only model the case where \ensuremath{te \.{\leq}} the policy end dt.
 Note, that endorsments can
 not change the policy\mbox{'}s holdback collection.
\end{cpar}%
\end{lcom}%
\@x{ endorsePolicy ( ts ,\, te ) \.{\defeq}}%
\@x{\@s{16.4} \.{\land} pc \.{<} maxPc}%
\@x{\@s{16.4} \.{\land} policy . start\_ts \.{<} policy . end\_ts}%
 \@x{\@s{16.4} \.{\land} ts \.{\geq} policy . start\_ts \.{\land} ts \.{<} te
 \.{\land} te \.{\leq} policy . end\_ts}%
\@x{\@s{16.4} \.{\land} Len ( cancelQ ) \.{=} 0}%
 \@x{\@s{16.4} \.{\land} \.{\LET} endorseInt \.{\defeq} [ start\_ts
 \.{\mapsto} ts ,\, end\_ts \.{\mapsto} te ]}%
 \@x{\@s{47.91} repriceSet\@s{3.21} \.{\defeq} reprice ( intersectAll (
 policyChs ,\, endorseInt ) ,\, pc )}%
 \@x{\@s{47.91} splitSet \.{\defeq} \{ ch \.{\in} policyChs \.{:} within ( ts
 ,\, ch ) \}}%
\@x{\@s{47.91} prorateSet \.{\defeq} prorate ( \{}%
 \@x{\@s{68.41} dh \.{\in} subtractAll ( splitSet ,\, endorseInt ) \.{:} dh .
 start\_ts \.{<} ts \} ,\, pc )}%
 \@x{\@s{47.91} unchangedSet \.{\defeq} \{ ch \.{\in} policyChs \.{:} ch .
 end\_ts \.{\leq} ts \}}%
 \@x{\@s{47.91} nextPolicyChs \.{\defeq} {\UNION} \{ repriceSet ,\, prorateSet
 ,\, unchangedSet \}}%
 \@x{\@s{47.91} nextMaxT \.{\defeq} maxOf ( \{ ch . end\_ts \.{:} ch \.{\in}
 nextPolicyChs \} )}%
 \@x{\@s{27.51} \.{\IN} \.{\land} Assert ( nextMaxT \.{\leq} te
 ,\,\@w{unexpected\ max} )}%
\@x{\@s{47.91} \.{\land} Assert (}%
 \@x{\@s{71.32} policy . end\_ts \.{=} te \.{\implies} sumPremium ( policyChs
 ) \.{=} sumPremium ( nextPolicyChs ) ,\,}%
\@x{\@s{71.32}\@w{unmatched\ premiums} )}%
 \@x{\@s{47.91} \.{\land} policy \.{'} \.{=} [ policy {\EXCEPT} {\bang} .
 end\_ts \.{=} nextMaxT ]}%
\@x{\@s{47.91} \.{\land} policyChs \.{'} \.{=} nextPolicyChs}%
\@x{\@s{47.91} \.{\land} pc \.{'} \.{=} pc \.{+} 1}%
 \@x{\@s{47.91} \.{\land} {\UNCHANGED} {\langle} cancelQ ,\, holdbacks
 {\rangle}}%
\@pvspace{8.0pt}%
\begin{lcom}{7.5}%
\begin{cpar}{0}{F}{F}{0}{0}{}%
 An abstract cancellation. When cancellations occur new \ensuremath{holdbacks}
 will be added to the
 policy and, if there are any existing \ensuremath{holdbacks}, those that
 will fall outside of the
 cancelled policy interval are recreated so that their interval remains
 within the policy
 interval. Since policies may have gaps in coverage the new policy end date
 will be less
 than or equal to the cancellation date, and this end date is determined from
 the set of
 valid policy characteristics.
\end{cpar}%
\end{lcom}%
\@x{ cancelPolicy ( t ) \.{\defeq}}%
\@x{\@s{16.4} \.{\land} pc \.{<} maxPc}%
\@x{\@s{16.4} \.{\land} policy . end\_ts \.{>} policy . start\_ts}%
\@x{\@s{16.4} \.{\land} t \.{\geq} policy . start\_ts}%
\@x{\@s{16.4} \.{\land} t \.{<} policy . end\_ts}%
 \@x{\@s{16.4} \.{\land} \.{\LET} cancelInt \.{\defeq} [ start\_ts \.{\mapsto}
 t ,\, end\_ts \.{\mapsto} policy . end\_ts ]}%
 \@x{\@s{47.91} nextPolicyChs \.{\defeq} reducedChs ( policyChs ,\, cancelInt
 ,\, pc )}%
 \@x{\@s{47.91} delHoldbacks \.{\defeq} overlapsAll ( holdbacks ,\, cancelInt
 )}%
 \@x{\@s{47.91} nextMaxT \.{\defeq} maxOf ( \{ ch . end\_ts \.{:} ch \.{\in}
 nextPolicyChs \} )}%
 \@x{\@s{27.51} \.{\IN} \.{\land} Assert ( nextMaxT \.{\leq} t ,\, {\langle}
 nextMaxT ,\, t {\rangle} )}%
 \@x{\@s{47.91} \.{\land} policy \.{'} \.{=} [ policy {\EXCEPT} {\bang} .
 end\_ts \.{=} nextMaxT ]}%
\@x{\@s{47.91} \.{\land} policyChs \.{'} \.{=} nextPolicyChs}%
\@x{\@s{47.91} \.{\land} cancelQ \.{'} \.{=} Append (}%
\@x{\@s{71.32} cancelQ ,\,}%
 \@x{\@s{71.32} [ start\_ts \.{\mapsto} t ,\, end\_ts \.{\mapsto} policy .
 end\_ts ,\, pc \.{\mapsto} pc \.{+} 1 ] )}%
\@x{\@s{47.91} \.{\land} holdbacks \.{'} \.{=} {\UNION} \{}%
\@x{\@s{71.32} ( holdbacks \.{\,\backslash\,} delHoldbacks ) ,\,}%
\@x{\@s{71.32} \{ [}%
\@x{\@s{88.62} start\_ts \.{\mapsto} policy . start\_ts ,\,}%
\@x{\@s{88.62} end\_ts\@s{4.56} \.{\mapsto} nextMaxT ,\,}%
\@x{\@s{88.62} amount \.{\mapsto} policy . end\_ts \.{-} t ,\,}%
\@x{\@s{88.62} pc \.{\mapsto} pc \.{+} 1}%
\@x{\@s{71.32} ] \} ,\,}%
 \@x{\@s{71.32} \{ [ dh {\EXCEPT} {\bang} . start\_ts \.{=} policy . start\_ts
 ,\,}%
 \@x{\@s{132.11} {\bang} . end\_ts \.{=} nextMaxT ] \.{:} dh \.{\in}
 delHoldbacks \} \}}%
\@x{\@s{47.91} \.{\land} pc \.{'} \.{=} pc \.{+} 1}%
\@pvspace{8.0pt}%
\begin{lcom}{7.5}%
\begin{cpar}{0}{F}{F}{0}{0}{}%
 An abstract creation or renewal is extended by adding a characteristic.
 Creations and
 renewals never create new \ensuremath{holdbacks}.
\end{cpar}%
\end{lcom}%
\@x{ extendPolicy ( t ) \.{\defeq}}%
\@x{\@s{16.4} \.{\land} pc \.{<} maxPc}%
\@x{\@s{16.4} \.{\land} t \.{>} policy . end\_ts}%
\@x{\@s{16.4} \.{\land} Len ( cancelQ ) \.{=} 0}%
 \@x{\@s{16.4} \.{\land} \.{\LET} addCh \.{\defeq} mkDfltCh ( policy . end\_ts
 ,\, t ,\, pc \.{+} 1 )}%
 \@x{\@s{27.51} \.{\IN} \.{\land} policy \.{'} \.{=} [ policy {\EXCEPT}
 {\bang} . end\_ts \.{=} t ]}%
 \@x{\@s{47.91} \.{\land} policyChs \.{'} \.{=} policyChs \.{\cup} \{ addCh
 \}}%
\@x{\@s{47.91} \.{\land} pc \.{'} \.{=} pc \.{+} 1}%
 \@x{\@s{47.91} \.{\land} {\UNCHANGED} {\langle} cancelQ ,\, holdbacks
 {\rangle}}%
\@pvspace{8.0pt}%
\begin{lcom}{7.5}%
\begin{cpar}{0}{F}{F}{0}{0}{}%
 An abstract reinstatment. When we reinstate we check if there are any
 \ensuremath{holdbacks
 } associated with the reinstatements cancellation. If there are then we can
 invalidate
 those previous \ensuremath{holdbacks}, as a reinstatement must reverse the
 effects of its
 cancellation. When a reinstatement changes the start date of a policy, we
 may also
 have to rewrite valid holdback to maintain holdback intervals within the
 policy
 interval.
\end{cpar}%
\end{lcom}%
\@x{ reinstatePolicy ( t ) \.{\defeq}}%
\@x{\@s{16.4} \.{\land} pc \.{<} maxPc}%
\@x{\@s{16.4} \.{\land} Len ( cancelQ ) \.{\neq} 0}%
\@x{\@s{16.4} \.{\land} \.{\LET} last \.{\defeq} cancelQ [ Len ( cancelQ ) ]}%
 \@x{\@s{47.91} addCh \.{\defeq} mkDfltCh ( t ,\, last . end\_ts ,\, pc \.{+}
 1 )}%
 \@x{\@s{47.91} delHb\@s{3.43} \.{\defeq} \{ hb \.{\in} holdbacks \.{:} hb .
 pc \.{=} last . pc \}}%
 \@x{\@s{47.91} policyStart \.{\defeq} {\IF} policy . start\_ts \.{=} policy .
 end\_ts \.{\THEN} t \.{\ELSE} policy . start\_ts}%
 \@x{\@s{27.51} \.{\IN} \.{\land} t \.{\geq} last . start\_ts \.{\land} t
 \.{<} last . end\_ts}%
 \@x{\@s{47.91} \.{\land} policy \.{'} \.{=} [ policy {\EXCEPT} {\bang} .
 start\_ts \.{=} policyStart ,\, {\bang} . end\_ts \.{=} last . end\_ts ]}%
 \@x{\@s{47.91} \.{\land} policyChs \.{'} \.{=} filter ( isNotZeroWidth ,\, (
 {\UNION} \{ policyChs ,\, \{ addCh \} \} ) )}%
\@x{\@s{47.91} \.{\land} holdbacks \.{'}\@s{0.76} \.{=} \{}%
 \@x{\@s{71.32} [ hb {\EXCEPT} {\bang} . start\_ts \.{=} max ( hb . start\_ts
 ,\, policyStart ) ,\,}%
 \@x{\@s{126.46} {\bang} . end\_ts \.{=} max ( hb . end\_ts ,\, policyStart )
 ] \.{:} hb \.{\in} ( holdbacks \.{\,\backslash\,} delHb )}%
\@x{\@s{71.32} \}}%
 \@x{\@s{47.91} \.{\land} cancelQ \.{'} \.{=} [ n \.{\in} 1 \.{\dotdot} ( Len
 ( cancelQ ) \.{-} 1 ) \.{\mapsto} cancelQ [ n ] ]}%
\@x{\@s{47.91} \.{\land} pc \.{'} \.{=} pc \.{+} 1}%
\@pvspace{8.0pt}%
\@x{ term \.{\defeq}}%
\@x{\@s{16.4} \.{\land} pc \.{\geq} maxPc}%
 \@x{\@s{16.4} \.{\land} {\UNCHANGED} {\langle} policy ,\, policyChs ,\,
 holdbacks ,\, pc ,\, cancelQ {\rangle}}%
\@pvspace{8.0pt}%
\@x{ Init \.{\defeq}}%
 \@x{\@s{16.4} \.{\land} policy \.{=} [ start\_ts \.{\mapsto} 0 ,\, end\_ts
 \.{\mapsto} 0 ]}%
\@x{\@s{16.4} \.{\land} policyChs \.{=} \{ \}}%
\@x{\@s{16.4} \.{\land} holdbacks\@s{0.76} \.{=} \{ \}}%
\@x{\@s{16.4} \.{\land} cancelQ \.{=} {\langle} {\rangle}}%
\@x{\@s{16.4} \.{\land} pc \.{=} 0}%
\@pvspace{8.0pt}%
\@x{ Next \.{\defeq}}%
\@x{\@s{16.4} \.{\lor} \exists\, t \.{\in} 0 \.{\dotdot} maxT \.{:}}%
\@x{\@s{48.01} \.{\lor} extendPolicy ( t )}%
\@x{\@s{48.01} \.{\lor} cancelPolicy ( t )}%
\@x{\@s{48.01} \.{\lor} reinstatePolicy ( t )}%
 \@x{\@s{16.4} \.{\lor} \exists\, ts ,\, te \.{\in} 0 \.{\dotdot} maxT \.{:}
 endorsePolicy ( ts ,\, te )}%
\@x{\@s{16.4} \.{\lor} term}%
\@pvspace{8.0pt}%
 \@x{ Spec \.{\defeq} Init \.{\land} {\Box} [ Next ]_{ {\langle} policy ,\,
 policyChs ,\, holdbacks ,\, pc ,\, cancelQ {\rangle}}}%
\@pvspace{8.0pt}%
\@x{}\bottombar\@xx{}%
\setboolean{shading}{false}
\begin{lcom}{0}%
\begin{cpar}{0}{F}{F}{0}{0}{}%
\ensuremath{\.{\,\backslash\,}\.{*} Modification} History
\end{cpar}%
\begin{cpar}{0}{F}{F}{0}{0}{}%
 \ensuremath{\.{\,\backslash\,}\.{*}} Last modified \ensuremath{Fri}
 \ensuremath{Apr} 02 12:06:16 \ensuremath{PDT} 2021 by \ensuremath{ASUS
}%
\end{cpar}%
\begin{cpar}{0}{F}{F}{0}{0}{}%
 \ensuremath{\.{\,\backslash\,}\.{*}} Last modified \ensuremath{Mon}
 \ensuremath{Feb} 15 11:44:51 \ensuremath{PST} 2021 by \ensuremath{marcderosa
}%
\end{cpar}%
\begin{cpar}{0}{F}{F}{0}{0}{}%
 \ensuremath{\.{\,\backslash\,}\.{*}} Created \ensuremath{Mon}
 \ensuremath{Feb} 15 10:49:49 \ensuremath{PST} 2021 by \ensuremath{marcderosa
}%
\end{cpar}%
\end{lcom}%
