
\begin{figure}[h]
\begin{tikzpicture}[->,>=stealth']

  \node[state, text width=2cm] (Product) {
    \begin{tabular}{l}
      \textbf{Product}
    \end{tabular}
  };

  \node[state,    	% layout (defined above)
    text width=4cm, 	% max text width
    right of=Product, 	% Position is to the right of Product
    node distance=6.5cm,  % distance to Product
    anchor=center       % posistion relative to the center of the 'box'
  ] (ProductConfig) {
    \begin{tabular}{l}
      \textbf{ProductConfig}
    \end{tabular}
  };

  \node[state,    	% layout (defined above)
    text width=4cm, 	% max text width
    xshift=1cm,
    below of=ProductConfig, 	% Position is to the right of Product
    node distance=1.4cm,
    anchor=center       % posistion relative to the center of the 'box'
  ] (ProductExposureConfig) {
    \begin{tabular}{l}
      \textbf{ProductExposureConfig}
    \end{tabular}
  };

  \node[state,    	% layout (defined above)
    text width=4cm, 	% max text width
    xshift=1cm,
    below of=ProductExposureConfig, 	% Position is to the right of Product
    node distance=1.4cm,
    anchor=center       % posistion relative to the center of the 'box'
  ] (ProductPerilConfig) {
    \begin{tabular}{l}
      \textbf{ProductPerilConfig}
    \end{tabular}
  };

  \node[state,    	% layout (defined above)
    text width=2cm,
    below of=Product,
    node distance=4.5cm,  % distance to Product
    anchor=center
  ] (Policy) {
    \begin{tabular}{l}
      \textbf{Policy}
    \end{tabular}
  };

  \node[state,
    text width=4cm,
    right of=Policy,
    node distance=5.5cm,
    anchor=center
  ] (PolicyCharacteristics) {
    \begin{tabular}{l}
      \textbf{PolicyCharacteristics}
    \end{tabular}
  };

  \node[state,
    text width=5cm,
    xshift=1cm,
    below of=PolicyCharacteristics,
    node distance=1.5cm,
    anchor=center
  ] (PolicyExposureCharacteristics) {
    \begin{tabular}{l}
      \textbf{PolicyExposureCharacteristics}
    \end{tabular}
  };

  \node[state,
    text width=4cm,
    xshift=1cm,
    below of=PolicyExposureCharacteristics,
    node distance=1.5cm,
    anchor=center
  ] (PolicyPerilCharacteristics) {
    \begin{tabular}{l}
      \textbf{PolicyPerilCharacteristics}
    \end{tabular}
  };

  \node[state,
    xshift=0.5cm,
    below of=Policy,
    node distance=4cm,
    anchor=center
  ] (Modifications) {
    \begin{tabular}{l}
      \textbf{Modifications}\\
      \parbox{3cm}{
        \begin{IEEEeqnarray*}{Cl}
          =& Creation \\
          |& Endorsement \\
          |& Renewal \\
          |& Cancellation \\
          |& Reinstatement
        \end{IEEEeqnarray*}
      }\\[4em]
    \end{tabular}
  };

  \path (ProductConfig) edge node[anchor=south,above]{$defines\;attributes\;of$} (Product)
  (ProductExposureConfig) edge[bend left=20] node[anchor=left,right]{$subgroup$} (ProductConfig)
  (ProductPerilConfig) edge[bend left=20] node[anchor=left,right]{$subgroup$} (ProductExposureConfig)
  (Policy)             edge[bend right=20] node[anchor=north,below]{$current$} (PolicyCharacteristics)
  (Policy) edge node[anchor=left,right]{$has\;associated$} (Product)
  (Policy) edge node[anchor=left,right]{$versioned\;to$} (ProductConfig)
  (PolicyCharacteristics) edge[bend right=20] node[anchor=south,above]{$stores\;attributes\;of$} (Policy)
  (PolicyExposureCharacteristics) edge[bend left=20] node[anchor=left,right]{$subgroup$} (PolicyCharacteristics)
  (PolicyPerilCharacteristics) edge[bend left=20] node[anchor=left,right]{$subgroup$} (PolicyExposureCharacteristics)
  (Modifications) edge node[anchor=left,right]{$audit\;history\;of$} (Policy);

\end{tikzpicture}
\caption{A simple, overview of the Socotra data schema}
\label{ch03:fig:1}
\end{figure}

In the previous chapter we described a contract as the combination of a product name, an interval, and a set of attributes.
In the actual Socotra implementation, contracts are called policies. Each policy has an associated interval and a set of attributes,
which are called characteristics. The idea of a product name is expanded into the concept of a product. This product is
best thought of as metadata describing the policy. You can think of a product as a parallel to the familiar idea of metadata in the
database world. In the database world metadata is a set of relation and field definitions. Records in the database are then instances
of those relations and field definitions. In the Socotra object model the product is the definition of possible attributes (characteristics)
and product behaviors. A policy is then, like a database record, a specific data instance that complies with the definitions set out in
the product (metadata).

In the framework, policies are contracts that are instances of what I have called a $CompoundContract$ in the previous chapter. In
particular Socotra has decided to organize policy data in the form of a three level tree. The root of the tree is called the
policy. The nodes at second level of the tree are called exposures and the nodes at the third level of the tree are called perils.
Attributes become more specialized as one moves down the tree towards the perils. The span of an interval at any node in the tree
is always contained within the interval of its parent. And the interval of the policy, at the root, defines the total time span of
the contract itself. \footnote {Trees are useful data structures to assist a computer in sorting and searching. As technologists
  we are proud to have trained ourselves to be comfortable thinking in and manipulating trees. But must users are not computers
  nor have they been trained to think in tree structures. In the real world, it is a classic design failure to expose users to trees
  and I have seen that design choice lose to simpler designs in the market over and over again. Trees are always the wrong data structure to expose
  to non technical users. There are no exceptions to this rule. Store that knowledge in your future design toolbox.}

Lastly, in addition to policy and product data, the framework organizes it's data to track change history. You remember from the previous
chapter that the operations $extend$, $reduce$, and $override$ can be applied to contracts. The applications of these functions are
tracked in the frameworks policy\_modification table. In the policy\_modification table the first change to a policy is tracked
as a create modification and further modifications are tracked in order from there. This whole organization of data with labeled
relationships on the data is shown in stylized form in image \ref{ch03:fig:1}.

