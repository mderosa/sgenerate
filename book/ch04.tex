\motto{To learn to make art you spend time working with the materials. That is it. When the urge arises to judge the quality of 
  your work, it will be tempting to say this is good or this is bad. Take that energy that you might put into assessing 
  the aesthetic and put it back into the materials instead. Persist and grow. Avoid aesthetic judgements, and instead
  produce work.}
\chapter{The Systems}
\label{intro04} % Always give a unique label
% use \chaptermark{}
% to alter or adjust the chapter heading in the running head

\abstract{
  Socotra has a very simple system setup. There are two user interfaces and a few backend
  systems, the most important of which is the API. The biggest problem with understanding
  the system setup is that the names of the components don't always imply what the component
  does. The description, below, will match up components with a function and where possible
  indicate some reasonable naming.
}

\section{Structure of the Overall System}
\label{sec:03:1}

Figure 3.1 shows the overall system. The Config Manager is a policy designer. From within Config
Manager product experts can define insurance products as a combination of json configuration
files, tables for value lookup, and script files for dynamic policy calculations. Once a product
is defined and deployed, then the Policy Editor is used by company representative to create
policy holders and to issue and modify policies based on a product.

Load Assets is part of the backend. It accepts zipped product, configuration files from Config Manger and
it orchestrates the deployment of those product, configuration files using internal APIs that the
Socotra API exposes. The Socotra API is then a very standard API application. It stores data in
a few external databases. And it relies on a javascript executor called Socol to do dynamic calculations,
for instance fees or taxes.

\import{ch04/}{SystemOrganization.tex}

\section{Structure of Socotra API}
The Socotra API application is a simple layered application built on the Spring Framework. The
layering is somewhat unconventional; however, which makes the structure not immediately evident.
Figure 3.2 shows the current API structure with a generic application structure for comparison. The
best way to think of the Socotra API is as having a very thin controller layer at the front and a
very thin layer at the back of request and response objects, that are used everywhere. Then
there is a very thick middle layer which contains everything else. The application could certainly use
more circumspect layering to organize code and developer thinking. It's likely that the application
will evolve over time to look more and more like the generic application structure on the right.
In fact, its unlikely that anything more nuanced could be needed.

\import{ch04/}{ApiOrganization.tex}


