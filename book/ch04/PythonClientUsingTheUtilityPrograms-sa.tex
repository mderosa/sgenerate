\hypertarget{using-the-socotratenant-program}{%
\subsection{Using the socotratenant
program}\label{using-the-socotratenant-program}}

This program allows one to submit a zipfile containing assets to a
running version of load-assets. It creates or updates a tenant An
example of usage is:

\begin{verbatim}
export PYTHONPATH=$PYTHONPATH:./:../stack-monitoring
python bin/socotratenant -z ~/Code/Configs/defaultConfig.zip -u docker-dev -p socotra -a http://localhost:8080 -l http://localhost:5000 -n docker-dev-configeditor --test
\end{verbatim}

** -z ** specifies the zip file to upload through load assets ** -u **
specifies the user name which has api access permissions ** -a ** is the
endpoint of the Java api ** -l ** is the endpoint of the load assets api
** -\/-test ** flag determines which load asset api is used to upload
the zip file. If -\/-test is specified then the assets are uploaded to
the endpoint /configuration/deployTest. If -\/-test is not specified
then the assets are uploaded to the endpoint /assets/v1/deploy. In both
cases the tenant type created in the tenant table is 'tenent.test' ** -n
** is a name related to the hostname / tenant combination that is being
created or updated. How the name maps to data in the database differs
depending on the -\/-test flag and is as follows:

\begin{enumerate}
\def\labelenumi{\arabic{enumi}.}
\tightlist
\item
  if -\/-test is specified, affected records are those linked to

  \begin{itemize}
  \tightlist
  \item
    tenant-hostname.hostname = {[}user name{]}-{[}name{]}.co.socotra.com
    and tenant.tenant\_name = {[}user name{]}-{[}name{]}
  \item
    As an example, the example command, above, creates or updates data
    associated with tenant-hostname.hostname =
    docker-dev-docker-dev-configeditor.co.socotra.com
    tenant.tenant\_name = docker-dev-docker-dev-configeditor
  \end{itemize}
\item
  if -\/-test is not specified, affected records are those linked to

  \begin{itemize}
  \tightlist
  \item
    tenant-hostname.hostname = {[}name{]}.co.socotra.com and
    tenant.tentant\_name = {[}name{]}
  \item
    As an example, the example command, above, without -\/-test, would
    create or update data associated with tenant-hostname.hostname =
    docker-dev-configeditor.co.socotra.com tenant.tenant\_name =
    docker-dev-configeditor
  \end{itemize}
\end{enumerate}

\hypertarget{using-the-socortraadmin-program}{%
\subsection{Using the socortraadmin
program}\label{using-the-socortraadmin-program}}

This program allows one to update and read information from an
installation of the socortra platform. The program takes a combination
of positional and named arguments the combination of which defines the
data to read or update. Some example usages in a local development
environment are listed below

\hypertarget{creating-a-bootstrap-account}{%
\subsubsection{Creating a bootstrap
account}\label{creating-a-bootstrap-account}}

The command usage is:

\begin{verbatim}
socotraadmin environment bootstrap_admin [--name NAME] [--username USERNAME]
                           [--password PASSWORD]
                           [--account_type [account.test.tenant.admin | account.internal]]
                           [--api_url] [--jwtsecret]
                           [--admin_username] [--admin_password]
\end{verbatim}

This command is usually used to create a special account of type
account.internal. These accounts are not associated with a tenant (the
tenant is '\_'), but once created the account can be used to log onto an
instance of Config Manager, create tenants, and users for specific
tenants. An example of the usage in a local dev environment is:

\begin{Shaded}
\begin{Highlighting}[]
\BuiltInTok{export} \VariableTok{PYTHONPATH=$PYTHONPATH}\NormalTok{:./:../stack-monitoring}
\ExtensionTok{python}\NormalTok{ bin/socotraadmin environment bootstrap_admin --name DockerDev --username docker_dev}
    \ExtensionTok{--password}\NormalTok{ socotra --account_type account.internal --api_url http://localhost:8080 --jwtsecret SGAGWfq31D2HRccsq87s33v1 }
\end{Highlighting}
\end{Shaded}

\hypertarget{adding-a-user-to-a-tenant}{%
\subsubsection{Adding a user to a
tenant}\label{adding-a-user-to-a-tenant}}

Once a tenant exists a user can be added to the tenant with a command of
the form

\begin{verbatim}
socotraadmin tenant add_user [tenant_name] [user_display_name] [username] [password] [email] [account_type]
\end{verbatim}

where the currently supported account types are account.tenant.employee
\textbar{} account.tenant.admin. Below is an example of a command that
creates a tenant admin account.

\begin{Shaded}
\begin{Highlighting}[]
\BuiltInTok{export} \VariableTok{PYTHONPATH=$PYTHONPATH}\NormalTok{:./:../stack-monitoring}
\ExtensionTok{python}\NormalTok{ bin/socotraadmin tenant add_user docker-dev-configeditor Admin admin.lee socotra admin.lee@email.com account.tenant.admin}
    \ExtensionTok{--api_url}\NormalTok{ http://localhost:8080 --jwtsecret SGAGWfq31D2HRccsq87s33v1 }
    \ExtensionTok{--admin_username}\NormalTok{ docker-dev --admin_password socotra}
\end{Highlighting}
\end{Shaded}

\hypertarget{finding-the-tenant-info-associated-with-a-hostname}{%
\subsubsection{Finding the tenant info associated with a
hostname}\label{finding-the-tenant-info-associated-with-a-hostname}}

The command usage is:

\begin{verbatim}
socotraadmin tenant find_tenant [--api_url] [--jwtsecret]
                                [--admin_username] [--admin_password]
                                hostname
\end{verbatim}

This will return the contents of the tenant table in Dynamodb given a
hostname. An example of the usage in a local dev environment is:

\begin{Shaded}
\begin{Highlighting}[]
\BuiltInTok{export} \VariableTok{PYTHONPATH=$PYTHONPATH}\NormalTok{:./:../stack-monitoring}
\ExtensionTok{python}\NormalTok{ bin/socotraadmin tenant find_tenant --api_url http://localhost:8080 --jwtsecret SGAGWfq31D2HRccsq87s33v1 --admin_username docker-dev }
    \ExtensionTok{--admin_password}\NormalTok{ socotra docker-dev-configeditor.co.socotra.com}
\end{Highlighting}
\end{Shaded}

