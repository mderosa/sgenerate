\hypertarget{overview}{%
\subsection{Overview}\label{overview}}

Config Manager allows user to create, modify, and deploy custom,
insurance Products.

\hypertarget{local-development-front-end}{%
\subsection{Local Development (Front
End)}\label{local-development-front-end}}

Install npm packages: \texttt{npm\ install} (You may have to restart
VSCode after this to pick up the installed components)

Generate the default configuration assets as .js files:
\texttt{npm\ run\ create-assets}

Compile the app and serve it via webpack-dev-server:
\texttt{npm\ run\ start-dev}

Chances are you'll need to open the app in a browser that has CORS
disabled (b/c the local frontend app will be hitting a deployed backend
(as opposed to a backend hosted locally). So this will produce a CORS
error). Run this command in a terminal to open a CORS-disabled version
of Chrome:
\texttt{open\ -n\ -a\ /Applications/Google\textbackslash{}\ Chrome.app/Contents/MacOS/Google\textbackslash{}\ Chrome\ -\/-args\ -\/-user-data-dir="/tmp/chrome\_dev\_test"\ -\/-disable-web-security}

\hypertarget{local-development-back-end}{%
\subsection{Local Development (Back
End)}\label{local-development-back-end}}

\begin{itemize}
\tightlist
\item
  Boot stack-api-service on port 8080 by running it through your IDE, as
  usual. Stack-api-services, directly, provides login services for
  config manager
\item
  Boot stack-load-assets on port 5000 by running the command
\end{itemize}

\begin{Shaded}
\begin{Highlighting}[]
\ExtensionTok{nvm}\NormalTok{ use 8.9.4}
\ExtensionTok{rvm}\NormalTok{ use ruby-2.6.3}
\FunctionTok{make}\NormalTok{ run}
\end{Highlighting}
\end{Shaded}

from the stack-load-assets base directory. Stack-load-assets provides
all of the asset publishing services for config manager.

\begin{itemize}
\tightlist
\item
  Boot stack-config-manager on port 9001 by running the command
\end{itemize}

\begin{Shaded}
\begin{Highlighting}[]
\FunctionTok{make}\NormalTok{ run}
\end{Highlighting}
\end{Shaded}

from the stack-config-manager base directory.

When all three applications are running you should then be able to
navigate to \href{http://localhost:9001}{http://localhost:9001} in any
browser, login, and submit assets all within a local development
environment.

\hypertarget{build-and-deploy}{%
\subsection{Build and deploy}\label{build-and-deploy}}

The build of Config Manager is coordinated by the build.sh (python 2) or
build3.sh (python 3) in the root of the project. The shell scripts
bundle the application and deploy it to
repo://source/stack-load-assets/assetload/static/configmanager. Once
deployed Config Manager can then be accessed through a running
load-assets application, at the path /studio/. So for example, running
load-assets locally the access url would be
\href{http://localhost:5000/studio/}{http://localhost:5000/studio/}

Build and deploy using Python 3 with:

\begin{Shaded}
\begin{Highlighting}[]
\ExtensionTok{rvm}\NormalTok{ use ruby-2.6.3}
\ExtensionTok{nvm}\NormalTok{ use 8.9.4}
\ExtensionTok{./build3.sh}
\end{Highlighting}
\end{Shaded}

\hypertarget{directory-layout}{%
\subsection{Directory Layout:}\label{directory-layout}}

\begin{itemize}
\tightlist
\item
  assets: Contains default configurations artifacts (i.e. default
  config, default product)
\item
  public: Contains icons and image assets
\item
  sass: Contains sass files for styling. Currently trying to move over
  to a BEM-style methodology
\item
  src

  \begin{itemize}
  \item
    api: Contains code to make requests to backend APIs
  \item
    components: Contains our React components

    \begin{itemize}
    \tightlist
    \item
      actionpane: code for the actions sidebar
    \item
      actions: Contains the Actions of the Trigger-Action component
      design
    \item
      modals: Contains modal components
    \item
      pages: Contains pages and layouts
    \item
      popups: Contains popup components (i.e. action menus, dropdown
      menus). Distinct from modals in that modals are fullscreen and
      darken the bg. Popups and modals also currently use different
      underlying libraries for their implementation. They should
      probably be consolidated at some point.
    \item
      various other components not organized into the above folders
    \end{itemize}
  \item
    context: Contains state management code. We're using React's Context
    API for state management.
  \item
    models: Contains any complex data models (i.e. the configuration)
  \item
    util: Contains various utility code
  \end{itemize}
\item
  history.tsx: Specifies the history library to be used with
  react-router
\item
  main.tsx: The main entry point for the app
\item
  routes.tsx: Specifies the app's URL routes. Uses react-router
\end{itemize}
