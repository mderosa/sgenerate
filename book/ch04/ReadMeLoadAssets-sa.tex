\hypertarget{overview}{%
\subsection{Overview}\label{overview}}

Load Assets takes configuration files from Configuration Manager,
validates them, and then orchestrates though the stack-api-service to
configure data stores needed for operation of the software

\hypertarget{development-environment-setup}{%
\subsection{Development environment
setup}\label{development-environment-setup}}

Load assets can be run locally on a developer computer against a locally
running instance of stack-api-service on port 8080. To setup your
environment you can run the command

\begin{Shaded}
\begin{Highlighting}[]
\FunctionTok{make}\NormalTok{ setup-debian}
\end{Highlighting}
\end{Shaded}

or

\begin{Shaded}
\begin{Highlighting}[]
\FunctionTok{make}\NormalTok{ setup}
\end{Highlighting}
\end{Shaded}

depending on the type of system you are on. The setup process (1)
installs pyenv to manage your virtual environments, (2) installs pipenv
and (3) installs all of the dependencies for the stack-load-assets
project

After running the setup you should then be able to start your virtual
environment by typing:

\begin{Shaded}
\begin{Highlighting}[]
\ExtensionTok{pipenv}\NormalTok{ shell}
\end{Highlighting}
\end{Shaded}

and exit the virtual environment by typing:

\begin{Shaded}
\begin{Highlighting}[]
\BuiltInTok{exit}
\end{Highlighting}
\end{Shaded}

\hypertarget{running-the-application}{%
\subsection{Running the application}\label{running-the-application}}

The application can be run on port 5000 by entering your virtual
environment and then entering the command

\begin{Shaded}
\begin{Highlighting}[]
\FunctionTok{make}\NormalTok{ run}
\end{Highlighting}
\end{Shaded}

The makefile sets up all of the environment variables needed for the
application to run against a locally running stack-api-service

\hypertarget{testing-the-application}{%
\subsection{Testing the application}\label{testing-the-application}}

Tests can be run from inside the virtual environment as described below.

\begin{itemize}
\tightlist
\item
  static analysis and type checking is run as part of all the test
  routines. Both types of checks can also be run stand alone. Static
  analysis can be run with the command:
\end{itemize}

\begin{Shaded}
\begin{Highlighting}[]
\FunctionTok{make}\NormalTok{ static}
\end{Highlighting}
\end{Shaded}

Type checking can be run with the comamnd:

\begin{Shaded}
\begin{Highlighting}[]
\FunctionTok{make}\NormalTok{ mypy}
\end{Highlighting}
\end{Shaded}

\begin{itemize}
\tightlist
\item
  unit tests can be run with the command:
\end{itemize}

\begin{Shaded}
\begin{Highlighting}[]
\FunctionTok{make}\NormalTok{ test}
\end{Highlighting}
\end{Shaded}

\begin{itemize}
\tightlist
\item
  There are currently integration tests for stack-load-assets that run
  against a locally running stack-api-service. The integration tests can
  be run with the command:
\end{itemize}

\begin{Shaded}
\begin{Highlighting}[]
\FunctionTok{make}\NormalTok{ test_intg}
\end{Highlighting}
\end{Shaded}

\hypertarget{building-the-application}{%
\subsection{Building the application}\label{building-the-application}}

Though not immediately obvious, Load Assets is built into a composite
application where (a) python code orchestrates communication with the
socotra api, and (b) where the assetload/static directory contains and
serves the build stack-config-studio-static, stack-load-assets-static,
and stack-config-manager projects. The assembly of all these elements is
managed by the build so just be aware there is more to Load Assets than
meets the eye.
