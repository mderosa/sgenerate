\hypertarget{to-run-locally}{%
\subsection{To run locally:}\label{to-run-locally}}

\hypertarget{front-end-development}{%
\subsubsection{Front end development}\label{front-end-development}}

Install npm packages: \texttt{npm\ install} (You may have to restart
VSCode after this to pick up the installed components)

Compile the app: \texttt{npm\ run\ build}

Serve it locally via webpack-dev-server: \texttt{npm\ start}

Before running locally, ensure the following environment variables are
set appropriately:

\begin{verbatim}
API_URL=<url_of_backend i.e. api.develop.socotra.com>
TENANT_HOSTNAME=<your_tenant_name i.e. johndoe-configeditor.co.develop.socotra.com>
PORT=<this variable is optional. Specify port number to serve the app. i.e. 8080>
\end{verbatim}

Note: If you load the app in a browser and log in but experience weird
errors, check the console. If there's weird errors (i.e. 'Uncaught
Error: Expected to find root ID'), you may need to try loading the app
in a CORS-disabled browser. On a Mac, run this command in a terminal to
open a CORS-disabled version of Chrome:
\texttt{open\ -n\ -a\ /Applications/Google\textbackslash{}\ Chrome.app/Contents/MacOS/Google\textbackslash{}\ Chrome\ -\/-args\ -\/-user-data-dir="/tmp/chrome\_dev\_test"\ -\/-disable-web-security}

When you run the app, you can log in with the default username and
password (Note: these credentials are only available in development.
They're disabled in production):

\begin{verbatim}
username: alice.lee
pw: socotra
\end{verbatim}

If you want to log in to the 'Administration' side of the app to manage
users and external integrations, you'll need to use your tenant login
(not alice.lee).

\hypertarget{back-end-development}{%
\subsubsection{Back end development}\label{back-end-development}}

If you have a standard backend setup and have setup the docker-dev admin
account, via

\begin{Shaded}
\begin{Highlighting}[]
\ExtensionTok{brew}\NormalTok{ install jq jo}
\BuiltInTok{cd}\NormalTok{ ~/Code/socotra-stack/docker-dev}
\FunctionTok{make}\NormalTok{ tenant}
\end{Highlighting}
\end{Shaded}

You will be able to start stack-app-static on port 8081 with the command

\begin{Shaded}
\begin{Highlighting}[]
\FunctionTok{make}\NormalTok{ run}
\end{Highlighting}
\end{Shaded}

With stack-app-static running you should then be able to navigate to
\href{http://docker-dev-configeditor.co.socotra.com:8081/}{http://docker-dev-configeditor.co.socotra.com:8081/}
locally in a browser (assuming the line '127.0.0.1
docker-dev-configeditor.co.socotra.com' in /etc/hosts) and login with
the username/password alice.lee/socotra

To run stack-app-static along with the entire complement of Config
Manager, Load Assets, and the API refer to the README.md in the root of
the stack-config-manager directory.
