\hypertarget{overview}{%
\subsection{Overview}\label{overview}}

The stack-python-client package provides an interface for communicating
with the stack-api-service. There are several utility programs in the
bin directory with can be used to setup a client. In addition this
package is used by stack-load-assets to help upload zip asset files.

\hypertarget{development-environment-setup}{%
\subsection{Development environment
setup}\label{development-environment-setup}}

This library is a dependency of the stack-load-assets application and
its development setup procedures are similar to stack-load-assets. To
setup your development environment for working on stack-python-client,
from the project base directory type:

\begin{Shaded}
\begin{Highlighting}[]
\FunctionTok{make}\NormalTok{ setup}
\end{Highlighting}
\end{Shaded}

This will install pyenv, pipenv and lastly install the project
dependences. After running the setup you should then be able to start
your virtual environment by typing:

\begin{Shaded}
\begin{Highlighting}[]
\ExtensionTok{pipenv}\NormalTok{ shell}
\end{Highlighting}
\end{Shaded}

and exit the virtual environment by typing:

\begin{Shaded}
\begin{Highlighting}[]
\BuiltInTok{exit}
\end{Highlighting}
\end{Shaded}

\hypertarget{testing-the-application}{%
\subsection{Testing the application}\label{testing-the-application}}

You will be able to run tests in your development environment by
activating your virtual environment (see: Development setup) and
entering the commands below.

\begin{itemize}
\tightlist
\item
  Static analysis can be run with pylint through the command
\end{itemize}

\begin{verbatim}
make static
\end{verbatim}

\begin{itemize}
\tightlist
\item
  The unit tests can be run wit the command
\end{itemize}

\begin{verbatim}
make test
\end{verbatim}

\begin{itemize}
\tightlist
\item
  The integration tests can be run by booting the Java API running on
  port 8080 and then entering the command
\end{itemize}

\begin{verbatim}
make test_intg
\end{verbatim}
