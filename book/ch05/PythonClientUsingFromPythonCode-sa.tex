\hypertarget{using-the-python-client-library}{%
\subsection{Using the Python Client
library}\label{using-the-python-client-library}}

Do the following to instantiate a Client class that will allow you to
communicate with a Socotra API:

\begin{Shaded}
\begin{Highlighting}[]
\ImportTok{from}\NormalTok{ socotra }\ImportTok{import}\NormalTok{ Client}
\NormalTok{c }\OperatorTok{=}\NormalTok{ Client(}\StringTok{'http://localhost:8080'}\NormalTok{)}
\NormalTok{c.authenticate(user_nm, pwd, tenant_loc)}
\end{Highlighting}
\end{Shaded}

Once you have authenticated, you may then call methods on the client
object which will in turn make calls into the Socotra API. Below are a
few documented methods. See the socotra/\textbf{init}.py file for more
possibilities.

\hypertarget{get_tenants}{%
\subsubsection{get\_tenants}\label{get_tenants}}

This grabs a json blob of all tenants available in the system currently.

\begin{Shaded}
\begin{Highlighting}[]
\NormalTok{tenants }\OperatorTok{=}\NormalTok{ c.get_tenants()}
\end{Highlighting}
\end{Shaded}

\hypertarget{add_tenant}{%
\subsubsection{add\_tenant}\label{add_tenant}}

Adds a tenant

Params:

\begin{itemize}
\tightlist
\item
  name (str): Name of the tenant being added
\item
  type (str): Type of tenant
\end{itemize}

\begin{Shaded}
\begin{Highlighting}[]
\NormalTok{c.add_tenant(}\StringTok{'Blah'}\NormalTok{, }\StringTok{'tenant.test'}\NormalTok{))}
\end{Highlighting}
\end{Shaded}

\hypertarget{add_hostname}{%
\subsubsection{add\_hostname}\label{add_hostname}}

Adds a hostname

Params:

\begin{itemize}
\tightlist
\item
  id (str): The ID of the tenant
\item
  hostname (str): The hostname of the tenant
\end{itemize}

\begin{Shaded}
\begin{Highlighting}[]
\NormalTok{c.add_hostname(tenant_data[}\StringTok{'id'}\NormalTok{], }\StringTok{'balls'}\NormalTok{)}

\end{Highlighting}
\end{Shaded}

\hypertarget{ping}{%
\subsubsection{ping}\label{ping}}

Ensures the server is up and ready for requests

Params:

\begin{itemize}
\tightlist
\item
  authorized (bool): Defaults to false
\end{itemize}

\begin{Shaded}
\begin{Highlighting}[]
\NormalTok{c.ping() }\CommentTok{# will use the endpoint that does not require login}
\NormalTok{c.ping(authorized}\OperatorTok{=}\VariableTok{True}\NormalTok{) }\CommentTok{# will use the endpoint that requires login}
\end{Highlighting}
\end{Shaded}

