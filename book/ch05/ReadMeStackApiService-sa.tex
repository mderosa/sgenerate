\hypertarget{overview}{%
\subsection{Overview}\label{overview}}

stack-api-service is the highest layer in the Socotra platform. It
depends on the statck-service-common project for business and
persistence services. It despends on the stack-api-client project for
generic and specialized model classes.

\hypertarget{testing}{%
\subsection{Testing}\label{testing}}

\hypertarget{overview-1}{%
\subsubsection{Overview}\label{overview-1}}

Testing is organized so that a developer can run unit tests and an
essential subset of the integration tests locally on a development
machine. These unit and essential integration tests are referred to as
the essential test subset and satisfy the following criteria:

\begin{itemize}
\tightlist
\item
  Runs must of the essential test subset must run within a time budget
  of 15 minutes.
\item
  The essential test subset may be run locally by a developer, but the
  essential subset must run and pass before developers are allowed to
  merge a pull request into the develop branch.
\end{itemize}

The full set of project tests, which is a superset of the essential
subset, is run continuously by the automated build tools. A failure of
the full set indicates a code issue on the develop branch and must be
resolved by the team as a their first priority.

\hypertarget{running-tests}{%
\subsubsection{Running tests}\label{running-tests}}

\begin{itemize}
\tightlist
\item
  The unit tests for stack-service-common will run on any machine with a
  Java installation. The unit tests can be run by the command
\end{itemize}

\begin{Shaded}
\begin{Highlighting}[]
\FunctionTok{make}\NormalTok{ test}
\end{Highlighting}
\end{Shaded}

This command insures the local maven repository contains a build of the
latest branch specific Java libraries and runs all of the Java unit
tests in the socotra-stack project. If one would like to run just the
unit tests for the stack-api-service project, that can be done by
issuing the command

\begin{Shaded}
\begin{Highlighting}[]
\ExtensionTok{mvn}\NormalTok{ test}
\end{Highlighting}
\end{Shaded}

from the root of the project.

\begin{itemize}
\tightlist
\item
  The essential test subset, as defined above, for stack-service-common
  will run on any machine with a standard development environment. The
  essential subset can be run by the command
\end{itemize}

\begin{verbatim}
make test_intg
\end{verbatim}

If a developer would like to run just the essential test subset for the
stack-service common project alone, that can be done by issuing the
command

\begin{Shaded}
\begin{Highlighting}[]
\ExtensionTok{mvn}\NormalTok{ verify -P essential-test}
\end{Highlighting}
\end{Shaded}

from the root of the project.

\begin{itemize}
\tightlist
\item
  Currently the full set of project tests will run with the command.
\end{itemize}

\begin{verbatim}
mvn verify -P integration-test
\end{verbatim}

A developer machine will likely choke on the processing that will ensue
after this command so you really don't want to run all the tests. Still,
everyone should be familiar with the tests commands that are used in the
orgainization.
