\batchmode %% Suppresses most terminal output.
\documentclass{article}
\usepackage{color}
\definecolor{boxshade}{gray}{0.85}
\setlength{\textwidth}{360pt}
\setlength{\textheight}{541pt}
\usepackage{latexsym}
\usepackage{ifthen}
% \usepackage{color}
%%%%%%%%%%%%%%%%%%%%%%%%%%%%%%%%%%%%%%%%%%%%%%%%%%%%%%%%%%%%%%%%%%%%%%%%%%%%%
% SWITCHES                                                                  %
%%%%%%%%%%%%%%%%%%%%%%%%%%%%%%%%%%%%%%%%%%%%%%%%%%%%%%%%%%%%%%%%%%%%%%%%%%%%%
\newboolean{shading} 
\setboolean{shading}{false}
\makeatletter
 %% this is needed only when inserted into the file, not when
 %% used as a package file.
%%%%%%%%%%%%%%%%%%%%%%%%%%%%%%%%%%%%%%%%%%%%%%%%%%%%%%%%%%%%%%%%%%%%%%%%%%%%%
%                                                                           %
% DEFINITIONS OF SYMBOL-PRODUCING COMMANDS                                  %
%                                                                           %
%    TLA+      LaTeX                                                        %
%    symbol    command                                                      %
%    ------    -------                                                      %
%    =>        \implies                                                     %
%    <:        \ltcolon                                                     %
%    :>        \colongt                                                     %
%    ==        \defeq                                                       %
%    ..        \dotdot                                                      %
%    ::        \coloncolon                                                  %
%    =|        \eqdash                                                      %
%    ++        \pp                                                          %
%    --        \mm                                                          %
%    **        \stst                                                        %
%    //        \slsl                                                        %
%    ^         \ct                                                          %
%    \A        \A                                                           %
%    \E        \E                                                           %
%    \AA       \AA                                                          %
%    \EE       \EE                                                          %
%%%%%%%%%%%%%%%%%%%%%%%%%%%%%%%%%%%%%%%%%%%%%%%%%%%%%%%%%%%%%%%%%%%%%%%%%%%%%
\newlength{\symlength}
\newcommand{\implies}{\Rightarrow}
\newcommand{\ltcolon}{\mathrel{<\!\!\mbox{:}}}
\newcommand{\colongt}{\mathrel{\!\mbox{:}\!\!>}}
\newcommand{\defeq}{\;\mathrel{\smash   %% keep this symbol from being too tall
    {{\stackrel{\scriptscriptstyle\Delta}{=}}}}\;}
\newcommand{\dotdot}{\mathrel{\ldotp\ldotp}}
\newcommand{\coloncolon}{\mathrel{::\;}}
\newcommand{\eqdash}{\mathrel = \joinrel \hspace{-.28em}|}
\newcommand{\pp}{\mathbin{++}}
\newcommand{\mm}{\mathbin{--}}
\newcommand{\stst}{*\!*}
\newcommand{\slsl}{/\!/}
\newcommand{\ct}{\hat{\hspace{.4em}}}
\newcommand{\A}{\forall}
\newcommand{\E}{\exists}
\renewcommand{\AA}{\makebox{$\raisebox{.05em}{\makebox[0pt][l]{%
   $\forall\hspace{-.517em}\forall\hspace{-.517em}\forall$}}%
   \forall\hspace{-.517em}\forall \hspace{-.517em}\forall\,$}}
\newcommand{\EE}{\makebox{$\raisebox{.05em}{\makebox[0pt][l]{%
   $\exists\hspace{-.517em}\exists\hspace{-.517em}\exists$}}%
   \exists\hspace{-.517em}\exists\hspace{-.517em}\exists\,$}}
\newcommand{\whileop}{\.{\stackrel
  {\mbox{\raisebox{-.3em}[0pt][0pt]{$\scriptscriptstyle+\;\,$}}}%
  {-\hspace{-.16em}\triangleright}}}

% Commands are defined to produce the upper-case keywords.
% Note that some have space after them.
\newcommand{\ASSUME}{\textsc{assume }}
\newcommand{\ASSUMPTION}{\textsc{assumption }}
\newcommand{\AXIOM}{\textsc{axiom }}
\newcommand{\BOOLEAN}{\textsc{boolean }}
\newcommand{\CASE}{\textsc{case }}
\newcommand{\CONSTANT}{\textsc{constant }}
\newcommand{\CONSTANTS}{\textsc{constants }}
\newcommand{\ELSE}{\settowidth{\symlength}{\THEN}%
   \makebox[\symlength][l]{\textsc{ else}}}
\newcommand{\EXCEPT}{\textsc{ except }}
\newcommand{\EXTENDS}{\textsc{extends }}
\newcommand{\FALSE}{\textsc{false}}
\newcommand{\IF}{\textsc{if }}
\newcommand{\IN}{\settowidth{\symlength}{\LET}%
   \makebox[\symlength][l]{\textsc{in}}}
\newcommand{\INSTANCE}{\textsc{instance }}
\newcommand{\LET}{\textsc{let }}
\newcommand{\LOCAL}{\textsc{local }}
\newcommand{\MODULE}{\textsc{module }}
\newcommand{\OTHER}{\textsc{other }}
\newcommand{\STRING}{\textsc{string}}
\newcommand{\THEN}{\textsc{ then }}
\newcommand{\THEOREM}{\textsc{theorem }}
\newcommand{\LEMMA}{\textsc{lemma }}
\newcommand{\PROPOSITION}{\textsc{proposition }}
\newcommand{\COROLLARY}{\textsc{corollary }}
\newcommand{\TRUE}{\textsc{true}}
\newcommand{\VARIABLE}{\textsc{variable }}
\newcommand{\VARIABLES}{\textsc{variables }}
\newcommand{\WITH}{\textsc{ with }}
\newcommand{\WF}{\textrm{WF}}
\newcommand{\SF}{\textrm{SF}}
\newcommand{\CHOOSE}{\textsc{choose }}
\newcommand{\ENABLED}{\textsc{enabled }}
\newcommand{\UNCHANGED}{\textsc{unchanged }}
\newcommand{\SUBSET}{\textsc{subset }}
\newcommand{\UNION}{\textsc{union }}
\newcommand{\DOMAIN}{\textsc{domain }}
% Added for tla2tex
\newcommand{\BY}{\textsc{by }}
\newcommand{\OBVIOUS}{\textsc{obvious }}
\newcommand{\HAVE}{\textsc{have }}
\newcommand{\QED}{\textsc{qed }}
\newcommand{\TAKE}{\textsc{take }}
\newcommand{\DEF}{\textsc{ def }}
\newcommand{\HIDE}{\textsc{hide }}
\newcommand{\RECURSIVE}{\textsc{recursive }}
\newcommand{\USE}{\textsc{use }}
\newcommand{\DEFINE}{\textsc{define }}
\newcommand{\PROOF}{\textsc{proof }}
\newcommand{\WITNESS}{\textsc{witness }}
\newcommand{\PICK}{\textsc{pick }}
\newcommand{\DEFS}{\textsc{defs }}
\newcommand{\PROVE}{\settowidth{\symlength}{\ASSUME}%
   \makebox[\symlength][l]{\textsc{prove}}\@s{-4.1}}%
  %% The \@s{-4.1) is a kludge added on 24 Oct 2009 [happy birthday, Ellen]
  %% so the correct alignment occurs if the user types
  %%   ASSUME X
  %%   PROVE  Y
  %% because it cancels the extra 4.1 pts added because of the 
  %% extra space after the PROVE.  This seems to works OK.
  %% However, the 4.1 equals Parameters.LaTeXLeftSpace(1) and
  %% should be changed if that method ever changes.
\newcommand{\SUFFICES}{\textsc{suffices }}
\newcommand{\NEW}{\textsc{new }}
\newcommand{\LAMBDA}{\textsc{lambda }}
\newcommand{\STATE}{\textsc{state }}
\newcommand{\ACTION}{\textsc{action }}
\newcommand{\TEMPORAL}{\textsc{temporal }}
\newcommand{\ONLY}{\textsc{only }}              %% added by LL on 2 Oct 2009
\newcommand{\OMITTED}{\textsc{omitted }}        %% added by LL on 31 Oct 2009
\newcommand{\@pfstepnum}[2]{\ensuremath{\langle#1\rangle}\textrm{#2}}
\newcommand{\bang}{\@s{1}\mbox{\small !}\@s{1}}
%% We should format || differently in PlusCal code than in TLA+ formulas.
\newcommand{\p@barbar}{\ifpcalsymbols
   \,\,\rule[-.25em]{.075em}{1em}\hspace*{.2em}\rule[-.25em]{.075em}{1em}\,\,%
   \else \,||\,\fi}
%% PlusCal keywords
\newcommand{\p@fair}{\textbf{fair }}
\newcommand{\p@semicolon}{\textbf{\,; }}
\newcommand{\p@algorithm}{\textbf{algorithm }}
\newcommand{\p@mmfair}{\textbf{-{}-fair }}
\newcommand{\p@mmalgorithm}{\textbf{-{}-algorithm }}
\newcommand{\p@assert}{\textbf{assert }}
\newcommand{\p@await}{\textbf{await }}
\newcommand{\p@begin}{\textbf{begin }}
\newcommand{\p@end}{\textbf{end }}
\newcommand{\p@call}{\textbf{call }}
\newcommand{\p@define}{\textbf{define }}
\newcommand{\p@do}{\textbf{ do }}
\newcommand{\p@either}{\textbf{either }}
\newcommand{\p@or}{\textbf{or }}
\newcommand{\p@goto}{\textbf{goto }}
\newcommand{\p@if}{\textbf{if }}
\newcommand{\p@then}{\,\,\textbf{then }}
\newcommand{\p@else}{\ifcsyntax \textbf{else } \else \,\,\textbf{else }\fi}
\newcommand{\p@elsif}{\,\,\textbf{elsif }}
\newcommand{\p@macro}{\textbf{macro }}
\newcommand{\p@print}{\textbf{print }}
\newcommand{\p@procedure}{\textbf{procedure }}
\newcommand{\p@process}{\textbf{process }}
\newcommand{\p@return}{\textbf{return}}
\newcommand{\p@skip}{\textbf{skip}}
\newcommand{\p@variable}{\textbf{variable }}
\newcommand{\p@variables}{\textbf{variables }}
\newcommand{\p@while}{\textbf{while }}
\newcommand{\p@when}{\textbf{when }}
\newcommand{\p@with}{\textbf{with }}
\newcommand{\p@lparen}{\textbf{(\,\,}}
\newcommand{\p@rparen}{\textbf{\,\,) }}   
\newcommand{\p@lbrace}{\textbf{\{\,\,}}   
\newcommand{\p@rbrace}{\textbf{\,\,\} }}

%%%%%%%%%%%%%%%%%%%%%%%%%%%%%%%%%%%%%%%%%%%%%%%%%%%%%%%%%
% REDEFINE STANDARD COMMANDS TO MAKE THEM FORMAT BETTER %
%                                                       %
% We redefine \in and \notin                            %
%%%%%%%%%%%%%%%%%%%%%%%%%%%%%%%%%%%%%%%%%%%%%%%%%%%%%%%%%
\renewcommand{\_}{\rule{.4em}{.06em}\hspace{.05em}}
\newlength{\equalswidth}
\let\oldin=\in
\let\oldnotin=\notin
\renewcommand{\in}{%
   {\settowidth{\equalswidth}{$\.{=}$}\makebox[\equalswidth][c]{$\oldin$}}}
\renewcommand{\notin}{%
   {\settowidth{\equalswidth}{$\.{=}$}\makebox[\equalswidth]{$\oldnotin$}}}


%%%%%%%%%%%%%%%%%%%%%%%%%%%%%%%%%%%%%%%%%%%%%%%%%%%%
%                                                  %
% HORIZONTAL BARS:                                 %
%                                                  %
%   \moduleLeftDash    |~~~~~~~~~~                 %
%   \moduleRightDash    ~~~~~~~~~~|                %
%   \midbar            |----------|                %
%   \bottombar         |__________|                %
%%%%%%%%%%%%%%%%%%%%%%%%%%%%%%%%%%%%%%%%%%%%%%%%%%%%
\newlength{\charwidth}\settowidth{\charwidth}{{\small\tt M}}
\newlength{\boxrulewd}\setlength{\boxrulewd}{.4pt}
\newlength{\boxlineht}\setlength{\boxlineht}{.5\baselineskip}
\newcommand{\boxsep}{\charwidth}
\newlength{\boxruleht}\setlength{\boxruleht}{.5ex}
\newlength{\boxruledp}\setlength{\boxruledp}{-\boxruleht}
\addtolength{\boxruledp}{\boxrulewd}
\newcommand{\boxrule}{\leaders\hrule height \boxruleht depth \boxruledp
                      \hfill\mbox{}}
\newcommand{\@computerule}{%
  \setlength{\boxruleht}{.5ex}%
  \setlength{\boxruledp}{-\boxruleht}%
  \addtolength{\boxruledp}{\boxrulewd}}

\newcommand{\bottombar}{\hspace{-\boxsep}%
  \raisebox{-\boxrulewd}[0pt][0pt]{\rule[.5ex]{\boxrulewd}{\boxlineht}}%
  \boxrule
  \raisebox{-\boxrulewd}[0pt][0pt]{%
      \rule[.5ex]{\boxrulewd}{\boxlineht}}\hspace{-\boxsep}\vspace{0pt}}

\newcommand{\moduleLeftDash}%
   {\hspace*{-\boxsep}%
     \raisebox{-\boxlineht}[0pt][0pt]{\rule[.5ex]{\boxrulewd
               }{\boxlineht}}%
    \boxrule\hspace*{.4em }}

\newcommand{\moduleRightDash}%
    {\hspace*{.4em}\boxrule
    \raisebox{-\boxlineht}[0pt][0pt]{\rule[.5ex]{\boxrulewd
               }{\boxlineht}}\hspace{-\boxsep}}%\vspace{.2em}

\newcommand{\midbar}{\hspace{-\boxsep}\raisebox{-.5\boxlineht}[0pt][0pt]{%
   \rule[.5ex]{\boxrulewd}{\boxlineht}}\boxrule\raisebox{-.5\boxlineht%
   }[0pt][0pt]{\rule[.5ex]{\boxrulewd}{\boxlineht}}\hspace{-\boxsep}}

%%%%%%%%%%%%%%%%%%%%%%%%%%%%%%%%%%%%%%%%%%%%%%%%%%%%%%%%%%%%%%%%%%%%%%%%%%%%%
% FORMATING COMMANDS                                                        %
%%%%%%%%%%%%%%%%%%%%%%%%%%%%%%%%%%%%%%%%%%%%%%%%%%%%%%%%%%%%%%%%%%%%%%%%%%%%%

%%%%%%%%%%%%%%%%%%%%%%%%%%%%%%%%%%%%%%%%%%%%%%%%%%%%%%%%%%%%%%%%%%%%%%%%%%%%%
% PLUSCAL SHADING                                                           %
%%%%%%%%%%%%%%%%%%%%%%%%%%%%%%%%%%%%%%%%%%%%%%%%%%%%%%%%%%%%%%%%%%%%%%%%%%%%%

% The TeX pcalshading switch is set on to cause PlusCal shading to be
% performed.  This changes the behavior of the following commands and
% environments to cause full-width shading to be performed on all lines.
% 
%   \tstrut \@x cpar mcom \@pvspace
% 
% The TeX pcalsymbols switch is turned on when typesetting a PlusCal algorithm,
% whether or not shading is being performed.  It causes symbols (other than
% parentheses and braces and PlusCal-only keywords) that should be typeset
% differently depending on whether they are in an algorithm to be typeset
% appropriately.  Currently, the only such symbol is "||".
%
% The TeX csyntax switch is turned on when typesetting a PlusCal algorithm in
% c-syntax.  This allows symbols to be format differently in the two syntaxes.
% The "else" keyword is the only one that is.

\newif\ifpcalshading \pcalshadingfalse
\newif\ifpcalsymbols \pcalsymbolsfalse
\newif\ifcsyntax     \csyntaxtrue

% The \@pvspace command makes a vertical space.  It uses \vspace
% except with \ifpcalshading, in which case it sets \pvcalvspace
% and the space is added by a following \@x command.
%
\newlength{\pcalvspace}\setlength{\pcalvspace}{0pt}%
\newcommand{\@pvspace}[1]{%
  \ifpcalshading
     \par\global\setlength{\pcalvspace}{#1}%
  \else
     \par\vspace{#1}%
  \fi
}

% The lcom environment was changed to set \lcomindent equal to
% the indentation it produces.  This length is used by the
% cpar environment to make shading extend for the full width
% of the line.  This assumes that lcom environments are not
% nested.  I hope TLATeX does not nest them.
%
\newlength{\lcomindent}%
\setlength{\lcomindent}{0pt}%

%\tstrut: A strut to produce inter-paragraph space in a comment.
%\rstrut: A strut to extend the bottom of a one-line comment so
%         there's no break in the shading between comments on 
%         successive lines.
\newcommand\tstrut%
  {\raisebox{\vshadelen}{\raisebox{-.25em}{\rule{0pt}{1.15em}}}%
   \global\setlength{\vshadelen}{0pt}}
\newcommand\rstrut{\raisebox{-.25em}{\rule{0pt}{1.15em}}%
 \global\setlength{\vshadelen}{0pt}}


% \.{op} formats operator op in math mode with empty boxes on either side.
% Used because TeX otherwise vary the amount of space it leaves around op.
\renewcommand{\.}[1]{\ensuremath{\mbox{}#1\mbox{}}}

% \@s{n} produces an n-point space
\newcommand{\@s}[1]{\hspace{#1pt}}           

% \@x{txt} starts a specification line in the beginning with txt
% in the final LaTeX source.
\newlength{\@xlen}
\newcommand\xtstrut%
  {\setlength{\@xlen}{1.05em}%
   \addtolength{\@xlen}{\pcalvspace}%
    \raisebox{\vshadelen}{\raisebox{-.25em}{\rule{0pt}{\@xlen}}}%
   \global\setlength{\vshadelen}{0pt}%
   \global\setlength{\pcalvspace}{0pt}}

\newcommand{\@x}[1]{\par
  \ifpcalshading
  \makebox[0pt][l]{\shadebox{\xtstrut\hspace*{\textwidth}}}%
  \fi
  \mbox{$\mbox{}#1\mbox{}$}}  

% \@xx{txt} continues a specification line with the text txt.
\newcommand{\@xx}[1]{\mbox{$\mbox{}#1\mbox{}$}}  

% \@y{cmt} produces a one-line comment.
\newcommand{\@y}[1]{\mbox{\footnotesize\hspace{.65em}%
  \ifthenelse{\boolean{shading}}{%
      \shadebox{#1\hspace{-\the\lastskip}\rstrut}}%
               {#1\hspace{-\the\lastskip}\rstrut}}}

% \@z{cmt} produces a zero-width one-line comment.
\newcommand{\@z}[1]{\makebox[0pt][l]{\footnotesize
  \ifthenelse{\boolean{shading}}{%
      \shadebox{#1\hspace{-\the\lastskip}\rstrut}}%
               {#1\hspace{-\the\lastskip}\rstrut}}}


% \@w{str} produces the TLA+ string "str".
\newcommand{\@w}[1]{\textsf{``{#1}''}}             


%%%%%%%%%%%%%%%%%%%%%%%%%%%%%%%%%%%%%%%%%%%%%%%%%%%%%%%%%%%%%%%%%%%%%%%%%%%%%
% SHADING                                                                   %
%%%%%%%%%%%%%%%%%%%%%%%%%%%%%%%%%%%%%%%%%%%%%%%%%%%%%%%%%%%%%%%%%%%%%%%%%%%%%
\def\graymargin{1}
  % The number of points of margin in the shaded box.

% \definecolor{boxshade}{gray}{.85}
% Defines the darkness of the shading: 1 = white, 0 = black
% Added by TLATeX only if needed.

% \shadebox{txt} puts txt in a shaded box.
\newlength{\templena}
\newlength{\templenb}
\newsavebox{\tempboxa}
\newcommand{\shadebox}[1]{{\setlength{\fboxsep}{\graymargin pt}%
     \savebox{\tempboxa}{#1}%
     \settoheight{\templena}{\usebox{\tempboxa}}%
     \settodepth{\templenb}{\usebox{\tempboxa}}%
     \hspace*{-\fboxsep}\raisebox{0pt}[\templena][\templenb]%
        {\colorbox{boxshade}{\usebox{\tempboxa}}}\hspace*{-\fboxsep}}}

% \vshade{n} makes an n-point inter-paragraph space, with
%  shading if the `shading' flag is true.
\newlength{\vshadelen}
\setlength{\vshadelen}{0pt}
\newcommand{\vshade}[1]{\ifthenelse{\boolean{shading}}%
   {\global\setlength{\vshadelen}{#1pt}}%
   {\vspace{#1pt}}}

\newlength{\boxwidth}
\newlength{\multicommentdepth}

%%%%%%%%%%%%%%%%%%%%%%%%%%%%%%%%%%%%%%%%%%%%%%%%%%%%%%%%%%%%%%%%%%%%%%%%%%%%%
% THE cpar ENVIRONMENT                                                      %
% ^^^^^^^^^^^^^^^^^^^^                                                      %
% The LaTeX input                                                           %
%                                                                           %
%   \begin{cpar}{pop}{nest}{isLabel}{d}{e}{arg6}                            %
%     XXXXXXXXXXXXXXX                                                       %
%     XXXXXXXXXXXXXXX                                                       %
%     XXXXXXXXXXXXXXX                                                       %
%   \end{cpar}                                                              %
%                                                                           %
% produces one of two possible results.  If isLabel is the letter "T",      %
% it produces the following, where [label] is the result of typesetting     %
% arg6 in an LR box, and d is is a number representing a distance in        %
% points.                                                                   %
%                                                                           %
%   prevailing |<-- d -->[label]<- e ->XXXXXXXXXXXXXXX                      %
%         left |                       XXXXXXXXXXXXXXX                      %
%       margin |                       XXXXXXXXXXXXXXX                      %
%                                                                           %
% If isLabel is the letter "F", then it produces                            %
%                                                                           %
%   prevailing |<-- d -->XXXXXXXXXXXXXXXXXXXXXXX                            %
%         left |         <- e ->XXXXXXXXXXXXXXXX                            %
%       margin |                XXXXXXXXXXXXXXXX                            %
%                                                                           %
% where d and e are numbers representing distances in points.               %
%                                                                           %
% The prevailing left margin is the one in effect before the most recent    %
% pop (argument 1) cpar environments with "T" as the nest argument, where   %
% pop is a number \geq 0.                                                   %
%                                                                           %
% If the nest argument is the letter "T", then the prevailing left          %
% margin is moved to the left of the second (and following) lines of        %
% X's.  Otherwise, the prevailing left margin is left unchanged.            %
%                                                                           %
% An \unnest{n} command moves the prevailing left margin to where it was    %
% before the most recent n cpar environments with "T" as the nesting        %
% argument.                                                                 %
%                                                                           %
% The environment leaves no vertical space above or below it, or between    %
% its paragraphs.  (TLATeX inserts the proper amount of vertical space.)    %
%%%%%%%%%%%%%%%%%%%%%%%%%%%%%%%%%%%%%%%%%%%%%%%%%%%%%%%%%%%%%%%%%%%%%%%%%%%%%

\newcounter{pardepth}
\setcounter{pardepth}{0}

% \setgmargin{txt} defines \gmarginN to be txt, where N is \roman{pardepth}.
% \thegmargin equals \gmarginN, where N is \roman{pardepth}.
\newcommand{\setgmargin}[1]{%
  \expandafter\xdef\csname gmargin\roman{pardepth}\endcsname{#1}}
\newcommand{\thegmargin}{\csname gmargin\roman{pardepth}\endcsname}
\newcommand{\gmargin}{0pt}

\newsavebox{\tempsbox}

\newlength{\@cparht}
\newlength{\@cpardp}
\newenvironment{cpar}[6]{%
  \addtocounter{pardepth}{-#1}%
  \ifthenelse{\boolean{shading}}{\par\begin{lrbox}{\tempsbox}%
                                 \begin{minipage}[t]{\linewidth}}{}%
  \begin{list}{}{%
     \edef\temp{\thegmargin}
     \ifthenelse{\equal{#3}{T}}%
       {\settowidth{\leftmargin}{\hspace{\temp}\footnotesize #6\hspace{#5pt}}%
        \addtolength{\leftmargin}{#4pt}}%
       {\setlength{\leftmargin}{#4pt}%
        \addtolength{\leftmargin}{#5pt}%
        \addtolength{\leftmargin}{\temp}%
        \setlength{\itemindent}{-#5pt}}%
      \ifthenelse{\equal{#2}{T}}{\addtocounter{pardepth}{1}%
                                 \setgmargin{\the\leftmargin}}{}%
      \setlength{\labelwidth}{0pt}%
      \setlength{\labelsep}{0pt}%
      \setlength{\itemindent}{-\leftmargin}%
      \setlength{\topsep}{0pt}%
      \setlength{\parsep}{0pt}%
      \setlength{\partopsep}{0pt}%
      \setlength{\parskip}{0pt}%
      \setlength{\itemsep}{0pt}
      \setlength{\itemindent}{#4pt}%
      \addtolength{\itemindent}{-\leftmargin}}%
   \ifthenelse{\equal{#3}{T}}%
      {\item[\tstrut\footnotesize \hspace{\temp}{#6}\hspace{#5pt}]
        }%
      {\item[\tstrut\hspace{\temp}]%
         }%
   \footnotesize}
 {\hspace{-\the\lastskip}\tstrut
 \end{list}%
  \ifthenelse{\boolean{shading}}%
          {\end{minipage}%
           \end{lrbox}%
           \ifpcalshading
             \setlength{\@cparht}{\ht\tempsbox}%
             \setlength{\@cpardp}{\dp\tempsbox}%
             \addtolength{\@cparht}{.15em}%
             \addtolength{\@cpardp}{.2em}%
             \addtolength{\@cparht}{\@cpardp}%
            % I don't know what's going on here.  I want to add a
            % \pcalvspace high shaded line, but I don't know how to
            % do it.  A little trial and error shows that the following
            % does a reasonable job approximating that, eliminating
            % the line if \pcalvspace is small.
            \addtolength{\@cparht}{\pcalvspace}%
             \ifdim \pcalvspace > .8em
               \addtolength{\pcalvspace}{-.2em}%
               \hspace*{-\lcomindent}%
               \shadebox{\rule{0pt}{\pcalvspace}\hspace*{\textwidth}}\par
               \global\setlength{\pcalvspace}{0pt}%
               \fi
             \hspace*{-\lcomindent}%
             \makebox[0pt][l]{\raisebox{-\@cpardp}[0pt][0pt]{%
                 \shadebox{\rule{0pt}{\@cparht}\hspace*{\textwidth}}}}%
             \hspace*{\lcomindent}\usebox{\tempsbox}%
             \par
           \else
             \shadebox{\usebox{\tempsbox}}\par
           \fi}%
           {}%
  }

%%%%%%%%%%%%%%%%%%%%%%%%%%%%%%%%%%%%%%%%%%%%%%%%%%%%%%%%%%%%%%%%%%%%%%%%%%%%%%
% THE ppar ENVIRONMENT                                                       %
% ^^^^^^^^^^^^^^^^^^^^                                                       %
% The environment                                                            %
%                                                                            %
%   \begin{ppar} ... \end{ppar}                                              %
%                                                                            %
% is equivalent to                                                           %
%                                                                            %
%   \begin{cpar}{0}{F}{F}{0}{0}{} ... \end{cpar}                             %
%                                                                            %
% The environment is put around each line of the output for a PlusCal        %
% algorithm.                                                                 %
%%%%%%%%%%%%%%%%%%%%%%%%%%%%%%%%%%%%%%%%%%%%%%%%%%%%%%%%%%%%%%%%%%%%%%%%%%%%%%
%\newenvironment{ppar}{%
%  \ifthenelse{\boolean{shading}}{\par\begin{lrbox}{\tempsbox}%
%                                 \begin{minipage}[t]{\linewidth}}{}%
%  \begin{list}{}{%
%     \edef\temp{\thegmargin}
%        \setlength{\leftmargin}{0pt}%
%        \addtolength{\leftmargin}{\temp}%
%        \setlength{\itemindent}{0pt}%
%      \setlength{\labelwidth}{0pt}%
%      \setlength{\labelsep}{0pt}%
%      \setlength{\itemindent}{-\leftmargin}%
%      \setlength{\topsep}{0pt}%
%      \setlength{\parsep}{0pt}%
%      \setlength{\partopsep}{0pt}%
%      \setlength{\parskip}{0pt}%
%      \setlength{\itemsep}{0pt}
%      \setlength{\itemindent}{0pt}%
%      \addtolength{\itemindent}{-\leftmargin}}%
%      \item[\tstrut\hspace{\temp}]}%
% {\hspace{-\the\lastskip}\tstrut
% \end{list}%
%  \ifthenelse{\boolean{shading}}{\end{minipage}  
%                                 \end{lrbox}%
%                                 \shadebox{\usebox{\tempsbox}}\par}{}%
%  }

 %%% TESTING
 \newcommand{\xtest}[1]{\par
 \makebox[0pt][l]{\shadebox{\xtstrut\hspace*{\textwidth}}}%
 \mbox{$\mbox{}#1\mbox{}$}} 

% \newcommand{\xxtest}[1]{\par
% \makebox[0pt][l]{\shadebox{\xtstrut{#1}\hspace*{\textwidth}}}%
% \mbox{$\mbox{}#1\mbox{}$}} 

%\newlength{\pcalvspace}
%\setlength{\pcalvspace}{0pt}
% \newlength{\xxtestlen}
% \setlength{\xxtestlen}{0pt}
% \newcommand\xtstrut%
%   {\setlength{\xxtestlen}{1.15em}%
%    \addtolength{\xxtestlen}{\pcalvspace}%
%     \raisebox{\vshadelen}{\raisebox{-.25em}{\rule{0pt}{\xxtestlen}}}%
%    \global\setlength{\vshadelen}{0pt}%
%    \global\setlength{\pcalvspace}{0pt}}
   
   %%%% TESTING
   
   %% The xcpar environment
   %%  Note: overloaded use of \pcalvspace for testing.
   %%
%   \newlength{\xcparht}%
%   \newlength{\xcpardp}%
   
%   \newenvironment{xcpar}[6]{%
%  \addtocounter{pardepth}{-#1}%
%  \ifthenelse{\boolean{shading}}{\par\begin{lrbox}{\tempsbox}%
%                                 \begin{minipage}[t]{\linewidth}}{}%
%  \begin{list}{}{%
%     \edef\temp{\thegmargin}%
%     \ifthenelse{\equal{#3}{T}}%
%       {\settowidth{\leftmargin}{\hspace{\temp}\footnotesize #6\hspace{#5pt}}%
%        \addtolength{\leftmargin}{#4pt}}%
%       {\setlength{\leftmargin}{#4pt}%
%        \addtolength{\leftmargin}{#5pt}%
%        \addtolength{\leftmargin}{\temp}%
%        \setlength{\itemindent}{-#5pt}}%
%      \ifthenelse{\equal{#2}{T}}{\addtocounter{pardepth}{1}%
%                                 \setgmargin{\the\leftmargin}}{}%
%      \setlength{\labelwidth}{0pt}%
%      \setlength{\labelsep}{0pt}%
%      \setlength{\itemindent}{-\leftmargin}%
%      \setlength{\topsep}{0pt}%
%      \setlength{\parsep}{0pt}%
%      \setlength{\partopsep}{0pt}%
%      \setlength{\parskip}{0pt}%
%      \setlength{\itemsep}{0pt}%
%      \setlength{\itemindent}{#4pt}%
%      \addtolength{\itemindent}{-\leftmargin}}%
%   \ifthenelse{\equal{#3}{T}}%
%      {\item[\xtstrut\footnotesize \hspace{\temp}{#6}\hspace{#5pt}]%
%        }%
%      {\item[\xtstrut\hspace{\temp}]%
%         }%
%   \footnotesize}
% {\hspace{-\the\lastskip}\tstrut
% \end{list}%
%  \ifthenelse{\boolean{shading}}{\end{minipage}  
%                                 \end{lrbox}%
%                                 \setlength{\xcparht}{\ht\tempsbox}%
%                                 \setlength{\xcpardp}{\dp\tempsbox}%
%                                 \addtolength{\xcparht}{.15em}%
%                                 \addtolength{\xcpardp}{.2em}%
%                                 \addtolength{\xcparht}{\xcpardp}%
%                                 \hspace*{-\lcomindent}%
%                                 \makebox[0pt][l]{\raisebox{-\xcpardp}[0pt][0pt]{%
%                                      \shadebox{\rule{0pt}{\xcparht}\hspace*{\textwidth}}}}%
%                                 \hspace*{\lcomindent}\usebox{\tempsbox}%
%                                 \par}{}%
%  }
%  
% \newlength{\xmcomlen}
%\newenvironment{xmcom}[1]{%
%  \setcounter{pardepth}{0}%
%  \hspace{.65em}%
%  \begin{lrbox}{\alignbox}\sloppypar%
%      \setboolean{shading}{false}%
%      \setlength{\boxwidth}{#1pt}%
%      \addtolength{\boxwidth}{-.65em}%
%      \begin{minipage}[t]{\boxwidth}\footnotesize
%      \parskip=0pt\relax}%
%       {\end{minipage}\end{lrbox}%
%       \setlength{\xmcomlen}{\textwidth}%
%       \addtolength{\xmcomlen}{-\wd\alignbox}%
%       \settodepth{\alignwidth}{\usebox{\alignbox}}%
%       \global\setlength{\multicommentdepth}{\alignwidth}%
%       \setlength{\boxwidth}{\alignwidth}%
%       \global\addtolength{\alignwidth}{-\maxdepth}%
%       \addtolength{\boxwidth}{.1em}%
%       \raisebox{0pt}[0pt][0pt]{%
%        \ifthenelse{\boolean{shading}}%
%          {\hspace*{-\xmcomlen}\shadebox{\rule[-\boxwidth]{0pt}{0pt}%
%                                 \hspace*{\xmcomlen}\usebox{\alignbox}}}%
%          {\usebox{\alignbox}}}%
%       \vspace*{\alignwidth}\pagebreak[0]\vspace{-\alignwidth}\par}
% % a multi-line comment, whose first argument is its width in points.
%  
   
%%%%%%%%%%%%%%%%%%%%%%%%%%%%%%%%%%%%%%%%%%%%%%%%%%%%%%%%%%%%%%%%%%%%%%%%%%%%%%
% THE lcom ENVIRONMENT                                                       %
% ^^^^^^^^^^^^^^^^^^^^                                                       %
% A multi-line comment with no text to its left is typeset in an lcom        % 
% environment, whose argument is a number representing the indentation       % 
% of the left margin, in points.  All the text of the comment should be      % 
% inside cpar environments.                                                  % 
%%%%%%%%%%%%%%%%%%%%%%%%%%%%%%%%%%%%%%%%%%%%%%%%%%%%%%%%%%%%%%%%%%%%%%%%%%%%%%
\newenvironment{lcom}[1]{%
  \setlength{\lcomindent}{#1pt} % Added for PlusCal handling.
  \par\vspace{.2em}%
  \sloppypar
  \setcounter{pardepth}{0}%
  \footnotesize
  \begin{list}{}{%
    \setlength{\leftmargin}{#1pt}
    \setlength{\labelwidth}{0pt}%
    \setlength{\labelsep}{0pt}%
    \setlength{\itemindent}{0pt}%
    \setlength{\topsep}{0pt}%
    \setlength{\parsep}{0pt}%
    \setlength{\partopsep}{0pt}%
    \setlength{\parskip}{0pt}}
    \item[]}%
  {\end{list}\vspace{.3em}\setlength{\lcomindent}{0pt}%
 }


%%%%%%%%%%%%%%%%%%%%%%%%%%%%%%%%%%%%%%%%%%%%%%%%%%%%%%%%%%%%%%%%%%%%%%%%%%%%%
% THE mcom ENVIRONMENT AND \mutivspace COMMAND                              %
% ^^^^^^^^^^^^^^^^^^^^^^^^^^^^^^^^^^^^^^^^^^^^                              %
%                                                                           %
% A part of the spec containing a right-comment of the form                 %
%                                                                           %
%      xxxx (*************)                                                 %
%      yyyy (* ccccccccc *)                                                 %
%      ...  (* ccccccccc *)                                                 %
%           (* ccccccccc *)                                                 %
%           (* ccccccccc *)                                                 %
%           (*************)                                                 %
%                                                                           %
% is typeset by                                                             %
%                                                                           %
%     XXXX \begin{mcom}{d}                                                  %
%            CCCC ... CCC                                                   %
%          \end{mcom}                                                       %
%     YYYY ...                                                              %
%     \multivspace{n}                                                       %
%                                                                           %
% where the number d is the width in points of the comment, n is the        %
% number of xxxx, yyyy, ...  lines to the left of the comment.              %
% All the text of the comment should be typeset in cpar environments.       %
%                                                                           %
% This puts the comment into a single box (so no page breaks can occur      %
% within it).  The entire box is shaded iff the shading flag is true.       %
%%%%%%%%%%%%%%%%%%%%%%%%%%%%%%%%%%%%%%%%%%%%%%%%%%%%%%%%%%%%%%%%%%%%%%%%%%%%%
\newlength{\xmcomlen}%
\newenvironment{mcom}[1]{%
  \setcounter{pardepth}{0}%
  \hspace{.65em}%
  \begin{lrbox}{\alignbox}\sloppypar%
      \setboolean{shading}{false}%
      \setlength{\boxwidth}{#1pt}%
      \addtolength{\boxwidth}{-.65em}%
      \begin{minipage}[t]{\boxwidth}\footnotesize
      \parskip=0pt\relax}%
       {\end{minipage}\end{lrbox}%
       \setlength{\xmcomlen}{\textwidth}%       % For PlusCal shading
       \addtolength{\xmcomlen}{-\wd\alignbox}%  % For PlusCal shading
       \settodepth{\alignwidth}{\usebox{\alignbox}}%
       \global\setlength{\multicommentdepth}{\alignwidth}%
       \setlength{\boxwidth}{\alignwidth}%      % For PlusCal shading
       \global\addtolength{\alignwidth}{-\maxdepth}%
       \addtolength{\boxwidth}{.1em}%           % For PlusCal shading
      \raisebox{0pt}[0pt][0pt]{%
        \ifthenelse{\boolean{shading}}%
          {\ifpcalshading
             \hspace*{-\xmcomlen}%
             \shadebox{\rule[-\boxwidth]{0pt}{0pt}\hspace*{\xmcomlen}%
                          \usebox{\alignbox}}%
           \else
             \shadebox{\usebox{\alignbox}}
           \fi
          }%
          {\usebox{\alignbox}}}%
       \vspace*{\alignwidth}\pagebreak[0]\vspace{-\alignwidth}\par}
 % a multi-line comment, whose first argument is its width in points.


% \multispace{n} produces the vertical space indicated by "|"s in 
% this situation
%   
%     xxxx (*************)
%     xxxx (* ccccccccc *)
%      |   (* ccccccccc *)
%      |   (* ccccccccc *)
%      |   (* ccccccccc *)
%      |   (*************)
%
% where n is the number of "xxxx" lines.
\newcommand{\multivspace}[1]{\addtolength{\multicommentdepth}{-#1\baselineskip}%
 \addtolength{\multicommentdepth}{1.2em}%
 \ifthenelse{\lengthtest{\multicommentdepth > 0pt}}%
    {\par\vspace{\multicommentdepth}\par}{}}

%\newenvironment{hpar}[2]{%
%  \begin{list}{}{\setlength{\leftmargin}{#1pt}%
%                 \addtolength{\leftmargin}{#2pt}%
%                 \setlength{\itemindent}{-#2pt}%
%                 \setlength{\topsep}{0pt}%
%                 \setlength{\parsep}{0pt}%
%                 \setlength{\partopsep}{0pt}%
%                 \setlength{\parskip}{0pt}%
%                 \addtolength{\labelsep}{0pt}}%
%  \item[]\footnotesize}{\end{list}}
%    %%%%%%%%%%%%%%%%%%%%%%%%%%%%%%%%%%%%%%%%%%%%%%%%%%%%%%%%%%%%%%%%%%%%%%%%
%    % Typesets a sequence of paragraphs like this:                         %
%    %                                                                      %
%    %      left |<-- d1 --> XXXXXXXXXXXXXXXXXXXXXXXX                       %
%    %    margin |           <- d2 -> XXXXXXXXXXXXXXX                       %
%    %           |                    XXXXXXXXXXXXXXX                       %
%    %           |                                                          %
%    %           |                    XXXXXXXXXXXXXXX                       %
%    %           |                    XXXXXXXXXXXXXXX                       %
%    %                                                                      %
%    % where d1 = #1pt and d2 = #2pt, but with no vspace between            %
%    % paragraphs.                                                          %
%    %%%%%%%%%%%%%%%%%%%%%%%%%%%%%%%%%%%%%%%%%%%%%%%%%%%%%%%%%%%%%%%%%%%%%%%%

%%%%%%%%%%%%%%%%%%%%%%%%%%%%%%%%%%%%%%%%%%%%%%%%%%%%%%%%%%%%%%%%%%%%%%
% Commands for repeated characters that produce dashes.              %
%%%%%%%%%%%%%%%%%%%%%%%%%%%%%%%%%%%%%%%%%%%%%%%%%%%%%%%%%%%%%%%%%%%%%%
% \raisedDash{wd}{ht}{thk} makes a horizontal line wd characters wide, 
% raised a distance ht ex's above the baseline, with a thickness of 
% thk em's.
\newcommand{\raisedDash}[3]{\raisebox{#2ex}{\setlength{\alignwidth}{.5em}%
  \rule{#1\alignwidth}{#3em}}}

% The following commands take a single argument n and produce the
% output for n repeated characters, as follows
%   \cdash:    -
%   \tdash:    ~
%   \ceqdash:  =
%   \usdash:   _
\newcommand{\cdash}[1]{\raisedDash{#1}{.5}{.04}}
\newcommand{\usdash}[1]{\raisedDash{#1}{0}{.04}}
\newcommand{\ceqdash}[1]{\raisedDash{#1}{.5}{.08}}
\newcommand{\tdash}[1]{\raisedDash{#1}{1}{.08}}

\newlength{\spacewidth}
\setlength{\spacewidth}{.2em}
\newcommand{\e}[1]{\hspace{#1\spacewidth}}
%% \e{i} produces space corresponding to i input spaces.


%% Alignment-file Commands

\newlength{\alignboxwidth}
\newlength{\alignwidth}
\newsavebox{\alignbox}

% \al{i}{j}{txt} is used in the alignment file to put "%{i}{j}{wd}"
% in the log file, where wd is the width of the line up to that point,
% and txt is the following text.
\newcommand{\al}[3]{%
  \typeout{\%{#1}{#2}{\the\alignwidth}}%
  \cl{#3}}

%% \cl{txt} continues a specification line in the alignment file
%% with text txt.
\newcommand{\cl}[1]{%
  \savebox{\alignbox}{\mbox{$\mbox{}#1\mbox{}$}}%
  \settowidth{\alignboxwidth}{\usebox{\alignbox}}%
  \addtolength{\alignwidth}{\alignboxwidth}%
  \usebox{\alignbox}}

% \fl{txt} in the alignment file begins a specification line that
% starts with the text txt.
\newcommand{\fl}[1]{%
  \par
  \savebox{\alignbox}{\mbox{$\mbox{}#1\mbox{}$}}%
  \settowidth{\alignwidth}{\usebox{\alignbox}}%
  \usebox{\alignbox}}



  
%%%%%%%%%%%%%%%%%%%%%%%%%%%%%%%%%%%%%%%%%%%%%%%%%%%%%%%%%%%%%%%%%%%%%%%%%%%%%
% Ordinarily, TeX typesets letters in math mode in a special math italic    %
% font.  This makes it typeset "it" to look like the product of the         %
% variables i and t, rather than like the word "it".  The following         %
% commands tell TeX to use an ordinary italic font instead.                 %
%%%%%%%%%%%%%%%%%%%%%%%%%%%%%%%%%%%%%%%%%%%%%%%%%%%%%%%%%%%%%%%%%%%%%%%%%%%%%
\ifx\documentclass\undefined
\else
  \DeclareSymbolFont{tlaitalics}{\encodingdefault}{cmr}{m}{it}
  \let\itfam\symtlaitalics
\fi

\makeatletter
\newcommand{\tlx@c}{\c@tlx@ctr\advance\c@tlx@ctr\@ne}
\newcounter{tlx@ctr}
\c@tlx@ctr=\itfam \multiply\c@tlx@ctr"100\relax \advance\c@tlx@ctr "7061\relax
\mathcode`a=\tlx@c \mathcode`b=\tlx@c \mathcode`c=\tlx@c \mathcode`d=\tlx@c
\mathcode`e=\tlx@c \mathcode`f=\tlx@c \mathcode`g=\tlx@c \mathcode`h=\tlx@c
\mathcode`i=\tlx@c \mathcode`j=\tlx@c \mathcode`k=\tlx@c \mathcode`l=\tlx@c
\mathcode`m=\tlx@c \mathcode`n=\tlx@c \mathcode`o=\tlx@c \mathcode`p=\tlx@c
\mathcode`q=\tlx@c \mathcode`r=\tlx@c \mathcode`s=\tlx@c \mathcode`t=\tlx@c
\mathcode`u=\tlx@c \mathcode`v=\tlx@c \mathcode`w=\tlx@c \mathcode`x=\tlx@c
\mathcode`y=\tlx@c \mathcode`z=\tlx@c
\c@tlx@ctr=\itfam \multiply\c@tlx@ctr"100\relax \advance\c@tlx@ctr "7041\relax
\mathcode`A=\tlx@c \mathcode`B=\tlx@c \mathcode`C=\tlx@c \mathcode`D=\tlx@c
\mathcode`E=\tlx@c \mathcode`F=\tlx@c \mathcode`G=\tlx@c \mathcode`H=\tlx@c
\mathcode`I=\tlx@c \mathcode`J=\tlx@c \mathcode`K=\tlx@c \mathcode`L=\tlx@c
\mathcode`M=\tlx@c \mathcode`N=\tlx@c \mathcode`O=\tlx@c \mathcode`P=\tlx@c
\mathcode`Q=\tlx@c \mathcode`R=\tlx@c \mathcode`S=\tlx@c \mathcode`T=\tlx@c
\mathcode`U=\tlx@c \mathcode`V=\tlx@c \mathcode`W=\tlx@c \mathcode`X=\tlx@c
\mathcode`Y=\tlx@c \mathcode`Z=\tlx@c
\makeatother

%%%%%%%%%%%%%%%%%%%%%%%%%%%%%%%%%%%%%%%%%%%%%%%%%%%%%%%%%%
%                THE describe ENVIRONMENT                %
%%%%%%%%%%%%%%%%%%%%%%%%%%%%%%%%%%%%%%%%%%%%%%%%%%%%%%%%%%
%
%
% It is like the description environment except it takes an argument
% ARG that should be the text of the widest label.  It adjusts the
% indentation so each item with label LABEL produces
%%      LABEL             blah blah blah
%%      <- width of ARG ->blah blah blah
%%                        blah blah blah
\newenvironment{describe}[1]%
   {\begin{list}{}{\settowidth{\labelwidth}{#1}%
            \setlength{\labelsep}{.5em}%
            \setlength{\leftmargin}{\labelwidth}% 
            \addtolength{\leftmargin}{\labelsep}%
            \addtolength{\leftmargin}{\parindent}%
            \def\makelabel##1{\rm ##1\hfill}}%
            \setlength{\topsep}{0pt}}%% 
                % Sets \topsep to 0 to reduce vertical space above
                % and below embedded displayed equations
   {\end{list}}

%   For tlatex.TeX
\usepackage{verbatim}
\makeatletter
\def\tla{\let\%\relax%
         \@bsphack
         \typeout{\%{\the\linewidth}}%
             \let\do\@makeother\dospecials\catcode`\^^M\active
             \let\verbatim@startline\relax
             \let\verbatim@addtoline\@gobble
             \let\verbatim@processline\relax
             \let\verbatim@finish\relax
             \verbatim@}
\let\endtla=\@esphack

\let\pcal=\tla
\let\endpcal=\endtla
\let\ppcal=\tla
\let\endppcal=\endtla

% The tlatex environment is used by TLATeX.TeX to typeset TLA+.
% TLATeX.TLA starts its files by writing a \tlatex command.  This
% command/environment sets \parindent to 0 and defines \% to its
% standard definition because the writing of the log files is messed up
% if \% is defined to be something else.  It also executes
% \@computerule to determine the dimensions for the TLA horizonatl
% bars.
\newenvironment{tlatex}{\@computerule%
                        \setlength{\parindent}{0pt}%
                       \makeatletter\chardef\%=`\%}{}


% The notla environment produces no output.  You can turn a 
% tla environment to a notla environment to prevent tlatex.TeX from
% re-formatting the environment.

\def\notla{\let\%\relax%
         \@bsphack
             \let\do\@makeother\dospecials\catcode`\^^M\active
             \let\verbatim@startline\relax
             \let\verbatim@addtoline\@gobble
             \let\verbatim@processline\relax
             \let\verbatim@finish\relax
             \verbatim@}
\let\endnotla=\@esphack

\let\nopcal=\notla
\let\endnopcal=\endnotla
\let\noppcal=\notla
\let\endnoppcal=\endnotla

%%%%%%%%%%%%%%%%%%%%%%%% end of tlatex.sty file %%%%%%%%%%%%%%%%%%%%%%% 
% last modified on Fri  3 August 2012 at 14:23:49 PST by lamport

\begin{document}
\tlatex
\setboolean{shading}{true}
 \@x{}\moduleLeftDash\@xx{ {\MODULE}
 AutoGraceLapseInvoiceMch0}\moduleRightDash\@xx{}%
\@x{ {\EXTENDS} AutoGraceLapseInvoiceCtx0 ,\, FiniteSets ,\, TLC}%
\@x{ {\CONSTANTS} endTime ,\, graceDelta}%
\@pvspace{8.0pt}%
\@x{ {\VARIABLES} invsIssued ,\, policy ,\, gracesUnsettled ,\, time}%
\@pvspace{8.0pt}%
\begin{lcom}{7.5}%
\begin{cpar}{0}{F}{F}{0}{0}{}%
 Here we model a simple yearly policy to illustrate the interactions between
 policy
 modifications, the scheduled grace/lapse jobs, payments, and payment
 reversal. Since,
 with yearly policies, all invoices are generated at the time of
 modification, we have
 simplified away the complicating factor that scheduled invoicing can, in the
 most general
 case, generate outstanding invoices at any time. Therefore this model is a
 simplest case
 analysis.
\end{cpar}%
\end{lcom}%
\@x{}\midbar\@xx{}%
\@x{}%
\@y{\@s{0}%
 The possible combinations of settlement status and invoice type
}%
\@xx{}%
\@x{ InvInvoices \.{\defeq}}%
\@x{\@s{16.4} \.{\land} \forall\, i \.{\in} invsIssued \.{:}}%
 \@x{\@s{48.01} \.{\land} i . settlementStatus \.{=}\@w{Outstanding}
 \.{\implies} i . settlementType \.{=}\@w{Null}}%
 \@x{\@s{48.01} \.{\land} i . settlementStatus \.{=}\@w{Settled} \.{\implies}
 i . settlementType \.{\in} \{}%
 \@x{\@s{78.28}\@w{ZeroDue} ,\,\@w{WrittenOff} ,\,\@w{Invalidated}
 ,\,\@w{Paid}}%
\@x{\@s{63.20} \}}%
\@pvspace{8.0pt}%
\begin{lcom}{7.5}%
\begin{cpar}{0}{F}{F}{0}{0}{}%
 At the time a policy is fully cancelled we insist that all invoices be
 settled. We
 can not establish that as an invariant however as with payment reversals
 there are no
 restrictions on what invoices can be subsequently flipped into the
 outstanding state.
\end{cpar}%
\end{lcom}%
\@x{ InvPolicy \.{\defeq} policy . startDt \.{\leq} policy . endDt}%
\@pvspace{8.0pt}%
\begin{lcom}{7.5}%
\begin{cpar}{0}{F}{F}{0}{0}{}%
 There must be at most one unsetteled grace record per policy. This property
 is due to
 the data definition of grace. A policy is defined as in grace when it has a
 record
 in the grace table that is unsettled.
\end{cpar}%
\end{lcom}%
\@x{ InvGraceRecs \.{\defeq}}%
\@x{\@s{16.4} \.{\land} Cardinality ( gracesUnsettled ) \.{\leq} 1}%
 \@x{\@s{16.4} \.{\land} \forall\, g \.{\in} gracesUnsettled \.{:} g \.{\in}
 Grace}%
\@pvspace{8.0pt}%
\begin{lcom}{7.5}%
\begin{cpar}{0}{F}{F}{0}{0}{}%
 If any outstanding policy invoices exist, then there must be a unsettled
 grace record
 for the policy. If there is an unsettled grace record then there must be an
 outstanding policy invoice.
\end{cpar}%
\end{lcom}%
\@x{ InvInvoicesGraceRecs \.{\defeq}}%
 \@x{\@s{16.4} \.{\land} \{ i \.{\in} invsIssued \.{:} \.{\land} i .
 settlementStatus \.{=}\@w{Outstanding}}%
 \@x{\@s{103.27} \.{\land} time \.{>} i . dueDt \} \.{\neq} \{ \} \.{\implies}
 gracesUnsettled \.{\neq} \{ \}}%
 \@x{\@s{16.4} \.{\land} gracesUnsettled \.{\neq} \{ \} \.{\implies} \exists\,
 i \.{\in} invsIssued \.{:}}%
\@x{\@s{150.17} \.{\land} i . settlementStatus \.{=}\@w{Outstanding}}%
\@x{\@s{150.17} \.{\land} i . dueDt \.{\leq} time}%
\@pvspace{8.0pt}%
\@x{}\midbar\@xx{}%
\begin{lcom}{7.5}%
\begin{cpar}{0}{F}{F}{0}{0}{}%
 The scheduled grace job tries to find policies that (1) dont have any
 associated,
 unsettled grace records in the \ensuremath{grace\_period} table and (2) do
 have outstanding invoices
 that are past their due date. If it finds such records it adds them to the
 \ensuremath{grace\_period} table, which roughly can be thought of as a queue
 of jobs which may or may
 not get to the top of the queueu and cause a lapse (cancellation)
\end{cpar}%
\end{lcom}%
\@x{ autoGraceCanRun ( tm ) \.{\defeq}}%
\@x{\@s{16.4} \.{\land} gracesUnsettled \.{=} \{ \}}%
 \@x{\@s{16.4} \.{\land} \exists\, i \.{\in} invsIssued \.{:} i . dueDt \.{=}
 tm \.{\land} i . settlementStatus \.{=}\@w{Outstanding}}%
\@pvspace{8.0pt}%
\begin{lcom}{7.5}%
\begin{cpar}{0}{F}{F}{0}{0}{}%
 The scheduled lapse job looks for unsettled records in the
 \ensuremath{grace\_period} period and
 cancels the associated policy if the end of the grace period
 \ensuremath{\.{<}} now.
\end{cpar}%
\end{lcom}%
 \@x{ autoLapseCanRun ( tm ) \.{\defeq} \exists\, g \.{\in} gracesUnsettled
 \.{:} g . endDt \.{=} tm}%
\@pvspace{8.0pt}%
\@x{ autoJobCanRun ( tm ) \.{\defeq}}%
\@x{\@s{16.4} \.{\lor} autoGraceCanRun ( tm )}%
\@x{\@s{16.4} \.{\lor} autoLapseCanRun ( tm )}%
\@pvspace{8.0pt}%
\@x{}%
\@y{\@s{0}%
 the definition of containment for a point within a closed - open interval.
}%
\@xx{}%
 \@x{ contains ( interval ,\, t )\@s{5.78} \.{\defeq} interval . startDt
 \.{\leq} t \.{\land} t \.{<} interval . endDt}%
\@pvspace{8.0pt}%
\@x{ overdueInvoices ( invs ) \.{\defeq} \{ i \.{\in} invs \.{:}}%
\@x{\@s{32.8} \.{\land} i . settlementStatus \.{=}\@w{Outstanding}}%
\@x{\@s{32.8} \.{\land} i . dueDt \.{\leq} time \}}%
\@pvspace{8.0pt}%
 \@x{ minOf ( S ) \.{\defeq} {\CHOOSE} x \.{\in} S \.{:} \A\, y \.{\in} S
 \.{:} x \.{\leq} y}%
\@x{ oneOf ( S )\@s{1.53} \.{\defeq} {\CHOOSE} x \.{\in} S \.{:} {\TRUE}}%
 \@x{ max ( t1 ,\, t2 ) \.{\defeq} {\IF} t1\@s{15.85} \.{>} t2 \.{\THEN} t1
 \.{\ELSE} t2}%
\@pvspace{8.0pt}%
\@x{}\midbar\@xx{}%
\@x{}%
\@y{\@s{0}%
 Extension is a generic operation extends a policy range, a renewal or
 reinstatement.
}%
\@xx{}%
\@x{ extend ( m ) \.{\defeq}}%
\@x{\@s{16.4} \.{\land} {\lnot} autoJobCanRun ( time )}%
\@x{\@s{16.4} \.{\land} time \.{<} endTime}%
\@x{\@s{16.4} \.{\land} m . startDt \.{=} policy . endDt}%
\@x{\@s{16.4} \.{\land} m . startDt \.{<} m . endDt}%
 \@x{\@s{16.4} \.{\land} policy \.{'} \.{=} [ policy {\EXCEPT} {\bang} . endDt
 \.{=} m . endDt ]}%
\@x{\@s{16.4} \.{\land} invsIssued \.{'} \.{=} invsIssued \.{\cup} \{ [}%
\@x{\@s{48.01} uid \.{\mapsto} time \.{+} 1 ,\,}%
\@x{\@s{48.01} totalDue \.{\mapsto} m . endDt \.{-} m . startDt ,\,}%
\@x{\@s{48.01} startDt \.{\mapsto} m . startDt ,\,}%
\@x{\@s{48.01} endDt \.{\mapsto} m . endDt ,\,}%
 \@x{\@s{48.01} dueDt\@s{0.25} \.{\mapsto} max ( m . startDt \.{+} 1 ,\, time
 \.{+} 1 ) ,\,}%
\@x{\@s{48.01} settlementStatus \.{\mapsto}\@w{Outstanding} ,\,}%
\@x{\@s{48.01} settlementType \.{\mapsto}\@w{Null}}%
\@x{\@s{31.61} ] \}}%
\@x{\@s{16.4} \.{\land} time \.{'} \.{=} time \.{+} 1}%
\@x{\@s{16.4} \.{\land} {\UNCHANGED} {\langle} gracesUnsettled {\rangle}}%
\@pvspace{8.0pt}%
\begin{lcom}{7.5}%
\begin{cpar}{0}{F}{F}{0}{0}{}%
Reduce is a generic operation that reduces a policy range, a cancellation.
 For the reduction I dont add any negative invoices as negative invoices dont
 have a
 grace date and so are not picked up by the automated jobs, which I am
 modeling here.
\end{cpar}%
\end{lcom}%
\@x{ reduce ( m ) \.{\defeq}}%
\@x{\@s{16.4} \.{\land} {\lnot} autoJobCanRun ( time )}%
\@x{\@s{16.4} \.{\land} time \.{<} endTime}%
\@x{\@s{16.4} \.{\land} m . startDt \.{\geq} policy . startDt}%
\@x{\@s{16.4} \.{\land} m . endDt \.{=} policy . endDt}%
\@x{\@s{16.4} \.{\land} m . startDt \.{<} m . endDt}%
 \@x{\@s{16.4} \.{\land} policy \.{'} \.{=} [ policy {\EXCEPT} {\bang} . endDt
 \.{=} m . startDt ]}%
\@x{\@s{16.4} \.{\land} time \.{'} \.{=} time \.{+} 1}%
 \@x{\@s{16.4} \.{\land} {\UNCHANGED} {\langle} invsIssued ,\, gracesUnsettled
 {\rangle}}%
\@pvspace{8.0pt}%
\@x{}%
\@y{\@s{0}%
 Change is a generic operations that updates policy coverage, an endorsement.
}%
\@xx{}%
\@x{ change ( m ) \.{\defeq}}%
\@x{\@s{16.4} \.{\land} {\lnot} autoJobCanRun ( time )}%
\@x{\@s{16.4} \.{\land} time \.{<} endTime}%
 \@x{\@s{16.4} \.{\land} policy . startDt \.{\leq} m . startDt \.{\land} m .
 startDt \.{<} policy . endDt}%
\@x{\@s{16.4} \.{\land} m . endDt \.{=} policy . endDt}%
\@x{\@s{16.4} \.{\land} m . startDt \.{<} m . endDt}%
\@x{\@s{16.4} \.{\land} invsIssued \.{'} \.{=} invsIssued \.{\cup} \{ [}%
\@x{\@s{48.01} uid \.{\mapsto} time \.{+} 1 ,\,}%
\@x{\@s{48.01} totalDue \.{\mapsto} m . endDt \.{-} m . startDt ,\,}%
\@x{\@s{48.01} startDt \.{\mapsto} m . startDt ,\,}%
\@x{\@s{48.01} endDt \.{\mapsto} m . endDt ,\,}%
 \@x{\@s{48.01} dueDt\@s{0.25} \.{\mapsto} max ( m . startDt \.{+} 1 ,\, time
 \.{+} 1 ) ,\,}%
\@x{\@s{48.01} settlementStatus \.{\mapsto}\@w{Outstanding} ,\,}%
\@x{\@s{48.01} settlementType \.{\mapsto}\@w{Null}}%
\@x{\@s{31.61} ] \}}%
\@x{\@s{16.4} \.{\land} time \.{'} \.{=} time \.{+} 1}%
 \@x{\@s{16.4} \.{\land} {\UNCHANGED} {\langle} policy ,\, gracesUnsettled
 {\rangle}}%
\@pvspace{8.0pt}%
\begin{lcom}{7.5}%
\begin{cpar}{0}{F}{F}{0}{0}{}%
 We have discussed grace processing above. For emphasis here, note that the
 guard clause
 below only adds a grace record if there are no unsettled records associated
 with the
 policy.
\end{cpar}%
\end{lcom}%
\@x{ doAutoGrace ( inv ) \.{\defeq}}%
\@x{\@s{16.4} \.{\land} time \.{<} endTime}%
\@x{\@s{16.4} \.{\land} gracesUnsettled \.{=} \{ \}}%
 \@x{\@s{16.4} \.{\land} inv . settlementStatus \.{=}\@w{Outstanding}
 \.{\land} inv . dueDt \.{=} time}%
\@x{\@s{16.4} \.{\land} gracesUnsettled \.{'} \.{=} \{ [}%
\@x{\@s{48.01} invUid \.{\mapsto} inv . uid ,\,}%
\@x{\@s{48.01} startDt \.{\mapsto} inv . dueDt ,\,}%
\@x{\@s{48.01} endDt \.{\mapsto} inv . dueDt \.{+} graceDelta}%
\@x{\@s{31.61} ] \}}%
\@x{\@s{16.4} \.{\land} time \.{'} \.{=} time \.{+} 1}%
 \@x{\@s{16.4} \.{\land} {\UNCHANGED} {\langle} policy ,\, invsIssued
 {\rangle}}%
\@pvspace{8.0pt}%
\begin{lcom}{7.5}%
\begin{cpar}{0}{F}{F}{0}{0}{}%
 Auto lapse will cancel the whole policy and that requires settling all the
 invoices,
 not just the particular invoice, associated with the grace record. This
 prevents
 multiple loops through grace/lapse and also established the grace invariant.
 In the
 implementation the invoice cancellation details are handled in the
 \ensuremath{CancellationsService}.
\end{cpar}%
\vshade{5.0}%
\begin{cpar}{0}{F}{F}{0}{0}{}%
 In the second branch of the \ensuremath{{\IF}} statement below we handle the
 case where a policy is to
 lapse after the policy has ended. We really cant cancel the policy as
 cancellation
 is a reduction in policy extent. So for this case we just settle all
 outstanding
 invoices and clear grace. Clearing all the outstanding invoices helps us
 maintain the
 processing invariant that \ensuremath{gracesUnsettled \.{=} \{\}
 \.{\implies}} there are no outstanding invoices.
\end{cpar}%
\end{lcom}%
\@x{ doAutoLapse ( g ) \.{\defeq}}%
\@x{\@s{16.4} \.{\land} time \.{<} endTime}%
\@x{\@s{16.4} \.{\land} g . endDt \.{=} time}%
\@x{\@s{16.4} \.{\land} {\IF} g . endDt \.{<} policy . endDt}%
\@x{\@s{27.51}}%
\@y{\@s{0}%
 \ensuremath{invalidates \.{=}} invoices written off if grace/lapsed looped
 multiple times at \ensuremath{t
}}%
\@xx{}%
 \@x{\@s{27.51} \.{\THEN} \.{\LET} invalidates \.{\defeq} \{ i \.{\in}
 invsIssued \.{:} \.{\land} i . settlementStatus \.{=}\@w{Outstanding}}%
\@x{\@s{220.94} \.{\land} ( \.{\lor} i . uid \.{=} g . invUid}%
\@x{\@s{236.16} \.{\lor} i . dueDt \.{+} graceDelta \.{\leq} time ) \}}%
 \@x{\@s{79.22} overdues \.{\defeq} \{ i \.{\in} invsIssued \.{:} \.{\land} i
 . settlementStatus \.{=}\@w{Outstanding}}%
\@x{\@s{211.87} \.{\land} i . dueDt \.{\leq} time}%
\@x{\@s{211.87} \.{\land} i . dueDt \.{+} graceDelta \.{>} time \}}%
 \@x{\@s{58.82} \.{\IN} \.{\land} policy \.{'} \.{=} [ policy {\EXCEPT}
 {\bang} . endDt \.{=} g . endDt ]}%
 \@x{\@s{79.22} \.{\land} invsIssued \.{'} \.{=} ( invsIssued
 \.{\,\backslash\,} invalidates )}%
\@x{\@s{151.48} \.{\cup}}%
 \@x{\@s{151.48} \{ [ i {\EXCEPT} {\bang} . settlementStatus \.{=}\@w{Settled}
 ,\,}%
 \@x{\@s{205.36} {\bang} . settlementType \.{=}\@w{WrittenOff} ] \.{:} i
 \.{\in} invalidates \}}%
 \@x{\@s{79.22} \.{\land} gracesUnsettled \.{'} \.{=} {\IF} overdues \.{=} \{
 \}}%
\@x{\@s{98.53} \.{\THEN} \{ \}}%
\@x{\@s{98.53} \.{\ELSE} \.{\LET} anOverdue \.{\defeq} oneOf ( overdues )}%
\@x{\@s{129.84} \.{\IN} \{ [}%
\@x{\@s{167.54} invUid \.{\mapsto} anOverdue . uid ,\,}%
\@x{\@s{167.54} startDt \.{\mapsto} anOverdue . dueDt ,\,}%
\@x{\@s{167.54} endDt \.{\mapsto} anOverdue . dueDt \.{+} graceDelta}%
\@x{\@s{150.24} ] \}}%
\@x{\@s{79.22} \.{\land} time \.{'} \.{=} time \.{+} 1}%
 \@x{\@s{27.51} \.{\ELSE} \.{\LET} invalidates \.{\defeq} \{ i \.{\in}
 invsIssued \.{:} i . settlementStatus \.{=}\@w{Outstanding} \}}%
 \@x{\@s{58.82} \.{\IN} \.{\land} Assert ( invalidates \.{\neq} \{ \}
 ,\,\@w{bad\ empty\ invalidates\ set} )}%
 \@x{\@s{79.22} \.{\land} invsIssued \.{'} \.{=} ( invsIssued
 \.{\,\backslash\,} invalidates )}%
\@x{\@s{151.48} \.{\cup}}%
 \@x{\@s{151.48} \{ [ i {\EXCEPT} {\bang} . settlementStatus \.{=}\@w{Settled}
 ,\,}%
 \@x{\@s{205.36} {\bang} . settlementType \.{=}\@w{WrittenOff} ] \.{:} i
 \.{\in} invalidates \}}%
\@x{\@s{79.22} \.{\land} gracesUnsettled \.{'} \.{=} \{ \}}%
\@x{\@s{79.22} \.{\land} time \.{'} \.{=} time \.{+} 1}%
\@x{\@s{79.22} \.{\land} {\UNCHANGED} {\langle} policy {\rangle}}%
\@pvspace{8.0pt}%
\begin{lcom}{7.5}%
\begin{cpar}{0}{F}{F}{0}{0}{}%
 When payments come in, the framework checks to see if the policy is in grace.
 If it is
 and the current payment will settle all outstandiing invoices, then the
 payment routines
 can also settle the grace record, bringing the policy out of grace. This
 processing
 rule tends to be hard to understand for engineers, so remember paying one
 invoice
 generally does not get a policy out of grace, all outstanding invoices have
 to be
 paid.
\end{cpar}%
\end{lcom}%
\@x{ doPayment ( inv ) \.{\defeq}}%
\@x{\@s{16.4} \.{\land} {\lnot} autoJobCanRun ( time )}%
\@x{\@s{16.4} \.{\land} time \.{<} endTime}%
\@x{\@s{16.4} \.{\land} inv . settlementStatus \.{=}\@w{Outstanding}}%
 \@x{\@s{16.4} \.{\land} invsIssued \.{'} \.{=} ( invsIssued
 \.{\,\backslash\,} \{ inv \} ) \.{\cup} \{}%
 \@x{\@s{48.01} [ inv {\EXCEPT} {\bang} . settlementStatus \.{=}\@w{Settled}
 ,\, {\bang} . settlementType \.{=}\@w{Paid} ]}%
\@x{\@s{48.01} \}}%
\@x{\@s{16.4} \.{\land} time \.{'} \.{=} time \.{+} 1}%
 \@x{\@s{16.4} \.{\land} \.{\LET} overdueInvs \.{\defeq} overdueInvoices (
 invsIssued \.{\,\backslash\,} \{ inv \} )}%
 \@x{\@s{27.51} \.{\IN} {\CASE} overdueInvs \.{=} \{ \} \.{\land}
 gracesUnsettled \.{=} \{ \} \.{\rightarrow}}%
\@x{\@s{77.98} {\UNCHANGED} {\langle} policy ,\, gracesUnsettled {\rangle}}%
 \@x{\@s{56.11} {\Box}\@s{10.30} overdueInvs \.{=} \{ \} \.{\land}
 gracesUnsettled \.{\neq} \{ \} \.{\rightarrow}}%
\@x{\@s{77.98} \.{\land} gracesUnsettled \.{'} \.{=} \{ \}}%
\@x{\@s{77.98} \.{\land} {\UNCHANGED} policy}%
\@x{\@s{56.11} {\Box}\@s{10.30} overdueInvs \.{\neq} \{ \} \.{\rightarrow}}%
\@x{\@s{77.98} {\UNCHANGED} {\langle} policy ,\, gracesUnsettled {\rangle}}%
\@pvspace{8.0pt}%
\begin{lcom}{7.5}%
\begin{cpar}{0}{F}{F}{0}{0}{}%
 When we reverse payment we want to (1) reset the due date, and grace date of
 the
 affected invoice. This prevents reversals of older invoices from immediately
 causing
 the policy to go into grace and immediately after be cancelled. This is a
 side effect
 that customers will find hard to understand. Presumably if a customer has to
 pay some
 time must be given to them to do so before cancelling their policy. (2) We
 dont want to
 reverse payments on policies that are fully cancelled. To reinterate a
 similiar
 thought above, it does not make sense in that a customer would have to
 suddenly owe
 money on a previous invoice and yet the coverage that invoice is based on is
 not being
 offered to them. Customer but also the code will expect this convention so
 if it is
 violated the automated lapse code will refuse to cooperate and someone,
 maybe you,
 will get a late night call.
\end{cpar}%
\end{lcom}%
\@x{ doRevPayment ( inv ) \.{\defeq}}%
\@x{\@s{16.4} \.{\land} {\lnot} autoJobCanRun ( time )}%
\@x{\@s{16.4} \.{\land} time \.{<} endTime}%
\@x{\@s{16.4} \.{\land} inv . settlementStatus \.{=}\@w{Settled}}%
 \@x{\@s{16.4} \.{\land} invsIssued \.{'} \.{=} ( invsIssued
 \.{\,\backslash\,} \{ inv \} ) \.{\cup} \{}%
 \@x{\@s{48.01} [ inv {\EXCEPT} {\bang} . settlementStatus
 \.{=}\@w{Outstanding} ,\,}%
\@x{\@s{107.17} {\bang} . settlementType \.{=}\@w{Null} ,\,}%
\@x{\@s{107.17} {\bang} . dueDt \.{=} max ( inv . dueDt ,\, time \.{+} 1 ) ]}%
\@x{\@s{48.01} \}}%
\@x{\@s{16.4} \.{\land} time \.{'} \.{=} time \.{+} 1}%
 \@x{\@s{16.4} \.{\land} {\UNCHANGED} {\langle} policy ,\, gracesUnsettled
 {\rangle}}%
\@pvspace{16.0pt}%
\@x{ tick \.{\defeq}}%
\@x{\@s{16.4} \.{\land} {\lnot} autoJobCanRun ( time )}%
\@x{\@s{16.4} \.{\land} time \.{<} endTime}%
\@x{\@s{16.4} \.{\land} time \.{'} \.{=} time \.{+} 1}%
 \@x{\@s{16.4} \.{\land} {\UNCHANGED} {\langle} invsIssued ,\, policy ,\,
 gracesUnsettled {\rangle}}%
\@pvspace{8.0pt}%
\@x{ term \.{\defeq}}%
\@x{\@s{16.4} \.{\land} time \.{=} endTime}%
 \@x{\@s{16.4} \.{\land} {\UNCHANGED} {\langle} invsIssued ,\, policy ,\,
 gracesUnsettled ,\, time {\rangle}}%
\@pvspace{8.0pt}%
\@x{ Init \.{\defeq}}%
\@x{\@s{16.4} \.{\land} \exists\, p \.{\in} Policy \.{:}}%
\@x{\@s{48.01} \.{\land} p . startDt \.{<} p . endDt}%
\@x{\@s{48.01} \.{\land} policy \.{=} p}%
\@x{\@s{16.4} \.{\land} time \.{=} 0}%
\@x{\@s{16.4} \.{\land} invsIssued \.{=} \{ [}%
\@x{\@s{48.01} uid \.{\mapsto} 0 ,\,}%
 \@x{\@s{48.01} totalDue \.{\mapsto} policy . endDt \.{-} policy . startDt
 ,\,}%
\@x{\@s{48.01} startDt \.{\mapsto} policy . startDt ,\,}%
\@x{\@s{48.01} endDt \.{\mapsto} policy . endDt ,\,}%
\@x{\@s{48.01} dueDt\@s{0.25} \.{\mapsto} policy . startDt \.{+} 1 ,\,}%
\@x{\@s{48.01} settlementStatus \.{\mapsto}\@w{Outstanding} ,\,}%
\@x{\@s{48.01} settlementType \.{\mapsto}\@w{Null}}%
\@x{\@s{31.61} ] \}}%
\@x{\@s{16.4} \.{\land} gracesUnsettled \.{=} \{ \}}%
\@pvspace{8.0pt}%
\@x{ Next \.{\defeq}}%
\@x{\@s{16.4} \.{\lor} \exists\, m \.{\in} Modification \.{:}}%
\@x{\@s{48.01} \.{\lor} change ( m )}%
\@x{\@s{48.01} \.{\lor} extend ( m )}%
\@x{\@s{48.01} \.{\lor} reduce ( m )}%
\@x{\@s{16.4} \.{\lor} \exists\, inv \.{\in} invsIssued \.{:}}%
\@x{\@s{48.01} \.{\lor} doAutoGrace ( inv )}%
\@x{\@s{48.01} \.{\lor} doPayment ( inv )}%
\@x{\@s{48.01} \.{\lor} doRevPayment ( inv )}%
 \@x{\@s{16.4} \.{\lor} \exists\, g \.{\in} gracesUnsettled \.{:} doAutoLapse
 ( g )}%
\@x{\@s{16.4} \.{\lor} term}%
\@pvspace{8.0pt}%
 \@x{ Spec \.{\defeq} Init \.{\land} {\Box} [ Next ]_{ {\langle} invsIssued
 ,\, policy ,\, gracesUnsettled ,\, time {\rangle}}}%
\@x{}\bottombar\@xx{}%
\setboolean{shading}{false}
\begin{lcom}{0}%
\begin{cpar}{0}{F}{F}{0}{0}{}%
\ensuremath{\.{\,\backslash\,}}* \ensuremath{Modification} History
\end{cpar}%
\begin{cpar}{0}{F}{F}{0}{0}{}%
 \ensuremath{\.{\,\backslash\,}}* Last modified \ensuremath{Tue}
 \ensuremath{Sep} 21 13:23:45 \ensuremath{PDT} 2021 by \ensuremath{ASUS
}%
\end{cpar}%
\begin{cpar}{0}{F}{F}{0}{0}{}%
 \ensuremath{\.{\,\backslash\,}}* Created \ensuremath{Wed} \ensuremath{Nov} 18
 17:58:19 \ensuremath{PST} 2020 by \ensuremath{ASUS
}%
\end{cpar}%
\end{lcom}%
\end{document}
